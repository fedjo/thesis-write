\chapter{Προτάσεις για μελλοντική έρευνα}\label{ch:conclusion}

\section{Concluding remarks}\label{sec:remarks}

This thesis described the design decisions, the technical-issues, and the
implementation details for replacing the Ganeti configuration and job queue
storage engine with CouchDB, a distributed document-oriented database. The goal
of this thesis was to examine if it was \emph{feasible} to integrate a NoSQL
database system in Ganeti, and measure the \emph{efficiency} of this approach.

After introducing the necessary theoretical background information about
Ganeti and all the related fields of interest, we presented the main drawbacks
that Ganeti appears, and the options that were evaluated to chose a NoSQL
system, and specifically CouchDB, to provide a solution to some of those issues.
We continued with the design and implementation details of the new storage
choice, and finally we evaluated our solution, in regard to two broad
dimensions: performance, and scalability potentials. Special attention was given
on being fully compliant with the current Ganeti requirements, such as
maintaining the fault-tolerant attribute by putting the safety of the data
first, security, and mainly not intervening with the parts of the Ganeti code
that are not related with our implementation.

Looking back at what we have created, we can say that we also covered another
need of Ganeti. The modular approach we followed for our design, supports both
CouchDB and file configuration as different storage engines, with different
limitations for each case. The current storage options can be easily extended
due to the transformation that we made to the base configuration modules, and we
could provide support for additional engines, such as MongoDB, or any other
option that would fit the requirements of Ganeti. As far as the implementation
complexity of the abstraction of some modules is concerned, we agree that we
introduced an important amount of transformations to the code, but taking into
account the extra features it introduces, and the fact that the code base is
quite clear and straightforward, we can state that the benefits are greater than
the cons, and the implementation can be consider \emph{successful}.

Benchmark results were quite promising. Both the storage engines have been
tested under various use cases and workloads. We measure the performance using
a variety of metrics which we believe that correspond to a real-world
environment. We have also studied the reasons that we consider responsible for
the differentiations that appeared in the performance of the two engines. Based
on this examination, we conclude that there are good indications that the NoSQL
approach will be able to give extra choices and provide additional support, to
the single storage solution which is currently used by Ganeti.

We admit that there are more factors we should consider before trying this new
approach to a real production environment. Since we are only in the first
version of the CouchDB driver, and since some of the limitations we presented
are fixed as of writing this chapter, and also considering the important
performance gains we observed, we strongly believe that this work can become the
basis for further work, and give the motivation for focusing on different
approaches for the Ganeti storage engine. A further integration of this design
in a demo environment, will help us gain the desirable feedback for more
improvements, and perhaps will provide us the basis for new ideas and feature
extensions.

\section{Future work}\label{sec:future}

The development of our implementation is far from over, as the tool has a lot of
room for further improvements. Our tasks for the future, on some of which we are
currently working on, will be classified into two separate categories. The
short-term goals, and the long-term ones, which are the following:

\subsection{Short-term plans}

This section, mainly contains improvements of the current implementation, and
the extension of the existing functionality.

\begin{description}
  \item[Fixing configuration ACID issue] \hfill \\
    In Section~\ref{sec:couch_details}, we discussed an issue that arisen from
    the transformation of the \texttt{config.data} file management. It is
    our primary priority dealing with this irregularity, and provide an
    efficient solution that will fix that issue, without affecting the overall
    performance of the tool.
  \item[Verification of the replication process] \hfill \\
    Currently, we do not make any kind of verification checks on the replication
    process. We do not have a way to ensure if a modification reached a majority
    of the candidate nodes, but we are based on the CouchDB server integrity.
    CouchDB will replicate, and all files will reach their destinations, as
    long as the server is running. We would like an external to Ganeti process,
    verify the status of the CouchDB instances, and warn the user for a possible
    failure. The current \texttt{ganeti-watcher} script, is a possible candidate
    that can be extended to provide the desired functionality, but more
    solutions can be discussed.
  \item[Improving the ssconf\_* management] \hfill \\
    Ganeti maintains the \texttt{ssconf\_*} set of files in all the nodes of the
    cluster. As a result, in an operation that modifies one or more of those
    files, the cost of applying the changes among the nodes of the cluster is
    aggregated. This cost is not negligible, and is one of the reasons that
    forbids Ganeti from scale. We could also add those files to the CouchDB
    server in a separate database called \texttt{ssconf}, and share this
    information only among the candidate nodes. Since, every node needs to have
    access to those files, we could give write permissions to the master node,
    while the rest nodes will only have read permissions to the database. It is
    an important performance fix which is intended to be applied soon.
  \item[Import all cluster information] \hfill \\
    Currently, CouchDB hosts only the job queue, the archive directory, and the
    \texttt{config.data} file. The rest Ganeti information, such as the SSL
    certificates, the daemon related keys, or the abovementioned
    \texttt{ssconf\_*} set of files, continues to be stored in the filesystem.
    We would like to import all those information in the CouchDB server, as
    well.
  \item[Backups for disaster-recovery] \hfill \\
    Ganeti keeps a backup for the configuration data file, as soon as a node is
    demoted from the master candidate role. CouchDB does not follow with this
    requirement. The code will be converted to follow that requirement, and
    subsequently it may be extended to keep flat backup files periodically, to
    add an extra layer of protection from hardware failures.
\end{description}

\subsection{Long-term plans}

This section contains our thoughts about extending Ganeti, improving its
overall performance, and its ability to scale better in bigger clusters.

\begin{description}
  \item[Improving candidate pool management] \hfill \\
    At this point, the CouchDB driver does not affect the management of the
    candidate nodes. Each node in the cluster has a running CouchDB server. As
    soon as a node is marked as candidate, the master node will extend its
    replication tasks to include this node as well. We are currently discussing
    another approach of managing the pool of candidate nodes. We could keep a
    set of hosts, even independent to Ganeti, that will be used for the
    candidate pool requirements only. These hosts will contain the live cluster
    configuration and the job queue information, which they will be shared. The
    master node will interact with those databases to store the cluster
    information, removing the requirement of maintaining the candidate pool
    exclusively inside Ganeti nodes. Using \emph{Solid State Drives (SSDs)}
    in that small set of nodes is more feasible than in a whole cluster, and it
    will be really handy for CouchDB.
  \item[Clustering support] \hfill \\
    The above idea can be amplified by the clustering techniques that the NoSQL
    systems provide. Since the NoSQL systems are designed to serve large data
    sets, some additional features are introduced to provide redundancy and
    high-availability in any case. Clustering and auto-sharding are some of
    those features, that provide the ability to the NoSQL systems to store and
    share data across multiple machines, using efficient algorithms for handling
    the read/write requests, and support the data growth demands. Currently,
    CouchDB supports clustering using external applications such as
    \emph{CouchDB Lounge}~\flink{http://tilgovi.github.io/couchdb-lounge/}. The
    merge of CouchDB that was announced with BigCouch, a Cloudant's clustered
    version of CouchDB, will bring soon all the clustering capabilities that
    currently CouchDB lacks of. Besides the facilitation in handling the set
    of candidate nodes, since they will be part of the cluster, with BigCouch
    merged in, CouchDB and Ganeti consequently will be able to replicate data at
    a much larger scale.
  \item[Extending storage choices] \hfill \\
    Abstracting the related to the storage management code of Ganeti, was our
    number one priority. We made it feasible to easily create additional storage
    engines for Ganeti. Moreover, the similarity among the NoSQL family that
    exists, can be exploited to design new driver solutions, such as
    \texttt{MongoDB}~\flink{http://www.mongodb.org/}, compare their performance,
    and generally having the ability to choose among several storage options
    according to our application's needs.
  \item[Improving lock congestion] \hfill \\
    The NoSQL systems are using their own locking policy. CouchDB uses the
    \texttt{MVCC} method to handle read and write requests. MongoDB
    uses a \texttt{readers-writer}, or \texttt{shared exclusive} per
    database lock, to deal with concurrent readers and writers. We could improve
    the current locking situation and be based on the locking level layer
    providing by the NoSQL systems. The \texttt{ConfigWriter}, and the queue
    level lock are the first contenders to be removed, since any serialization
    to accessing the configuration file and the job queue would be unnecessary.
\end{description}
