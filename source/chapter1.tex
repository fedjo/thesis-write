\chapter{Εισαγωγή}\label{ch:introduction}

Στις μέρες μας, η τεχνητή νοημοσύνη είναι ένας ταχύτατα αναπτυσσόμενος κλάδος
της επιστήμης των υπολογιστών. Η όραση υπολογιστών είναι ένα επιστημονικό πεδίο
της τεχνητής νοημοσύνης που δε θα μπορούσε να μείνει ανεπηρέαστο από αυτή την
εξέλιξη.

Η όραση των υπολογιστών (από και στο εξής ΟτΥ) ασχολείται με την απόκτηση,
ανάλυση και κατανόηση εικόνων, βίντεο και γενικά πολυμεσικού περιεχομένου
πολλών διαστάσεων από τον πραγματικό κόσμο. Έχει ως σκοπό να δώσει στα
υπολογιστικά συστήματα μια εποπτεία και κατανόηση του τετραδιάστατου
πραγματικού κόσμου.

Για να επιτευχθεί ο άνωθεν σκοπός χρειάζεται να αυτοματοποιηθεί μέσω μιας
υπολογιστικής αλγοριθμικής διαδικασίας η μέθοδος της ανθρώπινης όρασης. Έτσι
η πραγματική ν-διάσταση αναπαράσταση που απεικονίζει και αναγνωρίζει ο
ανθρώπινος εγκέφαλος, αναπαρίσταται με συμβολικό και αριθμητικό τρόπο.

Σε ένα υπολογιστικό σύστημα, μια εικόνα διαθέτει μια ψηφιακή αναπαράσταση. Στην πιο
απλή και συνιθισμένη της μορφή μιας δισδιάστατης εικόνας, αναπαρίσταται με ένα
ψηφιακό σήμα δυο διαστάσεων. Η τιμή του σήματος σε κάθε σημείο του επιπέδου
αφορά την τιμή του χρώματος της εικόνας στη θέση αυτή. Τα σημεία που αποτελούν το
σύνολο μια εικόνας είναι ευρέως γνωστά ως εικονοστοιχεία (pixels).

\section{Ανίχνευση προσώπων και αντικειμένων}
Η ανίχνευση προσώπων και γενικότερα αντικειμένων σε μια εικόνα συνίσταται στη
διαδικασία εύρεσης των χαρακτηριστικών εκείνων που καθιστούν ένα συγκεκριμένο
αντικείμενο μέλος μια κλάσης αντικειμένων. Είναι μια καθημερινή, αυτοματοποιημένη
και τετριμένη διαδικασία για τον άνθρωπο. Ο ανθρώπινος εγκέφαλος είναι
εκπαιδευμένος με τέτοιο τρόπο ώστε να μπορεί να αναγνωρίζει αντικείμενα ακαριαία.
Η αναγνώριση  αφορά την αναγνώριση μεμονομένων και συγκεκριμένων αντικειμένων
αλλά γενικότερα και την αναγνώριση της κλάσης η των κλάσεων στην/στις οποίες ανήκει
Αντίθετα, η διαδικασία αυτή δεν εκτελείται το ίδιο εύκολα και από ένα υπολογιστικό σύστημα.
Είναι απαραίτητη η εξαγωγή ενός μεγάλου πλήθους χαρακτηριστικών για την ταυτοποίηση
της κλάσης η πιθανών κλάσεων του αντικειμένου Ο υπολογισμός όλων αυτών των
χαρακτηριστικών είναι μια αρκετά κοστοβορα -σε κύκλους του επεξεργαστή διαδικασία
Για την επίτευξή της χρησιμοποιούνται τεχνικές της μηχανικής μάθησης και των
νευρωνικών δικτύων

\section{Αναγνώριση προσώπων}
Η αναγνωρη προσώπων αποτελει ενα επιπλεον σταδιο της ανιχνευσης ενος προσωπου,
οπου ενα εντοπισμενο προσωπο επιχειρειται να ταυτηστει με καποιο ηδη γνωστο προσωπο.
Ειναι μια διαδικασια που εκτελειται και παλι σχεδον ακαριαια απο τον ανθρωπινο εγκεφαλο
αλλα εν αντιθέση υπολογιστικά χρειαζεται αρκετη υπολογιστικη ισχυς και προεκπαιδευση
των συστηματων με τα προσωπα που θελει κανεις να αναγνωρισει.

\section{Συνεισφορά της διπλωματικής}

Στην παρούσα διπλωματική δημιουργήσαμε μια διαδικτυακη εφαρμογή η οποία μπορεί να παράγει
αυτόματες επισημάνσεις στο πολυμεσικό περιεχόμενο (video) που δέχεται ως
είσοδο. Οι τεχνικές για την εξαγωγή αυτών των επισημάνσεων χρησιμοποιούν
ως βαση state-of-art τεχνολογίες όσον αφορά την ανίχνευση και αναγνώρηση προσώπων
και κλασσεων αντικειμένων σε μια εικονα, προσαρμοσμενες στο περιβαλλον ενος βιντεο.
Επιπροσθετα, στο κομμάτι της αναγνώρισης προσώπων , επιχειρησαμε να βελτιωσουμε την ακριβεια
και την ταχυτητα εξαγωγης της προβλεψης, τροποποιωντας ενα μέρος της μεθοδου
Linear Binary Pattern Histograms.

Στο πλαίσιο της παρούσας διπλωματικής, επιχειρείται ,ως εκ τούτου, μια παρουσίαση
των υπάρχουσων τεχνικών για την αναγνώριση και ανίσχνευση προσώπων και αντικειμένων
και μια συγκριτική αξιολόγηση των τεχνικών για την αναγνώριση προσώπων.

\section{Οργάνωση κειμένου}

Το παρόν κείμενο έχει την εξής δομή:

\begin{description}
  \item[Κεφάλαιο~\ref{ch:facedetection}:] \hfill \\
    Στο κεφάλαιο αυτό θα κάνουμε μια γενική επισκόπηση των μεθόδων που
    χρησιμοποιούνται για την ανίχνευση κλάσεων αντικειμένων.
    Θα γίνει μια συνοπτική παρουσίαση του αλγορίθμου των Viola-Jones για την
    αναγνώριση προσώπων καθώς
    και ο τρόπος με τον οποίο τον χρησιμοποιήσαμε. Πρόκειται για τον αλγόριθμο
    πάνω στον οποίο βασίζεται η ανίχνευση προσώπων στις εικόνες και συνεπακόλουθα
    στα βίντεο. Θα συζητήσουμε πως σχετίζεται το αποτέλεσμα της μεθόδου με
    την μετέπειτα αναγνώριση των προσώπων.
  \item[Κεφάλαιο~\ref{ch:objectedetection}:] \hfill \\
    Θα παρουσιαστεί εδώ μια γενική επισκόπιση των νευρωνικών δικτύων. Θα συζητήσουμε
    για το Tensorflow framework το οποίο υποστηρίζει την παραγωγή νευρωνικών δικτύων
    και τέλος θα μιλήσουμε για το πως χρησιμοποιούμε το ανωτέρω νευρωνικών
    για να εξάγουμε σχολιασμούς για τις εικόνες και τα βίντεο.
  \item[Κεφάλαιο~\ref{ch:facerec}:] \hfill \\
    Γίνεται μια σύντομη παρουσίαση των μεθόδων που χρησιμοποιούνται για την
    αναγνώριση προσώπων. Θα αναλύσουμε επίσης και τη δική μας τεχνική αναγνώρισης
    και πως διαφοροποιείται από τις υπάρχουσες.
  \item[Κεφάλαιο~\ref{ch:results}:] \hfill \\
    Στο κομμάτι αυτό θα δούμε κάποια συγκριτικά αποτελέσματα των τεχνικών
    εξαγωγής σχολιασμών. Θα αναλυθούν τα σημεία και οι λόγοι διαφοροποίησης
    από άλλες μεθόδους και θα προταθούν τροποποιήσεις που μπορούν να οδηγήσουν
    σε ποιο εύστοχα αποτελέσματα.
  \item[Κεφάλαιο~\ref{ch:conclusion}:] \hfill \\
    Τέλος θα κάνουμε μια ανασκόπηση της διπλωματικής. Θα αναφερθούμε σε διάφορες
    άλλες τεχνικές και θα γίνει μια πρόταση για παραπέρα έρευνα
\end{description}
