\chapter{Εισαγωγή}\label{ch:introduction}

Στις μέρες μας, η τεχνητή νοημοσύνη είναι ένας ταχύτατα αναπτυσσόμενος κλάδος
της επιστήμης των υπολογιστών. Η όραση υπολογιστών είναι ένα επιστημονικό πεδίο
της τεχνητής νοημοσύνης που δε θα μπορούσε να μείνει ανεπηρέαστο από αυτή την
εξέλιξη.

Η όραση των υπολογιστών (από και στο εξής ΟτΥ) ασχολείται με την απόκτηση,
ανάλυση και κατανόηση εικόνων, βίντεο και γενικά πολυμεσικού περιεχομένου
πολλών διαστάσεων από τον πραγματικό κόσμο. Έχει ως σκοπό να δώσει στα
υπολογιστικά συστήματα μια εποπτεία και κατανόηση του τετραδιάστατου
πραγματικού κόσμου.

Για να επιτευχθεί ο άνωθεν σκοπός χρειάζεται να αυτοματοποιηθεί μέσω μιας
υπολογιστικής αλγοριθμικής διαδικασίας η μέθοδος της ανθρώπινης όρασης. Έτσι
η πραγματική ν-διάσταση αναπαράσταση που απεικονίζει και αναγνωρίζει ο
ανθρώπινος εγκέφαλος, αναπαρίσταται με συμβολικό και αριθμητικό τρόπο.

\section{Ανίχνευση προσώπων και αντικειμένων}
Η ανίχνευση προσώπων και γενικότερα αντικειμένων σε μια εικόνα είναι μια
καθημερινή, αυτοματοποιημένη και τετριμένη διαδικασία για τον άνθρωπο. Ο ανθρώπινος
εγκέφαλος είναι εκπαιδευμένος με τέτοιο τρόπο ώστε να μπορεί να αναγνωρίζει
αντικείμενα ακαριαία. Μάλιστα η αναγνώριση δεν αφορά δεν αφορά την αναγνώριση
μεμονομένων και συγκεκριμένων αντικειμένων αλλά γενικότερα την αναγνώριση
συγκεκριμένων κλάσεων αντικειμένων. Έτσι ο εγκέφαλός μας μπορεί να αναγνωρίζει
αντικέιμενα χωρίς προηγουμένως να τα έχει δει. Αντίθετα όμως η διαδικασία αυτή
δεν εκτελείται το ίδιο εύκολα και από ένα υπολογιστικό σύστημα.


\section{Αναγνώριση προσώπων}

\section{Συνεισφορά της διπλωματικής}

Στην παρούσα διπλωματική δημιουργήσαμε μια εφαρμογή η οποία μπορεί να παράγει
αυτόματες επισημάνσεις στο πολυμεσικό περιεχόμενο (video) που δέχεται ως
είσοδο. Οι τεχνικές για την εξαγωγή αυτών των επισημάνσεων χρησιμοποιούν
τις state-of-art τεχνολογίες όσον αφορά την ανίχνευση και αναγνώρηση προσώπων
και της ανίχνευσης των αντικειμένων. Επίσης στο κομμάτι της αναγνώρησης προσώπων
,πέρα από τις υπάρχουσες τεχνικές, αναπτύξαμε και μια καινούργια μέθοδο.

Παράλληλα με την εξαγωγή των επισημάνσεων γίνετα ,στο πλαίσιο της διπλωματικής,
και μια συγκριτική αξιολόγηση των τεχνικών για την αναγνώρηση προσώπων.

\section{Οργάνωση κειμένου}

Το παρόν κείμενο έχει την εξής δομή:

\begin{description}
  \item[Κεφάλαιο~\ref{ch:facedetection}:] \hfill \\
    Στο κεφάλαιο αυτό θα κάνουμε μια γενική επισκόπηση των μεθόδων που
    χρησιμοποιούνται για την ανίχνευση κλάσεων αντικειμένων.
    Θα γίνει μια συνοπτική παρουσίαση του αλγορίθμου των Viola-Jones για την
    αναγνώρηση προσώπων καθώς
    και ο τρόπος με τον οποίο τον χρησιμοποιήσαμε. Πρόκειται για τον αλγόριθμο
    πάνω στον οποίο βασίζεται η ανίχνευση προσώπων στις εικόνες και συνεπακόλουθα
    στα βίντεο. Θα συζητήσουμε πως σχετίζεται το αποτέλεσμα της μεθόδου με
    την μετέπειτα αναγνώριση των προσώπων.
  \item[Κεφάλαιο~\ref{ch:googlenet}:] \hfill \\
    Θα παρουσιαστεί εδώ μια γενική επισκόπιση των νευρωνικών δικτύων. Θα συζητήσουμε
    για το Caffe framework το οποίο υποστηρίζει την παραγωγή νευρωικών δικτύων
    και τέλος θα μιλήσουμε για το πως χρησιμοποιούμε το ανωτέρο νευρωνικών
    για να εξάγουμε σχολιασμούς για τις εικόνες και τα βίντεο.
  \item[Κεφάλαιο~\ref{ch:facerec}:] \hfill \\
    Γίνεται μια σύντομη παρουσίαση των μεθόδων που χρησιμοποιούνται για την
    αναγνώριση προσώπων. Θα αναλύσουμε επίσης και τη δική μας τεχνική αναγνώρισης
    και πως διαφοροποιείται από τις υπάρχουσες.
  \item[Κεφάλαιο~\ref{ch:results}:] \hfill \\
    Στο κομμάτι αυτό θα δούμε κάποια συγκριτικά αποτελέσματα των τεχνικών
    εξαγωγής σχολιασμών. Θα αναλυθούν τα σημεία και οι λόγοι διαφοροποίησης
    από άλλες μεθόδους και θα προταθούν τροποιήσεις που μπορούν να οδηγήσουν
    σε ποιο εύστοχα αποτελέσματα.
  \item[Κεφάλαιο~\ref{ch:conclusion}:] \hfill \\
    Τέλος θα κάνουμε μια ανασκόπηση της διπλωματικής. Θα αναφερθούμε σε διάφορες
    άλλες τεχνικές και θα γίνει μια πρόταση για παραπέρα έρευνα
\end{description}
