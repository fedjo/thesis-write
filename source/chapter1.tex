\chapter{Εισαγωγή}\label{ch:introduction}

Στις μέρες μας, η τεχνητή νοημοσύνη είναι ένας ταχύτατα αναπτυσσόμενος κλάδος
της επιστήμης των υπολογιστών. Η όραση υπολογιστών είναι ένα επιστημονικό πεδίο
της τεχνητής νοημοσύνης που δε θα μπορούσε να μείνει ανεπηρέαστο από αυτή την
εξέλιξη.

Η όραση των υπολογιστών (από και στο εξής ΟτΥ) ασχολείται με την απόκτηση,
ανάλυση και κατανόηση εικόνων, βίντεο και γενικά πολυμεσικού περιεχομένου
πολλών διαστάσεων από τον πραγματικό κόσμο. Έχει ως σκοπό να δώσει στα
υπολογιστικά συστήματα μια εποπτεία και κατανόηση του τετραδιάστατου
πραγματικού κόσμου.

Για να επιτευχθεί ο άνωθεν σκοπός χρειάζεται να αυτοματοποιηθεί μέσω μιας
υπολογιστικής αλγοριθμικής διαδικασίας η μέθοδος της ανθρώπινης όρασης. Έτσι
η πραγματική ν-διάσταση αναπαράσταση που απεικονίζει και αναγνωρίζει ο
ανθρώπινος εγκέφαλος, αναπαρίσταται με συμβολικό και αριθμητικό τρόπο.

\section{Ανίχνευση προσώπων και αντικειμένων}
Η ανίχνευση προσώπων και γενικότερα αντικειμένων σε μια εικόνα είναι μια
καθημερινή, αυτοματοποιημένη και τετριμένη διαδικασία για τον άνθρωπο. Ο ανθρώπινος
εγκέφαλος είναι εκπαιδευμένος με τέτοιο τρόπο ώστε να μπορεί να αναγνωρίζει
αντικείμενα ακαριαία. Μάλιστα η αναγνώριση δεν αφορά δεν αφορά την αναγνώριση
μεμονομένων και συγκεκριμένων αντικειμένων αλλά γενικότερα την αναγνώριση
συγκεκριμένων κλάσεων αντικειμένων. Έτσι ο εγκέφαλός μας μπορεί να αναγνωρίζει
αντικέιμενα χωρίς προηγουμένως να τα έχει δει. Αντίθετα όμως η διαδικασία αυτή
δεν εκτελείται το ίδιο εύκολα και από ένα υπολογιστικό σύστημα.


\section{Αναγνώριση προσώπων}

\section{Συνεισφορά της διπλωματικής}

In this thesis we will design, and study the performance impact of integrating a
NoSQL database in a software used for managing clusters of physical nodes. The
motivation behind this thesis emerged from concerns about the performance, and
scalability requirements of \textbf{Ganeti}
\flink{http://code.google.com/p/ganeti/}, a software tool used for the physical
node management of a cluster, and the low level VM management as well. Ganeti is
used from \textbf{Synnefo}~\flink{http://www.synnefo.org}, an open source cloud
software used to create massively scalable IaaS clouds. \textbf{Synnefo}
\cite{synnefo}, powers the \textbf{\raise.17ex\hbox{$\scriptstyle\sim$}okeanos}
public cloud service~\cite{okeanos}.
\textbf{\raise.17ex\hbox{$\scriptstyle\sim$}okeanos} is an \emph{IaaS}, i.e.,
\emph{Infrastructure as a Service}, that provides virtual machines, virtual
networks and storage services to the Greek Academic and Research Community. It
is an open-source service that has been running in production servers since
2011, by GRNET S.A.~\flink{https://www.grnet.gr/}.

Synnefo is a complete open source cloud stack written in Python, and has three
main components providing the corresponding services:

\begin{itemize}
  \item \textbf{\emph{Cyclades}}, Compute/Network/Image/Volume services.
  \item \textbf{\emph{Pithos}}, File/Object Storage services.
  \item \textbf{\emph{Astakos}}, Identify/Account services.
\end{itemize}

Synnefo manages multiple Ganeti clusters at the backend for handling the
low-level VM operations. As we mentioned previously, improving the performance
and scalability of Ganeti, by testing it under alternative storage engines, and
specifically \emph{CouchDB}~\flink{http://couchdb.apache.org/}, a NoSQL database
system, was our motivation. In addition, a design document
\flink{https://groups.google.com/forum/\#!topic/ganeti-devel/jLvStCCTZ2Q}, that
was proposed a long time ago by \emph{Guido Trotter}, one of Ganeti's senior
Engineers, amplified the conduction of this thesis.

\section{Οργάνωση κειμένου}

This thesis is organized in the following sections:

\begin{description}
  \item[Κεφάλαιο~\ref{ch:objdetection}:] \hfill \\
    Στο κεφάλαιο αυτό θα κάνουμε μια γενική επισκόπηση των μεθόδων που
    χρησιμοποιούνται για την ανίχνευση αντικειμένων.
  \item[Κεφάλαιο~\ref{ch:violajones}:] \hfill \\
    Εδώ θα γίνει μια συνοπτική παρουσίαση του ανωτέρου αλγορίθμου καθώς
    και ο τρόπος με τον οποίο τον χρησιμοποιήσαμε. Πρόκειται για τον αλγόριθμο
    πάνω στον οποίο βασίζεται η ανίχνευση προσώπων στις εικόνες και συνεπακόλουθα
    στα βίντεο. Θα συζητήσουμε πως σχετίζεται το αποτέλεσμα της μεθόδου με
    την μετέπειτα αναγνώριση των προσώπων.
  \item[Κεφάλαιο~\ref{ch:googlenet}:] \hfill \\
    Θα παρουσιαστεί εδώ μια γενική επισκόπιση των νευρωνικών δικτύων. Θα συζητήσουμε
    για το Caffe framework το οποίο υποστηρίζει την παραγωγή νευρωικών δικτύων
    και τέλος θα μιλήσουμε για το πως χρησιμοποιούμε το ανωτέρο νευρωνικών
    για να εξάγουμε σχολιασμούς για τις εικόνες και τα βίντεο.
  \item[Κεφάλαιο~\ref{ch:facerec}:] \hfill \\
    Γίνεται μια σύντομη παρουσίαση των μεθόδων που χρησιμοποιούνται για την
    αναγνώριση προσώπων. Θα αναλύσουμε επίσης και τη δική μας τεχνική αναγνώρισης
    και πως διαφοροποιείται από τις υπάρχουσες.
  \item[Κεφάλαιο~\ref{ch:results}:] \hfill \\
    Στο κομμάτι αυτό θα δούμε κάποια συγκριτικά αποτελέσματα των τεχνικών
    εξαγωγής σχολιασμών. Θα αναλυθούν τα σημεία και οι λόγοι διαφοροποίησης
    από άλλες μεθόδους και θα προταθούν τροποιήσεις που μπορούν να οδηγήσουν
    σε ποιο εύστοχα αποτελέσματα.
  \item[Κεφάλαιο~\ref{ch:conclusion}:] \hfill \\
    Τέλος θα κάνουμε μια ανασκόπηση της διπλωματικής. Θα αναφερθούμε σε διάφορες
    άλλες τεχνικές και θα γίνει μια πρόταση για παραπέρα έρευνα
\end{description}
