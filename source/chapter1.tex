\chapter{Εισαγωγή}\label{ch:introduction}

Στις μέρες μας, η τεχνητή νοημοσύνη είναι ένας ταχύτατα αναπτυσσόμενος κλάδος
της επιστήμης των υπολογιστών. Η όραση υπολογιστών είναι ένα επιστημονικό πεδίο
της τεχνητής νοημοσύνης που δε θα μπορούσε να μείνει ανεπηρέαστο από αυτή την
εξέλιξη.

Η όραση των υπολογιστών (από και στο εξής ΟτΥ) ασχολείται με την απόκτηση,
ανάλυση και κατανόηση εικόνων, βίντεο και γενικά πολυμεσικού περιεχομένου
πολλών διαστάσεων από τον πραγματικό κόσμο. Έχει ως σκοπό να δώσει στα
υπολογιστικά συστήματα μια εποπτεία και κατανόηση του τετραδιάστατου
πραγματικού κόσμου.

Για να επιτευχθεί ο άνωθεν σκοπός χρειάζεται να αυτοματοποιηθεί μέσω μιας
υπολογιστικής αλγοριθμικής διαδικασίας η μέθοδος της ανθρώπινης όρασης. Έτσι
η πραγματική ν-διάσταση αναπαράσταση που απεικονίζει και αναγνωρίζει ο
ανθρώπινος εγκέφαλος, αναπαρίσταται με συμβολικό και αριθμητικό τρόπο.

Σε ένα υπολογιστικό σύστημα, μια εικόνα διαθέτει μια ψηφιακή αναπαράσταση. Στην πιο
απλή και συνηθισμένη της μορφή μιας δισδιάστατης εικόνας, αναπαρίσταται με ένα
ψηφιακό σήμα δυο διαστάσεων. Η τιμή του σήματος σε κάθε σημείο του επιπέδου
αφορά την τιμή του χρώματος της εικόνας στη θέση αυτή. Τα σημεία που αποτελούν το
σύνολο μια εικόνας είναι ευρέως γνωστά ως εικονοστοιχεία (pixels).

\section{Ανίχνευση προσώπων και αντικειμένων}
Η ανίχνευση προσώπων και γενικότερα αντικειμένων σε μια εικόνα συνίσταται στη
διαδικασία εύρεσης των χαρακτηριστικών εκείνων που καθιστούν ένα συγκεκριμένο
αντικείμενο μέλος μια κλάσης αντικειμένων. Είναι μια καθημερινή, αυτοματοποιημένη
και τετριμμένη διαδικασία για τον άνθρωπο. Ο ανθρώπινος εγκέφαλος είναι
εκπαιδευμένος με τέτοιο τρόπο ώστε να μπορεί να αναγνωρίζει αντικείμενα ακαριαία.
Η αναγνώριση  αφορά την αναγνώριση μεμονωμένων και συγκεκριμένων αντικειμένων
αλλά γενικότερα και την αναγνώριση της κλάσης η των κλάσεων στην/στις οποίες ανήκει
Αντίθετα, η διαδικασία αυτή δεν εκτελείται το ίδιο εύκολα και από ένα υπολογιστικό σύστημα.
Είναι απαραίτητη η εξαγωγή ενός μεγάλου πλήθους χαρακτηριστικών για την ταυτοποίηση
της κλάσης ή πιθανών κλάσεων του αντικειμένου. Ο υπολογισμός όλων αυτών των
χαρακτηριστικών είναι μια αρκετά χρονοβόρα -σε κύκλους του επεξεργαστή- διαδικασία.
Για την επίτευξή της γίνεται χρήση τεχνικών μηχανικής μάθησης και των νευρωνικών δικτύων

\section{Αναγνώριση προσώπων}
Η αναγνώριση προσώπων αποτελεί ένα επιπλέον στάδιο της ανίχνευσης ενός προσώπου,
όπου ένα εντοπισμένο πρόσωπο επιχειρείται να ταυτιστεί με κάποιο ήδη γνωστό πρόσωπο
Είναι μια διαδικασία που εκτελείται και πάλι σχεδόν ακαριαία από τον ανθρώπινο εγκέφαλο
αλλά εν αντίθεση υπολογιστικά χρειάζεται αρκετή υπολογιστική ισχύς και προεκπαιδευση
των συστημάτων με τα πρόσωπα που θέλει κανείς να αναγνωρίσει

\section{Συνεισφορά της διπλωματικής}

Στην παρούσα διπλωματική δημιουργήσαμε μια διαδικτυακή εφαρμογή η οποία μπορεί να παράγει
αυτόματες επισημάνσεις στο πολυμεσικό περιεχόμενο (video) που δέχεται ως
είσοδο. Οι τεχνικές για την εξαγωγή αυτών των επισημάνσεων χρησιμοποιούν
ως βάση state-of-art τεχνολογίες όσον αφορά την ανίχνευση και αναγνώριση προσώπων
και κλάσεων αντικειμένων σε μια εικόνα, προσαρμοσμένες στο περιβάλλον ενός βίντεο
Επιπρόσθετα, στο κομμάτι της αναγνώρισης προσώπων , επιχειρήσαμε να βελτιώσουμε την ακρίβεια
και την ταχύτητα εξαγωγής της πρόβλεψης, τροποποιώντας ένα μέρος της μεθόδου
Linear Binary Pattern Histograms.

Στο πλαίσιο της παρούσας διπλωματικής, επιχειρείται ,ως εκ τούτου, μια παρουσίαση
των υπάρχουσων τεχνικών για την αναγνώριση και ανίχνευση προσώπων και αντικειμένων
και μια συγκριτική αξιολόγηση των τεχνικών για την αναγνώριση προσώπων.

\section{Οργάνωση κειμένου}

Το παρόν κείμενο έχει την εξής δομή:

\begin{description}
  \item[Κεφάλαιο~\ref{ch:facedetection}:] \hfill \\
    Στο κεφάλαιο αυτό θα κάνουμε μια γενική επισκόπηση των μεθόδων που
    χρησιμοποιούνται για την ανίχνευση κλάσεων αντικειμένων.
    Θα γίνει μια παρουσίαση του αλγορίθμου των Viola-Jones.
    και ο τρόπος με τον οποίο τον χρησιμοποιήσαμε. Πρόκειται για τον αλγόριθμο
    πάνω στον οποίο βασίζεται η ανίχνευση προσώπων στις εικόνες και συνεπακόλουθα
    στα βίντεο. Θα συζητήσουμε πως σχετίζεται το αποτέλεσμα της μεθόδου με
    την μετέπειτα αναγνώριση των προσώπων.
  \item[Κεφάλαιο~\ref{ch:objectdetection}:] \hfill \\
    Θα κάνουμε μια γενική επισκόπηση των σύγχρονων τεχνικών ανίχνευσης κλάσεων αντικειμένων
    με τη χρήση νευρωνικών δικτύων. Θα συγκρίνουμε τα μεταξύ τους πλεονεκτήματα
    και μειονεκτήματα, ενώ τέλος θα μιλήσουμε αναλυτικότερα για τους ανιχνευτές μονής
    λήψης και τους λόγους που επιλέχθηκαν να χρησιμοποιηθούν στην υλοποίηση της
    παρούσας διπλωματικής.
  \item[Κεφάλαιο~\ref{ch:facerec}:] \hfill \\
    Θα παρουσιάσουμε τις μεθόδους που χρησιμοποιούνται για την
    αναγνώριση προσώπων. Θα αναλύσουμε τον τρόπο και τους λόγους για τον οποίο
    επιλέξαμε να τροποποιήσουμε μία από αυτές.
  \item[Κεφάλαιο~\ref{ch:results}:] \hfill \\
    Στο κομμάτι αυτό θα δούμε κάποια συγκριτικά αποτελέσματα των τεχνικών
    αναγνώρισης προσώπων. Θα αναλύσουμε τα σημεία και τους λόγους διαφοροποίησης
    από άλλες μεθόδους και θα προταθούν τροποποιήσεις που πιθανόν να οδηγήσουν
    σε ποιο εύστοχα αποτελέσματα.
  \item[Κεφάλαιο~\ref{ch:conclusion}:] \hfill \\
    Τέλος θα κάνουμε μια ανασκόπηση της διπλωματικής. Θα αναφερθούμε σε διάφορες
    άλλες τεχνικές και θα γίνει μια πρόταση για παραπέρα έρευνα
\end{description}
