\chapter{Εισαγωγή}\label{ch:introduction}

Στις μέρες μας, η τεχνητή νοημοσύνη είναι ένας ταχύτατα αναπτυσσόμενος κλάδος
της επιστήμης των υπολογιστών. Η όραση υπολογιστών είναι ένα επιστημονικό πεδίο
της τεχνητής νοημοσύνης που δε θα μπορούσε να μείνει ανεπηρέαστο από αυτή την
εξέλιξη.

Η όραση των υπολογιστών (από και στο εξής ΟτΥ) ασχολείται με την απόκτηση,
ανάλυση και κατανόηση εικόνων, βίντεο και γενικά πολυμεσικού περιεχομένου
πολλών διαστάσεων από τον πραγματικό κόσμο. Έχει ως σκοπό να δώσει στα
υπολογιστικά συστήματα μια εποπτεία και κατανόηση του τετραδιάστατου
πραγματικού κόσμου.

Για να επιτευχθεί ο άνωθεν σκοπός χρειάζεται να αυτοματοποιηθεί μέσω μιας
υπολογιστικής αλγοριθμικής διαδικασίας η μέθοδος της ανθρώπινης όρασης. Έτσι
η πραγματική ν-διάσταση αναπαράσταση που απεικονίζει και αναγνωρίζει ο
ανθρώπινος εγκέφαλος, αναπαρίσταται με συμβολικό και αριθμητικό τρόπο.


\section{Thesis motivation}

In this thesis we will design, and study the performance impact of integrating a
NoSQL database in a software used for managing clusters of physical nodes. The
motivation behind this thesis emerged from concerns about the performance, and
scalability requirements of \textbf{Ganeti}
\flink{http://code.google.com/p/ganeti/}, a software tool used for the physical
node management of a cluster, and the low level VM management as well. Ganeti is
used from \textbf{Synnefo}~\flink{http://www.synnefo.org}, an open source cloud
software used to create massively scalable IaaS clouds. \textbf{Synnefo}
\cite{synnefo}, powers the \textbf{\raise.17ex\hbox{$\scriptstyle\sim$}okeanos}
public cloud service~\cite{okeanos}.
\textbf{\raise.17ex\hbox{$\scriptstyle\sim$}okeanos} is an \emph{IaaS}, i.e.,
\emph{Infrastructure as a Service}, that provides virtual machines, virtual
networks and storage services to the Greek Academic and Research Community. It
is an open-source service that has been running in production servers since
2011, by GRNET S.A.~\flink{https://www.grnet.gr/}.

Synnefo is a complete open source cloud stack written in Python, and has three
main components providing the corresponding services:

\begin{itemize}
  \item \textbf{\emph{Cyclades}}, Compute/Network/Image/Volume services.
  \item \textbf{\emph{Pithos}}, File/Object Storage services.
  \item \textbf{\emph{Astakos}}, Identify/Account services.
\end{itemize}

Synnefo manages multiple Ganeti clusters at the backend for handling the
low-level VM operations. As we mentioned previously, improving the performance
and scalability of Ganeti, by testing it under alternative storage engines, and
specifically \emph{CouchDB}~\flink{http://couchdb.apache.org/}, a NoSQL database
system, was our motivation. In addition, a design document
\flink{https://groups.google.com/forum/\#!topic/ganeti-devel/jLvStCCTZ2Q}, that
was proposed a long time ago by \emph{Guido Trotter}, one of Ganeti's senior
Engineers, amplified the conduction of this thesis.

\section{Thesis structure}

This thesis is organized in the following sections:

\begin{description}
  \item[Chapter~\ref{ch:background}:] \hfill \\
    We provide all the necessary theoretical background for the concepts
    discussed in this thesis.
  \item[Chapter~\ref{ch:ganeti}:] \hfill \\
    We present the architecture of Ganeti. A small documentation to facilitate
    the reader with the most basic components of Ganeti, and how they interact
    with each other.
  \item[Chapter~\ref{ch:backend}:] \hfill \\
    We analyze two of the basic components of Ganeti by providing more technical
    details, and specifically the configuration data file and the job queue
    components. We also examine the main factors that prevent Ganeti from
    achieving better performance, and reduce its scalability capabilities. Next,
    we discuss the main reasons we chosen CouchDB to provide solution to some of
    those issues, and we make a quick presentation of its main features.
    Finally, we explain in details the design of the CouchDB driver, along with
    all the compromises we have to make during the implementation.
  \item[Chapter~\ref{ch:performance}:] \hfill \\
    We evaluate the performance of the CouchDB driver. We compare it with the
    Ganeti's default filesystem approach, and we weight the pros and cons of
    each implementation. Finally, we extensively explain the reasons behind the
    differentiations that are arisen.
  \item[Chapter~\ref{ch:conclusion}:] \hfill \\
    We provide our conclusion remarks, and our thoughts about future work for
    further improvements on the tool we designed, along with the final deduction
    of our work.
\end{description}
