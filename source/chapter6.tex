\chapter{Επίλογος}\label{ch:conclusion}

\section{Συμπεράσματα}\label{sec:remarks}

Αυτό που εύκολα παρατηρεί κανείς είναι το γεγονός ότι η χρήσης της  απόσταση
\emph{ChiSquare} για τη σύγκριση των ιστογραμμάτων στη μέθοδό μας φαίνεται να
δίνει μεγαλύτερη ακρίβεια για όλες της βάσεις προσώπων που χρησιμοποιήσαμε.

Παρατηρούμε επίσης μια σημαντική διαφορά στην ακρίβεια της μεθόδου στη βάσε MyLucce.
Το γεγονός αυτό οφείλεται στο ότι η βάση MyLucce δημιουργήθηκε παίρνοντας τις εικόνες
των προσώπων από πραγματικό πολυμεσικό υλικό (βίντεο). Ως αποτέλεσμα δεν είχαμε καθόλου
επιρροή στις συνθήκες φωτισμού, στις εκφράσεις των προσώπων και στα χαρακτηριστικά
σε αντίθεση με τις άλλες βάσεις οι οποίες σχεδιάστηκαν ειδικά για σκοπούς πειραματισμού.
Θα μπορούσαμε όμως να θεωρήσουμε ότι η βάση MyLucce μας δίνει μια εικόνα για την
ακρίβεια της μεθόδου όταν προσπαθούμε να κάνουμε αναγνώριση χωρίς να έχουμε ένα
προεκπαιδευμένο σύστημα (Unsupervised Learning\flink{https://en.wikipedia.org/wiki/Unsupervised_learning})

Όσος αφορά την αξιολόγηση της τροποποιημένης μεθόδου, παρατηρούμε μια διαφοροποίηση
στη συμπεριφορά της ανάμεσα στη βάση AT\&Τ και τις υπόλοιπες. Στη βάση ΑΤ\&Τ φαίνεται
πως οποιαδήποτε αύξηση της τιμής του \emph{k} μειώνει την ακρίβειά της συνολικά.
Όμως όπως προαναφέραμε, λόγω των συνθηκών δημιουργίας της, η βάση ΑΤ\&Τ είναι
χρήσιμη για κάποια αρχικά πειράματα αλλά στην ουσία θεωρείται μια 'εύκολη' σχετικά
βάση. Επομένως δεν προσφέρεται για ασφαλή συμπεράσματα. Αντίθετα στις βάσεις Yale A
και Yale B ακόμη και σε ένα βαθμό και στην MyLucce παρατηρούμε ότι η αύξηση της τιμής
του \emph{k} δαιτηρεί σταθερή την ακρίβεια ή ακόμη και την αυξάνει μέχρι ένα σημείο
καμπής όπου από εκεί και πέρα η ακρίβεια της μεθόδου πέφτει δραματικά. Είναι λοιπόν
εμφανές ότι αρκετές φορές πρέπει να συμπεριλάβουμε περισσότερους από έναν κοντινότερους
γείτονες στη πρόβλεψη για να έχουμε ένα καλύτερο αποτέλεσμα.

Τέλος στα τελευταία γραφήματα επιχειρήσαμε και μια σύγκριση της μεθόδου με τη μέθοδο
FisherFaces ή οποία δεν βασίζεται στα local binary pattern histograms αλλά στην
Linear Discriminant Analysis \ref{sec:fisher}.
Η συμπεριφορά της FisherFaces είναι ίδια όσο αυξάνεται το \emph{k}. Όμως η ακρίβειά
της φθίνει πολύ πιο γρήγορα από εκείνη της μεθόδου μας. Αντίθετα,  στα μικρά \emph{k} η
ακρίβεια είναι σχετικά ίδια και για τις δύο μεθόδους.


Συμπερασματικά, θα λέγαμε ότι η τροποποίηση που πραγματοποιήσαμε εδώ κάποια ενδιαφέροντα
στοιχεία. Το βασικότερο είναι ότι η τιμή του \emph{k} όντως επηρεάζει την ακρίβεια
της μεθόδου και κατά περιπτώσεις την αυξάνει, μέχρι ένα σημείο καμπής. Γίνεται
λοιπόν εμφανές ότι αν θέλουμε να μειώσουμε τα false positives της μεθόδου LBPH πρέπει
η τελική πρόβλεψη να μην βασίζεται μόνο στον κοντινότερο γείτονα αλλά και στους υπολοίπους.
Πρέπει να γίνεται ξεκάθαρο σε κάθε βάση δεδομένων πιο είναι το σημείο καμπής της
όπου από κει και πέρα η ακρίβεια πέφτει δραματικά. Αυτό εξαρτάται από της συνθήκες
των προσώπων της κάθε βάσης. Και τέλος να σημειώσουμε το εύρος της διαφοράς στην ακρίβεια
που παρουσιάζει η αναγνώριση με μια βάση κατασκευασμένη σε ελεγχόμενες συνθήκες
και μια βάση δημιουργημένη με μη ελεγχόμενες.


\section{Προτάσεις για μελλοντική έρευνα}\label{sec:future}

Η ανίχνευση προσώπων και αντικειμένων καθώς και η αναγνώριση προσώπων, είναι
προβλήματα που απασχολούν ιδιαίτερα την επιστημονική κοινότητα τα τελευταία χρόνια,
με μεθόδους και τεχνικές να εμφανίζονται με εξαιρετικά μεγάλο ρυθμό. Ήδη από την
στιγμή που αρχίσαμε την εκπόνηση της εργασίας μέχρι σήμερα έχουν γίνει εξαιρετικά
βήματα προς την επίτευξη καλύτερων και πιο γρήγορων ανιχνεύσεων.

Ειδικότερα, οι σύγχρονες τεχνικές βασίζονται πλέον στα νευρωνικά δίκτυα. Αρχικά
χρησιμοποιείται ένα νευρωνικό δίκτυο για την εξαγωγή υποψήφιων θέσεων ενός
αντικειμένου και στη συνέχεια οι θέσεις αυτές ταξινομούνται στην κατάλληλη κλάση
και εξάγεται το αντίστοιχο παράθυρο με τη χρήση κάποιου άλλου συνελικτικού
κυρίως νευρωνικού δικτύου. Τα τελευταία χρόνια εμφανίστηκε η τάση της εξαγωγής
της κλάσης και του παραθύρου ταυτόχρονα με τον προσδιορισμό των υποψήφιων θέσεων
των αντικειμένων. Οι ανιχνευτές μονής λήψης όπως ονομάζονται φαίνεται να δίνουν
της κατεύθυνση της ανίχνευσης αντικειμένων στο μέλλον καθώς παρουσιάζουν αρκετά
καλή ακρίβεια σε μικρό χρόνο εκτέλεσης.

Παράλληλα, λόγω της αυξημένης χρήσης συσκευών περιορισμένης επεξεργαστικής ισχύς
εμφανίζεται η ανάγκη περιορισμού των απαιτήσεων που χρειάζονται οι παραπάνω μέθοδοι
τόσο σε μνήμη όσο και σε επεξεργαστική ισχύ. Το νευρωνικό δίκτυο MobileNet σχεδιάστηκε
με τέτοιο τρόπο ώστε η μέθοδος ανίχνευσης αντικειμένου που το χρησιμοποιεί να μπορεί
να τρέξει με μια συσκευή κινητής τηλεφωνίας. Επομένως φαίνεται να πως ενώ στον τομέα
της ακρίβειας του αποτελέσματος της ανίχνευσης έχει επιτευχθεί ένα ικανοποιητικό
ποσοστό, το μέλλον προϋποθέτει την προσαρμογή των μεθόδων σε συσκευές με μικρότερες
δυνατότητες όπως κινητά τηλέφωνα και tablets με τη χρήση 'ελαφριών' νευρωνικών
δικτύων.
