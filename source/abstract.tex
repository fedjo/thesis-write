%% Greek Abstract
\begin{greek}
\begin{abstract}

Στις μέρες μας, οι υποδομές Cloud Computing προσφέρουν ευελιξία, διαφάνεια, και
ασφάλεια για την εκτέλεση ενός συνεχώς αυξανόμενου πλήθους εφαρμογών και
υπηρεσιών. Οι υποδομές αυτές αποτελούνται κατά κανόνα από συστοιχίες υπολογιστών
(clusters), χρησιμοποιώντας τεχνικές εικονοποίησης για να διαμοιράσουν τους
φυσικούς πόρους σε εικονικούς, οι οποίοι θα χρησιμοποιηθούν από το cloud
περιβάλλον. Οι πάροχοι των cloud υπηρεσιών, θέλουν οι υπηρεσίες τους να έχουν
δυνατότητες κλιμάκωσης (scaling), και να λειτουργούν με χαμηλούς χρόνους
απόκρισης (latency), ανεξαρτήτως του φόρτου των υπηρεσιών τους. Αρκετοί
παράγοντες επηρεάζουν την απόδοση των cloud περιβαλλόντων, όπως το δίκτυο που
χρησιμοποιείται για στη διασύνδεση των φυσικών πόρων, ή το υλικό που
χρησιμοποιήθηκε για την υποδομή, όπως η CPU, η μνήμη, και ο δίσκος. Συνήθως,
κάποιο εργαλείο λογισμικού αναλαμβάνει τη διαχείριση των κόμβων της συστοιχίας
των υπολογιστών, όπως και την διαχείριση των εικονικών πόρων. Η παρούσα
διπλωματική στοχεύει στη βελτίωση της απόδοσης ενός τέτοιου λογισμικού, και
συγκεκριμένα του Ganeti, παρέχοντας υποστήριξη για εναλλακτικές μεθόδους που θα
εξυπηρετούν τις απαιτήσεις του εργαλείου σε αποθηκευτικό χώρο. Η υλοποίησή μας,
ενσωματώνει την CouchDB, μία NoSQL βάση διαχείρισης δεδομένων, χωρίς σχήμα, και
προσανατολισμένη γύρω από έγγραφα στο Ganeti, και αξιολογεί την απόδοση του
λογισμικού μετά από αυτή την τροποποίηση. Οι πρώτες μετρήσεις είναι ιδιαίτερα
ενθαρρυντικές, καθώς παρουσιάζουν εμφανή βελτίωση στην απόδοση του Ganeti. Οι
λόγοι αυτής της βελτίωσης θα παρουσιασθούν λεπτομερώς στη συνέχεια της παρούσας
διπλωματικής.

\begin{keywords}
cloud computing, cloud, εικονοποίηση, εικονική μηχανή, NoSQL, Ganeti, JSON,
CouchDB, Synnefo, okeanos, διαμοιρασμός, python, κλιμάκωση (scaling), απόδοση
(throughput), συστοιχία (cluster), κόμβος, instance, qemu, KVM, module, πακέτο
δομή b+δέντρου, MVCC, ACID, CAP, views, daemons
\end{keywords}

\end{abstract}
\end{greek}

%% English Abstract
\begin{abstracteng}

Nowadays, cloud computing exhibits agility, transparency, and security to the
execution of a continuously increasing number of applications and services.
Those infrastructures are designed on top of clusters of physical nodes, using
virtualization techniques to appropriately separate the physical resources to
create virtual dedicated ones, which will power the cloud environment. Cloud
providers want their applications have the ability to scale, and operate in
low-time latency, regardless of the load of the cloud services. Many factors
affect the performance of those environments such as the network that is used
for the intra-cluster communication, or the underlying hardware resources used,
in terms of CPU, memory, and disk i/o. A software tool is commonly used that
manages the physical nodes of the cluster, and the virtual resources as well.
This thesis aims to improve the performance of such a tool, and specifically
Ganeti's, by providing support for alternative engines to serve its storage
requirements. Our design integrates CouchDB, a NoSQL, schema-less, and document
oriented database in Ganeti, and evaluates the performance of the tool under
the new storage layer. Early performance evaluations look very promising and
show a noteworthy speedup on the performance of Ganeti, that will be discussed
in details in the rest of the document.

\begin{keywordseng}
cloud computing, cloud, virtualization, virtual machine, NoSQL, Ganeti, JSON,
CouchDB, Synnefo, okeanos, replication, python, scaling, throughput, cluster,
node, instance, qemu, KVM, module, package, b+tree structure, MVCC, ACID, CAP,
views, daemons
\end{keywordseng}

\end{abstracteng}

%% Greek Acknowledgements
\begin{greek}
\begin{acknowledgements}

Αρχικά, θα ήθελα να ευχαριστήσω τον αναπληρωτή καθηγητή κ. Νεκτάριο Κοζύρη για
την ευκαιρία που μου έδωσε να ασχοληθώ με το συγκεκριμένο τομέα της επιστήμης
των υπολογιστών στο Εργαστήριο Υπολογιστικών Συστημάτων, καθώς και για τη θετική
συμβολή του καθόλη τη διάρκεια των σπουδών μου.

Η εκπόνηση της διπλωματικής αυτής εργασίας, αποτελεί έμπνευση του διδάκτορα
Ευάγγελου Κούκη, τον οποίο θα ήθελα να ευχαριστήσω θερμά τόσο για την βοήθειά
του και τις συμβουλές του σε επίπεδο σχεδιασμού και υλοποίσης, όσο και για τις
τεχνικές γνώσεις που μου μετέδωσε κατά τη διάρκειά της. Επίσης θα ήθελα να
ευχαριστήσω ιδιαίτερα τον Χρήστο Σταυρακάκη για την συμβολή του και τις
παρεμβάσεις του που βοήθησαν στην επίτευξη της διπλωματικής μου εργασίας.

Τέλος ένα μεγάλο ευχαριστώ στην οικογένειά μου, για τη συνεχή τους στήριξη όλο
αυτό το διάστημα των σπουδών μου, καθώς και στον κύκλο των φίλων μου για την
ωραίες αναμνήσεις που μου προσφέρουν όλα αυτά τα χρόνια.

\begin{flushright}Μαρινέλλης Γιώργος\end{flushright}

\end{acknowledgements}
\end{greek}
