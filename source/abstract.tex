%% Greek Abstract
\begin{greek}
\begin{abstract}

Στις μέρες μας, ο όλο και μεγαλύτερος όγκος πολυμεσικών δεδομένων
που δημιουργείται καθιστά τα δεδομένα μη διαχειρίσιμα και αναξιοποίητα. Εμφανίζεται
έτσι η ανάγκη της επισήμανσης και ταξινόμησης του πολυμεσικού αυτού περιεχομένου
για την καλύτερη αξιοποίησή του τόσο από απλούς χρήστες όσο και από άλλους επαγγελματικούς
κλάδους. Η προσθήκη αυτόματων επισημάνσεων στο περιεχόμενο είναι κάτι παραπάνω από
απαραίτητη καθώς ,όπως είπαμε, ο όγκος των δεδομένων δε αφήνει χώρο και χρόνο για
χειροκίνητες προσθήκες.

Στα πλαίσια της συγκεκριμένης εργασίας, επιλέξαμε να παράγουμε αυτόματες επισημάνσεις
χρησιμοποιώντας αλγορίθμους του κλάδου της όρασης των υπολογιστών. Πιο συγκεκριμένα
υλοποιήθηκε μια εφαρμογή η οποία πραγματοποιεί ανίχνευση και αναγνώριση προσώπων και
αντικειμένων. Στο κείμενό παρουσιάζουμε όλες τις σύγχρονες μεθόδους για την
ανίχνευση και αναγνώριση προσώπων και αντικειμένων, αναλύουμε τα πλεονεκτήματα και
τα μειονεκτήματα της καθεμίας καθώς και τους λόγους που
επιλέξαμε να χρησιμοποιήσουμε συγκεκριμένες από αυτές. Χρησιμοποιούμε επίσης
και μια παραλλαγή της μεθόδου Local Binary Patterns Histogram για την αναγνώριση προσώπων η οποία σχεδιάστηκε
και τροποποιήθηκε από εμάς.

Στο τέλος γίνεται μια αποτίμηση της παραλλαγμένης  αυτής μεθόδου χρησιμοποιώντας
κατάλληλες μετρικές πάνω σε εικόνες από τις βάσεις εικόνων προσώπων AT\&T Facedatabe,
Yale Facedatabase A, Extended Yale Facedatabase B και MyLucce Facedatabase. Οι
εικόνες για την τελευταία βάση συλλέχθηκαν από εμάς. Τα αποτελέσματα που συλλέξαμε
παρουσιάζουν ενδιαφέρον και δίνουν μια συνολικότερη αντίληψη πάνω στο γενικότερο
πρόβλημα της αναγνώρισης προσώπων.


\begin{keywords}
artificial intelligence, machine learning, convolutional neural
networks, object detection, face detection, face recognition, recognition
systems, viola jones, LBPH, Mobilenet, SSD, Faster R-CNN, cascade classifier,
Local Binary Patterns
\end{keywords}

\end{abstract}
\end{greek}

%% English Abstract
\begin{abstracteng}

Nowadays, the continuously growing amount of multimedia data makes them unmanagable
and useless. Shows up thus the need to mark and classify this multimedia content
for better use by both simple and professional users. The extraction of automatic
annotations though from this content is an essential process because the volume of data
is such that there is no space to add manual annotations.

In the context of this thesis, we decided to extract automatic annotations using
computer vision algorithms. More specifically, we implemented an application which
performs face and object detection and recognition. In the following text we present
all the state-of-the art methods for face and object detection and recognition, we
analyse each method's pros and cons and we explain the reasons we choosed to use
the particular ones. We also use a modified version of the Local Binary Patterns Histogram
method for face recognition.

Finally, we perform an evaluation of the modified method using well known face databases
such as AT\&T Facedatabase, Yale Facedabase A, Extended Yale Facedatabase B and
MyLucce Facedatabase. The last one was created by us. The results are promising
and provide a complete overview over the general problem of face recognition.

\begin{keywordseng}
artificial intelligence, machine learning, convolutional neural
networks, object detection, face detection, face recognition, recognition
systems, viola jones, LBPH, Mobilenet, SSD, Faster R-CNN, cascade classifier,
Local Binary Patterns
\end{keywordseng}

\end{abstracteng}

%% Greek Acknowledgements
\begin{greek}
\begin{acknowledgements}

Αρχικά, θα ήθελα να ευχαριστήσω τον καθηγητή κ. Συμεών Παπαβασιλείου για
την ευκαιρία που μου έδωσε να ασχοληθώ με το συγκεκριμένο τομέα της επιστήμης
των υπολογιστών στο Εργαστήριο Διαχείρισης και Βέλτιστου Σχεδιασμού Δικτύων Τηλεματικής,
καθώς και για τη θετική συμβολή του καθόλη τη διάρκεια των σπουδών μου.

Η εκπόνηση της διπλωματικής αυτής εργασίας, δε θα μπορούσε να ολοκληρωθεί χωρίς
την καίρια συμβολή του επί χρόνια συναδέλφου μου και υποψήφιου διδάκτορα Γιώργου Μήτση
καθώς και ολόκληρου του εργαστηρίου Διαχείρισης και Βέλτιστου Σχεδιασμού Δικτύων Τηλεματικής και του
εργαστηρίου Ευφυών Συστημάτων, Περιεχομένου και Αλληλεπίδρασης.

Τέλος ένα μεγάλο ευχαριστώ στην οικογένειά μου και στους φίλους μου για τη συνεχή
τους στήριξη όλο αυτό το διάστημα των σπουδών μου και για τα ωραιότερα συναισθήματα
και αναμνήσεις που μου προσφέρουν όλα αυτά τα χρόνια.

\begin{flushright}Μαρινέλλης Γιώργος\end{flushright}

\end{acknowledgements}
\end{greek}
