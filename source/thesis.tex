% DOCUMENT FORMAT ==============================================================

\documentclass{ntua-thesis} % a4paper,11pt,twoside,titlepage already set
% \pagestyle{plain} % pagestyle already set
% \usepackage[margin=2.5cm]{geometry} % margins already set


% PACKAGE SETTINGS =============================================================

\usepackage[cm-default]{fontspec}
\usepackage{amsmath}
\usepackage{amsfonts}
\usepackage{courier}
\usepackage{multirow}
\usepackage{array}
\usepackage{mdwlist}
\usepackage{graphicx} % insert jpg/png pictures
\usepackage{gensymb}
\usepackage{xunicode}
\usepackage{xltxtra}
\usepackage{polyglossia}
\setmainlanguage{greek}
\setotherlanguage{english}
\newfontfamily\greekfont[Script=Greek]{CMU Serif}
\usepackage{url}
\usepackage{hyperref}
\hypersetup{%
    xetex,
    hyperfootnotes=true,
    colorlinks=true,
    pdfpagemode=UseOutlines,
    pdfstartview=FitH,
%    linkcolor=blue,
%    citecolor=green,
%    urlcolor=magenta,
    linkcolor=black,
    citecolor=black,
    urlcolor=black,
    pdftitle={Αυτόματος Σχολιασμός Προσώπων και Αντικειμένων σε πολυμεσικό περιεχόμενο},
    pdfauthor={Μαρινέλλης Δ. Γεώργιος},
    pdfsubject={},
    pdfkeywords={artificial inteligence, machine learning, convolutional neural
        networks, object detection, face detection, face recognition, recognition
        systems, viola jones, LBPH, Mobilenet, SSD, Faster R-CNN, cascade classifier,
        Local Binary Patterns}
}
\usepackage{rotating}
\usepackage[format=hang,textformat=simple]{caption}
\usepackage{wrapfig}
\usepackage{subfig}

\setromanfont[Mapping=tex-text]{CMU Serif}
\setsansfont[Mapping=tex-text]{CMU Sans Serif}
\setmonofont[Mapping=tex-text]{CMU Typewriter Text}
\setmainfont[Mapping=tex-text]{CMU Serif}


% CUSTOM COMMANDS ==============================================================
\newcommand{\flink}[1]{\footnote{\url{#1}}}
\newcommand{\todo}{\textrm{\textbf{\textcolor{yellow}{TODO: }}}}
\newcommand{\fixme}{\textrm{\textbf{\textcolor[RGB]{255,0,0}{FIXME: }}}}
\newcommand{\subscript}[1]{\ensuremath{{\textrm{#1}}}}
% add color in footnote
%\renewcommand\@makefnmark{\hbox{\@textsuperscript{\normalfont\color{green}\@thefnmark}}}
\newenvironment{fulltable}[3]{
    \def\tempcaption{#2}
    \def\templabel{#3}
    \begin{table}[hbtp]
    \begin{center}
    \begin{tabular}[c]{#1}
}{
    \end{tabular}
    \end{center}
    \caption{\tempcaption\label{\templabel}}
    \end{table}
}
%macro που δίνει το μέγιστο επιτρεπτό μέγεθος σε μια εικόνα χωρίς να
%παραβιάζει τα όρια του LaTeX
\makeatletter
\def\maxwidth{%
  \ifdim\Gin@nat@width>\linewidth
  \linewidth
  \else
  \Gin@nat@width
  \fi
}


% CODE HIGHLIGHTING ============================================================

%\usepackage{listings}
% Count listings per chapter
\usepackage[chapter]{minted}

\newcommand{\includeminted}[5][python]{%
  \nopagebreak
  \inputminted[numbersep=5pt,frame=lines,framesep=2mm,#5]{#1}{#2}
  \captionof{listing}{#3}
  \label{#4}
}

% DOCUMENT INFORMATION =========================================================

\title{Αυτόματη προσθήκη επισημάνσεων για πρόσωπα και αντικείμενα σε πολυμεσικό περιεχόμενο}
\author{Μαρινέλλης Γεώργιος}
\thesis[του]{Μαρινέλλη Δ. Γεώργιου}
\presenting{17}{10}{2018}
\supervisor[Καθηγητής ]{Συμεών Παπαβασιλείου} % the space is necessary
\epitropiS[Αν. Καθηγήτρια ]{Θεοδώρα Βαρβαρίγου}
\epitropiF[Επίκ. Καθηγήτρια ]{Ιωάννα Ρουσσάκη}
\department{Σχολή Ηλεκτρολόγων Μηχανικών και Μηχανικών Υπολογιστών}
\division{Τομέας Επικοινωνιών, Ηλεκτρονικής και Συστημάτων Πληροφορικής}
\lab{Εργαστήριο Διαχείρισης και Βέλτιστου Σχεδιασμού Δικτύων Τηλεματικής}


% MAIN DOCUMENT ================================================================

\begin{document}

\frontmatter
\maketitle

\def\templen{\parindent}
\setlength{\parindent}{0pt}
\setlength{\parskip}{1.5ex plus 0.5ex minus 0.2ex}
%% Greek Abstract
\begin{greek}
\begin{abstract}

Στις μέρες μας, ο όλο και μεγαλύτερος όγκος πολυμεσικών δεδομένων
που δημιουργείται καθιστά τα δεδομένα μη διαχειρίσιμα και αναξιοποίητα. Εμφανίζεται
έτσι η ανάγκη της επισήμανσης και ταξινόμησης του πολυμεσικού αυτού περιεχομένου
για την καλύτερη αξιοποίησή του τόσο από απλούς χρήστες όσο και από άλλους επαγγελματικούς
κλάδους. Η προσθήκη αυτόματων επισημάνσεων στο περιεχόμενο είναι κάτι παραπάνω από
απαραίτητη καθώς ,όπως είπαμε, ο όγκος των δεδομένων δε αφήνει χώρο και χρόνο για
χειροκίνητες προσθήκες.

Στα πλαίσια της συγκεκριμένης εργασίας, επιλέξαμε να παράγουμε αυτόματες επισημάνσεις
χρησιμοποιώντας αλγορίθμους του κλάδου της όρασης των υπολογιστών. Πιο συγκεκριμένα
υλοποιήθηκε μια εφαρμογή η οποία πραγματοποιεί ανίχνευση και αναγνώριση προσώπων και
αντικειμένων. Στο κείμενό παρουσιάζουμε όλες τις σύγχρονες μεθόδους για την
ανίχνευση και αναγνώριση προσώπων και αντικειμένων, αναλύουμε τα πλεονεκτήματα και
τα μειονεκτήματα της καθεμίας καθώς και τους λόγους που
επιλέξαμε να χρησιμοποιήσουμε συγκεκριμένες από αυτές. Χρησιμοποιούμε επίσης
και μια παραλλαγή της μεθόδου Local Binary Patterns Histogram για την αναγνώριση προσώπων η οποία σχεδιάστηκε
και τροποποιήθηκε από εμάς.

Στο τέλος γίνεται μια αποτίμηση της παραλλαγμένης  αυτής μεθόδου χρησιμοποιώντας
κατάλληλες μετρικές πάνω σε εικόνες από τις βάσεις εικόνων προσώπων AT\&T Facedatabe,
Yale Facedatabase A, Extended Yale Facedatabase B και MyLucce Facedatabase. Οι
εικόνες για την τελευταία βάση συλλέχθηκαν από εμάς. Τα αποτελέσματα που συλλέξαμε
παρουσιάζουν ενδιαφέρον και δίνουν μια συνολικότερη αντίληψη πάνω στο γενικότερο
πρόβλημα της αναγνώρισης προσώπων.


\begin{keywords}
artificial intelligence, machine learning, convolutional neural
networks, object detection, face detection, face recognition, recognition
systems, viola jones, LBPH, Mobilenet, SSD, Faster R-CNN, cascade classifier,
Local Binary Patterns
\end{keywords}

\end{abstract}
\end{greek}

%% English Abstract
\begin{abstracteng}

Nowadays, the continuously growing amount of multimedia data makes them unmanagable
and useless. Shows up thus the need to mark and classify this multimedia content
for better use by both simple and professional users. The extraction of automatic
annotations though from this content is an essential process because the volume of data
is such that there is no space to add manual annotations.

In the context of this thesis, we decided to extract automatic annotations using
computer vision algorithms. More specifically, we implemented an application which
performs face and object detection and recognition. In the following text we present
all the state-of-the art methods for face and object detection and recognition, we
analyse each method's pros and cons and we explain the reasons we choosed to use
the particular ones. We also use a modified version of the Local Binary Patterns Histogram
method for face recognition.

Finally, we perform an evaluation of the modified method using well known face databases
such as AT\&T Facedatabase, Yale Facedabase A, Extended Yale Facedatabase B and
MyLucce Facedatabase. The last one was created by us. The results are promising
and provide a complete overview over the general problem of face recognition.

\begin{keywordseng}
artificial intelligence, machine learning, convolutional neural
networks, object detection, face detection, face recognition, recognition
systems, viola jones, LBPH, Mobilenet, SSD, Faster R-CNN, cascade classifier,
Local Binary Patterns
\end{keywordseng}

\end{abstracteng}

%% Greek Acknowledgements
\begin{greek}
\begin{acknowledgements}

Αρχικά, θα ήθελα να ευχαριστήσω τον καθηγητή κ. Συμεών Παπαβασιλείου για
την ευκαιρία που μου έδωσε να ασχοληθώ με το συγκεκριμένο τομέα της επιστήμης
των υπολογιστών στο Εργαστήριο Διαχείρισης και Βέλτιστου Σχεδιασμού Δικτύων Τηλεματικής,
καθώς και για τη θετική συμβολή του καθόλη τη διάρκεια των σπουδών μου.

Η εκπόνηση της διπλωματικής αυτής εργασίας, δε θα μπορούσε να ολοκληρωθεί χωρίς
την καίρια συμβολή του επί χρόνια συναδέλφου μου και υποψήφιου διδάκτορα Γιώργου Μήτση
καθώς και ολόκληρου του εργαστηρίου Διαχείρισης και Βέλτιστου Σχεδιασμού Δικτύων Τηλεματικής και του
εργαστηρίου Ευφυών Συστημάτων, Περιεχομένου και Αλληλεπίδρασης.

Τέλος ένα μεγάλο ευχαριστώ στην οικογένειά μου και στους φίλους μου για τη συνεχή
τους στήριξη όλο αυτό το διάστημα των σπουδών μου και για τα ωραιότερα συναισθήματα
και αναμνήσεις που μου προσφέρουν όλα αυτά τα χρόνια.

\begin{flushright}Μαρινέλλης Γιώργος\end{flushright}

\end{acknowledgements}
\end{greek}


\setlength{\parindent}{\templen}
\setlength{\parskip}{0pt}
\tableofcontents
\listoffigures
\listoftables
\renewcommand\listoflistingscaption{Κατάλογος Αλγορίθμων}
\listoflistings
% Fix the vertical spacing of listoflistings to look like the
% listoffigures
\makeatletter
\let\my@chapter\@chapter
\renewcommand*{\@chapter}{%
  \addtocontents{lol}{\protect\addvspace{10pt}}%
  \my@chapter}
\makeatother
% Due to unknown conflict between toc and minted, manualy add the Listings entry
\addcontentsline{toc}{chapter}{Κατάλογος Αλγορίθμων}

\mainmatter
% moved these two commands here so that they don't influence the toc
\setlength{\parindent}{0pt}
\setlength{\parskip}{1.5ex plus 0.5ex minus 0.2ex}

\renewcommand{\floatpagefraction}{.7}

\chapter{Εισαγωγή}\label{ch:introduction}

Στις μέρες μας, η τεχνητή νοημοσύνη είναι ένας ταχύτατα αναπτυσσόμενος κλάδος
της επιστήμης των υπολογιστών. Η όραση υπολογιστών είναι ένα επιστημονικό πεδίο
της τεχνητής νοημοσύνης που δε θα μπορούσε να μείνει ανεπηρέαστο από αυτή την
εξέλιξη.

Η όραση των υπολογιστών (από και στο εξής ΟτΥ) ασχολείται με την απόκτηση,
ανάλυση και κατανόηση εικόνων, βίντεο και γενικά πολυμεσικού περιεχομένου
πολλών διαστάσεων από τον πραγματικό κόσμο. Έχει ως σκοπό να δώσει στα
υπολογιστικά συστήματα μια εποπτεία και κατανόηση του τετραδιάστατου
πραγματικού κόσμου.

Για να επιτευχθεί ο άνωθεν σκοπός χρειάζεται να αυτοματοποιηθεί μέσω μιας
υπολογιστικής αλγοριθμικής διαδικασίας η μέθοδος της ανθρώπινης όρασης. Έτσι
η πραγματική ν-διάσταση αναπαράσταση που απεικονίζει και αναγνωρίζει ο
ανθρώπινος εγκέφαλος, αναπαρίσταται με συμβολικό και αριθμητικό τρόπο.

\section{Ανίχνευση προσώπων και αντικειμένων}
Η ανίχνευση προσώπων και γενικότερα αντικειμένων σε μια εικόνα είναι μια
καθημερινή, αυτοματοποιημένη και τετριμένη διαδικασία για τον άνθρωπο. Ο ανθρώπινος
εγκέφαλος είναι εκπαιδευμένος με τέτοιο τρόπο ώστε να μπορεί να αναγνωρίζει
αντικείμενα ακαριαία. Μάλιστα η αναγνώριση δεν αφορά δεν αφορά την αναγνώριση
μεμονομένων και συγκεκριμένων αντικειμένων αλλά γενικότερα την αναγνώριση
συγκεκριμένων κλάσεων αντικειμένων. Έτσι ο εγκέφαλός μας μπορεί να αναγνωρίζει
αντικέιμενα χωρίς προηγουμένως να τα έχει δει. Αντίθετα όμως η διαδικασία αυτή
δεν εκτελείται το ίδιο εύκολα και από ένα υπολογιστικό σύστημα.


\section{Αναγνώριση προσώπων}

\section{Συνεισφορά της διπλωματικής}

Στην παρούσα διπλωματική δημιουργήσαμε μια εφαρμογή η οποία μπορεί να παράγει
αυτόματες επισημάνσεις στο πολυμεσικό περιεχόμενο (video) που δέχεται ως
είσοδο. Οι τεχνικές για την εξαγωγή αυτών των επισημάνσεων χρησιμοποιούν
τις state-of-art τεχνολογίες όσον αφορά την ανίχνευση και αναγνώρηση προσώπων
και της ανίχνευσης των αντικειμένων. Επίσης στο κομμάτι της αναγνώρησης προσώπων
,πέρα από τις υπάρχουσες τεχνικές, αναπτύξαμε και μια καινούργια μέθοδο.

Παράλληλα με την εξαγωγή των επισημάνσεων γίνετα ,στο πλαίσιο της διπλωματικής,
και μια συγκριτική αξιολόγηση των τεχνικών για την αναγνώρηση προσώπων.

\section{Οργάνωση κειμένου}

Το παρόν κείμενο έχει την εξής δομή:

\begin{description}
  \item[Κεφάλαιο~\ref{ch:facedetection}:] \hfill \\
    Στο κεφάλαιο αυτό θα κάνουμε μια γενική επισκόπηση των μεθόδων που
    χρησιμοποιούνται για την ανίχνευση κλάσεων αντικειμένων.
    Θα γίνει μια συνοπτική παρουσίαση του αλγορίθμου των Viola-Jones για την
    αναγνώρηση προσώπων καθώς
    και ο τρόπος με τον οποίο τον χρησιμοποιήσαμε. Πρόκειται για τον αλγόριθμο
    πάνω στον οποίο βασίζεται η ανίχνευση προσώπων στις εικόνες και συνεπακόλουθα
    στα βίντεο. Θα συζητήσουμε πως σχετίζεται το αποτέλεσμα της μεθόδου με
    την μετέπειτα αναγνώριση των προσώπων.
  \item[Κεφάλαιο~\ref{ch:googlenet}:] \hfill \\
    Θα παρουσιαστεί εδώ μια γενική επισκόπιση των νευρωνικών δικτύων. Θα συζητήσουμε
    για το Caffe framework το οποίο υποστηρίζει την παραγωγή νευρωικών δικτύων
    και τέλος θα μιλήσουμε για το πως χρησιμοποιούμε το ανωτέρο νευρωνικών
    για να εξάγουμε σχολιασμούς για τις εικόνες και τα βίντεο.
  \item[Κεφάλαιο~\ref{ch:facerec}:] \hfill \\
    Γίνεται μια σύντομη παρουσίαση των μεθόδων που χρησιμοποιούνται για την
    αναγνώριση προσώπων. Θα αναλύσουμε επίσης και τη δική μας τεχνική αναγνώρισης
    και πως διαφοροποιείται από τις υπάρχουσες.
  \item[Κεφάλαιο~\ref{ch:results}:] \hfill \\
    Στο κομμάτι αυτό θα δούμε κάποια συγκριτικά αποτελέσματα των τεχνικών
    εξαγωγής σχολιασμών. Θα αναλυθούν τα σημεία και οι λόγοι διαφοροποίησης
    από άλλες μεθόδους και θα προταθούν τροποιήσεις που μπορούν να οδηγήσουν
    σε ποιο εύστοχα αποτελέσματα.
  \item[Κεφάλαιο~\ref{ch:conclusion}:] \hfill \\
    Τέλος θα κάνουμε μια ανασκόπηση της διπλωματικής. Θα αναφερθούμε σε διάφορες
    άλλες τεχνικές και θα γίνει μια πρόταση για παραπέρα έρευνα
\end{description}

\chapter{Ανίχνευση προσώπων}\label{ch:facedetection}
Η επιστημονική περιοχή της ανίχνευσης αντικειμένου μπορεί να χωριστεί
σε τρεις επιμέρους ανεξάρτητες διαδικασίες.
\begin{itemize}
    \item Εντοπισμός πιθανών περιοχών αντικειμένου/ων
    \item Ταξινόμηση αντικειμένου
    \item Προσδιορισμός θέσης αντικειμένου
\end{itemize}


H διαδικασία του εντοπισμού των περιοχών αντικειμένων αφορά την εύρεση των περιοχών
εντός μιας εικόνας όπου είναι πιθανό να υπάρχουν αντικείμενα. Πιο συγκεκριμένα, αφορά τον προσδιορισμό
του συνόλου των εικονοστοιχείων εκείνων τα οποία με μια γρήγορη επεξεργασία δίνουν
μεγάλη πιθανότητα να εμπεριέχουν ένα αντικείμενο.

Η ταξινόμηση αντικειμένου αφορά τον προσδιορισμό της κλάσης του αντικειμένου που
έχει προσδιοριστεί. Η διαδικασία της ταξινόμησης απαιτεί την εξαγωγή διάφορων
χαρακτηριστικών από την εικόνα και επεξεργασία αυτών με κάποιο αλγόριθμο ταξινόμησης
για τον συμπερασμό της κλάσης του αντικειμένου

Ο τελικός προσδιορισμός της θέσης του αντικειμένου εντός της εικόνας μπορεί να γίνει σε
τρία επίπεδα: α) σε επίπεδο εικόνας, β) σε επίπεδο περιγράμματος γ) σε επίπεδο παραθύρου
(βλ. Σχήμα \ref{fig:bee}).

\begin{figure}[htp]
    \centering
    \includegraphics[width=.49\textwidth]{../figures/bee.png}\hfill
    \includegraphics[width=.49\textwidth]{../figures/beescissors.png}\hfill
    \includegraphics[width=.48\textwidth]{../figures/beebbox.png}\hfill

    \caption{α) σε επίπεδο εικόνας, β) σε επίπεδο περιγράμματος γ) σε επίπεδο παραθύρου.}
    \label{fig:bee}

\end{figure}

Στη συνέχεια του κειμένου της παρούσας διπλωματικής όταν αναφερόμαστε στον όρο
ανίχνευση αντικειμένου θα αναφερόμαστε στην ανίχνευση σε επίπεδο παραθύρου.

Η ανίχνευση προσώπου αποτελεί μια υποπεριοχή της ανίχνευσης αντικείμενου. Επομένως
η μεθοδολογία που ακολουθείται είναι όμοια με αυτή που περιγράψαμε ανωτέρω. Ειδικότερα,
η ανίχνευση προσώπου παρουσιάζει μια ευκολία σε σχέση με την ανίχνευση κλάσης αντικειμένου
με την έννοια ότι κάθε πρόσωπο έχει συγκεκριμένα και καθολικά χαρακτηριστικά στο
σύνολο των ανθρώπων. Συνεπώς, η ανίχνευση προσώπου δεν αφορά το γενικότερο πρόβλημα
του εντοπισμού της κλάσης ενός αντικειμένου αλλά το υποπρόβλημα του εντοπισμού
-εντός μια εικόνας- μιας και μόνον κλάσης αντικειμένου (του προσώπου) με αυστηρώς
προσδιορισμένα χαρακτηριστικά.

Παρακάτω θα κάνουμε μια σύντομη περιγραφή των βασικότερων μεθόδων αναγνώρισης
προσώπου που έχουν χρησιμοποιηθεί έως σήμερα.

\section{Συνοπτική παρουσίαση βασικών μεθόδων αναγνώρισης προσώπων}\label{sec:facedetmethods}



\subsection{Η μέθοδος Rowley, Baluja και Kanade~\cite{Rowley:1998:NNF:275341.275344}}

To 1998 οι \emph{Rowley, Baluja και Kanade} περιέγραψαν μια μέθοδο για την αναγνώριση
προσώπων βασισμένη στο συνδυασμό αποτελεσμάτων από νευρωνικά δίκτυα. Στην εικόνα
δοκιμάζεται ένα σετ από φίλτρα βασισμένα σε νευρωνικά δίκτυα. Στη συνέχεια τα
αποτελέσματα επεξεργάζονται από ένα διαμεσολαβητή και συνδυάζονται Πιο συγκεκριμένα
η μέθοδος αποτελείται από 2 βήματα:

\begin{description}\label{item:rowley}
  \item[Βήμα 1] \hfill \\
    Στο βήμα αυτό επιλέγεται αρχικά ένα παράθυρο μεγέθους 20x20 από την αρχική εικόνα.
    Το παράθυρο αυτό υπόκειται μια προεργασία για την ισοστάθμιση της έντασης του
    χρώματος, την αντίθεση κ.α. και στη συνέχεια δίνεται ως είσοδο σε ένα ή και
    περισσότερα (προεκπαιδευμένα) νευρωνικά δίκτυα. Τα δίκτυα αυτά υπολογίζουν κάποια
    συγκεκριμένα χαρακτηριστικά που αφορούν τα πρόσωπα (πχ μύτη, μάτια, στόμα) και
    αποφασίζουν αν στο παράθυρο αυτό υπάρχει ή όχι πρόσωπο. Η εικόνα σαρώνεται
    ολόκληρη σε παράθυρα μεγέθους 20x20 τα οποία ακολουθούν την παραπάνω ροή.
    Το μέγεθος του παραθύρου αυξάνεται σταδιακά με ένα προκαθορισμένο βάρος για
    εντοπίσει πρόσωπα μεγαλύτερα του προαναφερθέντος μεγέθους.
  \item[Βήμα 2] \hfill \\
    Το παρόν βήμα αποτελείται από δύο υπο-βήματα. Αφενός συλλέγονται τα επιμέρους
    αποτελέσματα του κάθε νευρωνικού δικτύου και με βάση μια ευριστική απορρίπτονται
    κάποιες λανθασμένες (false positive) προβλέψεις. Αφετέρου, συνδυάζονται τα
    αποτελέσματα από όλα τα επιμέρους νευρωνικά δίκτυα και χρησιμοποιώντας κάποια
    άλλη ευριστική παράγονται οι τελικές προβλέψεις του αλγορίθμου για τα πρόσωπα.
\end{description}

Βάση μετρήσεων η παραπάνω μέθοδος μπορεί να επιτύχει ποσοστά επιτυχούς αναγνώρισης
προσώπων μεταξύ $77.9\% - 90.3\%$ -με ένα ασφαλή αριθμό λανθασμένων προβλέψεων-
ανάλογα με τις ευριστικές μεθόδους που χρησιμοποιούνται.

\subsection{H μέθοδος SNow(Sparse Network of Winnows)~\cite{Yang00asnow-based}}

Αργότερα, το \emph{2000} οι \emph{Yang, Roth} και \emph{Ahuja} πρότειναν μια νέα
μέθοδο βασισμένη στην αρχιτεκτονική εκμάθησης \emph{SNoW}~\cite{Roth98learningto}~\cite{CCRR99}.
Η αρχιτεκτονική αυτή -\emph{SNoW}- (Sparse Network of Winnows)
είναι στην πραγματικότητα ένα δίκτυο από γραμμικές συναρτήσεις που χρησιμοποιούν
τον κανόνα Winnow~\cite{Littlestone:1988:LQI:639961.639994} και επιτρέπει την
ανίχνευση προσώπων με διαφορετικά χαρακτηριστικά, σε διαφορετικές θέσεις και
στάσεις και με διαφορές στις φωτιστικές συνθήκες. Το σύστημα αυτό είναι ειδικά
σχεδιασμένο για μάθηση σε τομείς στους οποίους ο δυνητικός αριθμός των
χαρακτηριστικών που συμμετέχουν στις αποφάσεις είναι πολύ μεγάλο, αλλά μπορεί να
είναι άγνωστο a priori.

Η διαδικασία της αναγνώρισης προσώπου είναι όμοια με αυτή των \emph{Rowley, Baluja και Kanade}
όπου η εικόνα σαρώνεται αρχικά σε παράθυρα μεγέθους 20x20 και το μέγεθος αυτό
αυξάνεται σε κάθε επανάληψη κατά μια σταθερή, προκαθορισμένη τιμή $(1.2)$. Η
διαδικασία αυτή επαναλαμβάνεται 10 φορές. Σε κάθε επανάληψη το τρέχον παράθυρο
από την εικόνα δίνεται ως είσοδο στο προεκπαιδευμένο δίκτυο SNoW το οποίο λαμβάνει
την απόφαση για την ύπαρξη ή όχι προσώπου υπολογίζοντας κάθε φορά ένα διαφορετικό
αριθμό χαρακτηριστικών.

Η μέθοδος αυτή μπορεί να επιτυγχάνει ποσοστά επιτυχίας κοντά στο $93\%$.

\section{H μέθοδος Viola-Jones}\label{sec:violjon}

O αλγόριθμος των Viola-Jones~\cite{Viola01rapidobject} υπήρξε τομή στο πεδίο
της ανίχνευσης αντικειμένων σε εικόνες. Ουσιαστικά δημιούργησε την πρώτη
υποδομή ανίχνευσης αντικειμένων η οποία παρήγαγε ανταγωνιστικά αποτέλεσμα σε
πραγματικό χρόνο. Παρόλο που σχεδιάστηκε ώστε να μπορεί να εκπαιδευτεί με σκοπό
να αναγνωρίζει οποιαδήποτε κλάση αντικειμένων, ουσιαστικά ο αρχικός σχεδιασμός
του αλγορίθμου έγινε με γνώμονα την αναγνώριση προσώπων σε εικόνες. Εν τέλει
εκεί εντοπίζεται και το μεγαλύτερο ποσοστό της χρήσης του.

Τα βασικά χαρακτηριστικά του ανωτέρω αλγορίθμου είναι:
\begin{description}
  \item[Αξιοπιστία] \hfill \\
      Ο αλγόριθμος έχει πάντα υψηλό ποσοστό σωστών ανιχνεύσεων (true-positives)
        και χαμηλό ποσοστό λάθος ανιχνεύσεων (false-positives)
  \item[Ταχύτητα πραγματικού χρόνου] \hfill \\
      Σε εφαρμογές πραγματικού περιβάλλοντος επεξεργάζονται τουλάχιστον 2 εικόνες
        (frames) ανά δευτερόλεπτο
  \item[Αναγνώριση προσώπων] \hfill \\
      Ο αλγόριθμος έχει στόχο να ανιχνεύει πρόσωπα από μη-πρόσωπα. Δεν έχει
        σκοπό να αναγνωρίζει τα πρόσωπα που ανιχνεύει
\end{description}

Για να το πετύχει αυτό ο αλγόριθμος χρησιμοποιεί κατά σειρά τα παρακάτω στάδια
τα οποία θα αναλύσουμε εκτενέστερα στη συνέχεια:
\begin{itemize}
 \item Επιλογή χαρακτηριστικών τύπου Haar (Haar features selection)
 \item Κατασκευή ολοκληρώματος εικόνας (Integral image creation)
 \item Εκπαίδευση του αλγορίθμου AdaBoost (AdaBoost training)
 \item Χρήση ενός διαδοχικά διασυνδεδεμένου ταξινομητή (Cascade classifier)
\end{itemize}

\subsection{Χαρακτηριστικά τύπου Haar}

Για να ανιχνεύσουμε αντικείμενα σε εικόνες απαιτείται κατάλληλη επεξεργασία και
αναπαράσταση του περιεχομένου τους. Για την αναπαράσταση του περιεχομένου της εικόνας στη
μέθοδο που εξετάζουμε, χρησιμοποιούμε τα χαρακτηριστικά τύπου Haar, τα οποία προκύπτουν
από την εφαρμογή του μετασχηματισμού Wavelet σε μια εικόνα με χρήση των συναρτήσεων
τύπου Haar~\cite{Viola01rapidobject}.

Η χρησιμοποίηση των συναρτήσεων Haar στο μετασχηματισμό Wavelet ξεκινά από την
παρατήρηση ότι η τιμή της φωτεινότητας κάθε εικονοστοιχείου επηρεάζεται έντονα από τις
αλλαγές στο φωτισμό της σκηνής~\cite{OrePapSinOsu97}. Αυτή η αλλαγή όμως, επηρεάζει ομοιόμορφα
όλα τα pixel της εικόνας. Έτσι, η τιμή μιας συνάρτησης που εξετάζει τη μέση διαφορά
ανάμεσα σε δύο ή τρεις περιοχές της ίδιας εικόνας, θα παραμένει σε μεγάλο βαθμό ανεπηρέαστη.
Χρησιμοποιώντας, λοιπόν, τις συναρτήσεις Haar, η διαδικασία της ανίχνευσης αντικειμένων δε θα
επηρεάζεται από τις διαφορές στη φωτεινότητα από εικόνα σε εικόνα.

Οι συναρτήσεις Haar υπολογίζουν τη διαφορά ανάμεσα στους μέσους όρους των τιμών των
εικονοστοιχείων δύο (ή τριών) περιοχών. Ας θεωρήσουμε τη συνάρτηση Haar που παριστάνεται με
το ορθογώνιο a.i από το Σχήμα \ref{fig:haarfeatures}. Υπολογίζεται ο μέσος όρος των εικονοστοιχείων που
βρίσκονται μέσα στο άσπρο ορθογώνιο, καθώς και αυτών που βρίσκονται μέσα στο μαύρο
ορθογώνιο. Έπειτα, ο μέσος όρος του μαύρου ορθογωνίου αφαιρείται από τον μέσο όρο του
άσπρου. Η τιμή που προκύπτει αποτελεί την τιμή του Haar χαρακτηριστικού.

\begin{figure}[htbp]
  \begin{center}
    \includegraphics[width=0.8\maxwidth]{../figures/haarfeatures.png}
    \caption{Χαρακτηριστικά τύπου Haar\label{fig:haarfeatures}}
   \end{center}
\end{figure}


Εφαρμόζοντας τον μετασχηματισμό Wavelet με τη συναρτησιακή βάση Haar, προκύπτει
ένας περιορισμένος αριθμός χαρακτηριστικών~\cite{OrePapSinOsu97}. Στο μονοδιάστατο μετασχηματισμό, η
απόσταση ανάμεσα σε δύο γειτονικά κυματίδια (wavelets), σε επίπεδο $n$ , θα είναι $2 n$ . Η απόσταση αυτή είναι
πολύ μεγάλη, κι έτσι δεν λαμβάνουμε όσες πληροφορίες θέλουμε από μια εικόνα ώστε να την
περιγράψουμε λεπτομερώς. Για να έχουμε, λοιπόν, μια πιο λεπτομερή, χωρικά, αναπαράσταση του
περιεχομένου της εικόνας χρειαζόμαστε ένα σύνολο από πλεονάζουσες συναρτήσεις βάσης. Για να
το πετύχουμε αυτό, εφαρμόζουμε τις συναρτήσεις Haar με μεταξύ τους απόσταση ένα
εικονοστοιχείο κάθε φορά. Έτσι, θα έχουμε μια πολύ πιο πυκνή αναπαράσταση. Επίσης, στο
μετασχηματισμό Wavelet, το μέγεθος των συναρτήσεων Haar, κανονικά διπλασιάζεται σε κάθε
επανάληψη. Για να αυξήσουμε ακόμα περισσότερο την λαμβανόμενη πληροφορία από την εικόνα,
ορίζουμε ότι το μέγεθος των συναρτήσεων Haar θα αυξάνει κάθε φορά κατά ένα μόνο
εικονοστοιχείο. Έτσι, το σύνολο των χαρακτηριστικών Haar σε μία εικόνα γίνεται
υπερπολλαπλάσιο του αρχικού. Αυξάνουμε, δηλαδή, την ποσότητα της πληροφορίας που αντλούμε
από μια εικόνα, αυξάνοντας τα χαρακτηριστικά τύπου Haar που θα υπολογιστούν σε αυτήν.

Τα κλασσικά Haar χαρακτηριστικά φαίνονται στο Σχήμα \ref{fig:haarfeatures}.a (Edge features) ~\cite{OrePapSinOsu97}.
Είναι σχετικά απλά και μπορούν να εντοπίσουν ακμές οριζόντια και κατακόρυφα καθώς και διαγώνιες γραμμές. Για να
μπορέσουμε να αναπαραστήσουμε γραμμές, ράβδους και τετράγωνα καλύτερα, προσθέτουμε τα
χαρακτηριστικά που φαίνονται στο Σχήμα \ref{fig:haarfeatures}.b (Line features)(τα χαρακτηριστικά iv και v εμφανίζονται στο
~\cite{Viola01rapidobject}, ενώ τα υπόλοιπα στο ~\cite{Lienhart02anextended},
τα οποία υπολογίζονται χωρίς να αυξάνεται ιδιαίτερα η
πολυπλοκότητα, όπως θα δούμε στην ενότητα 2.2.2. Μια μεγάλη προσθήκη είναι τα χαρακτηριστικά
που είναι περιστραμμένα κατά 45 \degree και φαίνονται στο Σχήμα \ref{fig:haarfeatures}.c ~\cite{Lienhart02anextended}. Με τη χρήση αυτών
βελτιώνεται σημαντικά η αναπαράσταση των διαγώνιων σχημάτων. Με την προσθήκη όλων αυτών
των χαρακτηριστικών, το σύνολο γίνεται υπερπλήρες και αναπαριστά πολύ καλύτερα την
πληροφορία που περιέχεται σε μία εικόνα.

Για τον υπολογισμό του πλήθους των Haar χαρακτηριστικών σε κάποιο παράθυρο εικόνας
πλάτους W και ύψους H, ακολουθούμε την παρακάτω διαδικασία ~\cite{Lienhart02anextended}. Έστω ότι w και h
είναι το πλάτος και ύψος του ορθογωνίου της συνάρτησης Haar που εξετάζουμε. Το μέγεθος του
ορθογωνίου θα αυξάνεται κατά ένα σε κάθε βήμα. Άρα, οι μέγιστοι συντελεστές μεγέθυνσης των
ορθογωνίων σε πλάτος και ύψος θα είναι $X = \lfloor \frac{W}{w} \rfloor $ και $Y = \lfloor \frac{H}{h} \rfloor$ , αντίστοιχα. Το πλήθος των
χαρακτηριστικών που προκύπτουν από την εφαρμογή ενός κατακόρυφου Haar χαρακτηριστικού
στο παράθυρο εικόνας, είναι:
$$
XY \cdot \Bigg(W+1-w \frac{X+1}{2} \Bigg) \cdot \Bigg(H+1-h \frac{Y+1}{2} \Bigg)
$$

Μπορούμε τώρα να εφαρμόσουμε τους παραπάνω τύπους για ένα παράθυρο μεγέθους 30x30.
Τότε θα έχουμε $W=20$ και $Η=20$ και το πλήθος των χαρακτηριστικών στο παράθυρο φαίνεται
στον παρακάτω πίνακα ~\ref{tab:haarsum}:


\begin{table}[htbp]
  \centering
    \begin{tabular}{ | l | l | l | l | l | l | }
    \hline
        χαρακτηριστικό  & w & h  & X  & Y & πλήθος \\ \hline
    \hline
        \hspace{27pt}\includegraphics[width=0.4\maxwidth]{../figures/h1.png}
        & 2 & 1 & 10 & 20 & 21.000 \\
    \hline
        \hspace{27pt}\includegraphics[width=0.4\maxwidth]{../figures/h2.png}
        & 1 & 2 & 20 & 10 & 21.000 \\
    \hline
        \hspace{27pt}\includegraphics[width=0.4\maxwidth]{../figures/h3.png}
        & 2 & 2 & 10 & 10  & 10.000\\
    \hline
        \hspace{23pt}\includegraphics[width=0.4\maxwidth]{../figures/h4.png}
        & 3 & 1 & 6 & 20 & 13.230 \\
    \hline
        \hspace{26pt}\includegraphics[width=0.4\maxwidth]{../figures/h5.png}
        & 1 & 3 & 20 & 6 & 13.230 \\
    \hline
        \hspace{18pt}\includegraphics[width=0.4\maxwidth]{../figures/h6.png}
        & 4 & 1 & 5 & 20 & 9.450\\
    \hline
        \hspace{27pt}\includegraphics[width=0.4\maxwidth]{../figures/h7.png}
        & 1 & 4 & 20 & 5 & 9.450\\
    \hline
        \hspace{25pt}\includegraphics[width=0.4\maxwidth]{../figures/h8.png}
        & 3 & 3 & 6 & 6 & 3.969 \\
    \hline
        \hspace{23pt}\includegraphics[width=0.4\maxwidth]{../figures/h9.png}
        & 1 & 2 & 6 & 6 & 3.969\\
    \hline
        \hspace{23pt}\includegraphics[width=0.4\maxwidth]{../figures/h10.png}
        & 2 & 1 & 6 & 6 & 3.969\\
    \hline
        \hspace{16pt}\includegraphics[width=0.4\maxwidth]{../figures/h11.png}
        & 3 & 1 & 5 & 5 & 2.025 \\
    \hline
        \hspace{16pt}\includegraphics[width=0.4\maxwidth]{../figures/h12.png}
        & 1 & 3 & 5 & 5 & 2.025 \\
    \hline
        \hspace{16pt}\includegraphics[width=0.4\maxwidth]{../figures/h13.png}
        & 4 & 1 & 4 & 4 & 1.156 \\
    \hline
        \hspace{16pt}\includegraphics[width=0.4\maxwidth]{../figures/h14.png}
        & 1 & 4 & 4 & 4 & 1.156 \\
    \hline
        \hspace{16pt}\includegraphics[width=0.4\maxwidth]{../figures/h15.png}
        & 3 & 3 & 3 & 3 & 729 \\
    \hline
        Σύνολο & & & & &116.358 \\
    \hline
  \end{tabular}
  \caption{Πλήθος χαρακτηριστικών Haar ανά παράθυρο}
  \label{tab:haarsum}
\end{table}

Αυτό που παρατηρούμε είναι ότι σε ένα παράθυρο μεγέθους 20x20 δηλαδή 400 pixel
μπορούν να υπολογιστούν 116.358 χαρακτηριστικά. Το πλήθος αυτό είναι
εντελώς δυσανάλογο με τον αριθμό των pixel σε εκείνο το παράθυρο.
Η εξήγηση είναι ότι το σύνολο των χαρακτηριστικών που εξετάσαμε είναι υπερπλήρες
για την περιγραφή ενός παραθύρου.
Στη συνέχεια θα δούμε πως μπορούμε να υπολογίσουμε τα χαρακτηριστικά αυτά
πολύ γρήγορα (σε γραμμικό χρόνο) με τη βοήθεια του ολοκληρώματος εικόνας (Integral Image)

\subsection{Ολοκλήρωμα εικόνας}

Όπως είδαμε το πλήθος των χαρακτηριστικών για ένα παράθυρο 20x20 είναι περίπου 120.000.
Για να υπολογίσουμε ένα χαρακτηριστικό ο προφανής τρόπος είναι να αθροίσουμε τις
τιμές των pixel που απαρτίζουν κάθε ορθογώνιο. Ένας τέτοιος υπολογισμός όμως για 120.000
χαρακτηριστικά είναι αρκετά χρονοβόρος. Σε αυτή την περίπτωση μπορούμε να χρησιμοποιήσουμε
μια ενδιάμεση αναπαράσταση της εικόνας, το ολοκλήρωμα εικόνας (\emph{Integral Image})
\footnote{Στη βιβλιογραφία συχνά αναφέρεται και ως Πίνακας Προστιθέμενου Εμβαδού (
Summed Area Table)}
~\cite{Lienhart02anextended}~\cite{Lienhart2003}.
Χρησιμοποιώντας το ολοκλήρωμα εικόνας μπορούμε να υπολογίσουμε κάθε Haar feature
σε σταθερό χρόνο

Υπολογίζουμε το ολοκλήρωμα εικόνας σε ένα συγκεκριμένο σημείο $(x,y)$
χρησιμοποιώντας την ακόλουθη σχέση:
$$
ii(x,y) = \displaystyle\sum_{x'\leq x, y'\leq y} i(x',y')
$$

όπου $ii(x,y)$ είναι η τιμή του ολοκληρώματος εικόνας και $i(x,y)$ είναι η τιμή
της πραγματικής εικόνας (δλδ η τιμή χρώματος του εικονοστοιχείου )στο σημείο $(x,y)$.
Ουσιαστικά η τιμή $ii(x,y)$ είναι το άθροισμα των τιμών των εικονοστοιχείων που
βρίσκονται από πάνω και αριστερά του σημείου $(x,y)$.

Χρησιμοποιώντας τις ακόλουθες σχέσεις:
$$
s(x,y) = s(x,y-1) + i(x,y) \hspace{1cm} (1)
$$
$$
ii(x,y) = ii(x-1,y) + s(x,y) \hspace{1cm} (2)
$$
(όπου $s(x,y)$ είναι το άθροισμα των στοιχείων (pixel) μιας γραμμής και $s(x, -1)=0, ii(-1,y)=0$)
καταλήγουμε στη σχέση:

$$
ii(x,y) = i(x,y) + ii(x,y-1) + ii(x-1,y) - ii(x-1,y-1)
$$

Μπορούμε λοιπόν να υπολογίσουμε το ολοκλήρωμα εικόνας για όλα τα σημεία σε γραμμικό
χρόνο κάνοντας μόνο ένα πέρασμα από όλα τα στοιχεία της εικόνας.

%Να μπει εδώ εικόνα για Integral Image & Rotated Integral Image
%Να μπει εδώ εικόνα για τρόπο υπολογισμού του αθροίσματος ενός ορθογωνίου
\begin{figure}[htbp]
  \begin{center}
    \includegraphics[width=1.0\maxwidth]{../figures/integrimgr3.png}
    \caption{Υπολογισμός ολοκληρώματος εικόνας\label{fig:integrimg}}
   \end{center}
\end{figure}


Αν το δούμε σχηματικά, έχοντάς το Ολοκλήρωμα Εικόνας (Integral Image), κάθε κάθετο ορθογώνιο
πάνω στην εικόνα μπορεί να υπολογιστεί χρησιμοποιώντας
4 αναφορές στον πίνακα (βλ Σχήμα \ref{fig:integrimg}). Αντίστοιχα, η διαφορά μεταξύ
δύο ορθογωνίων μπορεί να υπολογιστεί με 8 αναφορές στον πίνακα. Παρακάτω θα δούμε
πως αυτά εφαρμόζονται για τον υπολογισμό των χαρακτηριστικών τύπου Haar σε σταθερό
χρόνο με όσο το δυνατόν λιγότερες πράξεις. Παραδείγματος χάρη από το Σχ. \ref{fig:integrimg}
$$
S_{ABCD} = ii_{C} + ii_(A) - ii_(B) - ii_(D)
$$

Το περιστραμμένο κατά 45\degree ολοκλήρωμα  εικόνας ~\cite{Lienhart02anextended}
~\cite{Lienhart2003} στη θέση $(x,y)$ περιλαμβάνει
το άθροισμα των τιμών όλων των στοιχείων (pixel) που βρίσκονται στο περιστραμμένο
κατά 45\degree ορθογώνιο που έχει το κατώτερο σημείο του στο σημείο $(x,y)$
(βλ Σχήμα \ref{fig:rotintegrimg}). Έτσι το 45\degree Rotated Integral Image ορίζεται ως:
$$
rii(x,y) = \displaystyle\sum_{y'\leq y, y'\leq y-|x-x'|} i(x',y')
$$

Αντίστοιχα χρησιμοποιώντας τις σχέσεις:
$$
rii(x,y) = rii(x-1,y-1) + rii(x+1,y-1) - rii(x,y-2) + i(x,y) + i(x,y-1)
$$
και
$$
rii(-1,y)=rii(x,-1)=rii(x,-2)=rii(-1,-1)=rii(-1,-2)=0
$$

μπορούμε να υπολογίσουμε τον πίνακα περιστραμμένου προστιθέμενου εμβαδού με ένα
πέρασμα όλων των στοιχείων της εικόνας, από αριστερά προς τα δεξιά και από πάνω
προς τα κάτω

\begin{figure}[htbp]
  \begin{center}
    \includegraphics[width=0.8\maxwidth]{../figures/rotintegrimg2.png}
    \caption{Υπολογισμός 45\degree-περιστραμμένου ολοκληρώματος εικόνας\label{fig:rotintegrimg}}
   \end{center}
\end{figure}

Επομένως όπως μπορούμε να καταλάβουμε και από το Σχήμα \ref{fig:rotintegrimg}
 για να υπολογίσουμε το εμβαδόν ενός περιστραμμένου ορθογωνίου εντός μιας εικόνας
 χρειαζόμαστε μόνο 4 αναφορές στον ανωτέρω πίνακα, και άρα σταθερό χρόνο.

Ας εφαρμόσουμε όμως όλη την παραπάνω λογική για να υπολογίσουμε ένα χαρακτηριστικό
τύπου Haar.

Ας δούμε τώρα, το κόστος υπολογισμού κάθε χαρακτηριστικού που χρησιμοποιούμε, με
χρήση του ολοκληρώματος εικόνας. Για τον υπολογισμό των χαρακτηριστικών που αποτελούνται από
δύο γειτονικά ορθογώνια (από τον Πίνακα \ref{tab:haarsum} τα i, ii, ix, x) θα χρειαστούμε 6 αναφορές σε πίνακα
και 6 αριθμητικές πράξεις για τον υπολογισμό των δύο ορθογωνίων και 1 αριθμητική πράξη για τη
μεταξύ τους αφαίρεση. Άρα, συνολικά 6 αναφορές σε πίνακα και 7 αριθμητικές πράξεις. Για τον
υπολογισμό των χαρακτηριστικών που αποτελούνται από τρία γειτονικά ορθογώνια (από  τον Πίνακα
\ref{tab:haarsum} τα iv-vii, xi-xiv) θα χρειαστούμε 8 αναφορές σε πίνακα και 11 αριθμητικές πράξεις (9 για τον
υπολογισμό των ορθογωνίων και 2 για τις μεταξύ τους πράξεις). Οι πράξεις τελικά μειώνονται σε 8
ακολουθώντας το σκεπτικό που φαίνεται στον Πίνακα \ref{tab:opthaar} ~\cite{Lienhart02anextended}. Τα χαρακτηριστικά viii και xv
που φαίνονται στον Πίνακα \ref{tab:haarsum}, παρότι αποτελούνται από δύο ορθογώνια, αυτά δεν είναι γειτονικά
μεταξύ τους. Έτσι, για τον υπολογισμό τους, σύμφωνα με τον Πίνακα \ref{tab:opthaar}, χρειάζονται 8 αναφορές
σε πίνακα και 8 αριθμητικές πράξεις. Για το χαρακτηριστικό iii που αποτελείται από τέσσερα
ορθογώνια, απαιτούνται 9 αναφορές σε πίνακα και 12 αριθμητικές πράξεις. Βλέπουμε, λοιπόν, ότι
όλα τα χαρακτηριστικά που περιγράφηκαν στην ενότητα 2.2.1 μπορούν να υπολογιστούν σε σταθερό
χρόνο, ανεξάρτητα από το μέγεθος του χαρακτηριστικού, με τη χρήση των δύο αυτών πινάκων. Το
γεγονός αυτό, επιταχύνει δραστικά το σύστημα ανίχνευσης.


\begin{table}[htbp]
  \centering
    \begin{tabular}{ | l | l | }
    \hline
        \includegraphics[width=0.4\maxwidth]{../figures/im1.png}
        & 3 ορθογώνια $\rightarrow$ 8 αναφορές και 11 πράξεις \\
    \hline
        \includegraphics[width=0.4\maxwidth]{../figures/im2.png}
        & 2 ορθογώνια $\rightarrow$ 8 αναφορές και 6 πράξεις \\
    \hline
        \includegraphics[width=0.4\maxwidth]{../figures/im3.png}
        &  8 αναφορές και 8 πράξεις \\
    \hline
  \end{tabular}
  \caption{Μείωση του κόστους υπολογισμού χαρακτηριστικών με 3 ορθογώνια}
  \label{tab:opthaar}
\end{table}


Η ανίχνευση αντικειμένων, θα πρέπει να γίνει σε κάθε δυνατή θέση της εικόνας, καθώς επίσης
και σε κάθε δυνατή κλίμακα. Για τον έλεγχο σε κάθε θέση, ο ανιχνευτής κινείται μέσα στην εικόνα,
διατρέχοντάς την ολόκληρη, και εφαρμόζοντας τη μέθοδο ανίχνευσης σε κάθε υποπαράθυρο. Για
τον έλεγχο σε κάθε κλίμακα, άλλες μέθοδοι δημιουργούν μια πυραμίδα από σμικρύνσεις της
εικόνας, και εφαρμόζουν σε κάθε κλίμακα της πυραμίδας τον ανιχνευτή, διατηρώντας σταθερό το
μέγεθός του. Με αυτή τη μέθοδο, πέρα από το κόστος της ίδιας της ανίχνευσης, προστίθεται και
το χρονικό κόστος της κατασκευής της πυραμίδας εικόνων, το οποίο είναι αρκετά σημαντικό. Στη
μέθοδο που εξετάζουμε, αντί να αλλάζουμε το μέγεθος της εικόνας όπου γίνεται η ανίχνευση,
αλλάζουμε το μέγεθος του ίδιου του ανιχνευτή. Αυτό είναι εφικτό, καθώς τα χαρακτηριστικά τύπου
Haar μπορούν να μεταβληθούν σε μέγεθος. Επίσης, με τη χρήση των δύο πινάκων που είδαμε
προηγουμένως, ο υπολογισμός ενός χαρακτηριστικού δεν επηρεάζεται χρονικά από το μέγεθός
του. Έτσι, η εφαρμογή της μεθόδου μπορεί να γίνει στον ίδιο ακριβώς χρόνο για οποιαδήποτε
κλίμακα του παραθύρου ανίχνευσης. Από μετρήσεις που έχουν γίνει, έχει βρεθεί ότι ο χρόνος που
χρειάζεται για την κατασκευή της πυραμίδας εικόνων που χρησιμοποιούν άλλες μέθοδοι, είναι
παραπλήσιος με το χρόνο που χρειάζεται η μέθοδος που εξετάζουμε, για όλη τη διαδικασία
ανίχνευσης [ViJo02]. Έτσι, βλέπουμε ότι οποιαδήποτε μέθοδος χρησιμοποιεί πυραμίδες εικόνων
για την ανίχνευση αντικειμένων, θα είναι αναγκαστικά πιο χρονοβόρα από την εξεταζόμενη
μέθοδο.

\subsection{Ο αλγόριθμος AdaBoost}

Στην προηγούμενη ενότητα παρουσιάσαμε το σύνολο των Haar features. Tα χαρακτηριστικά
αυτά μας βοηθούν να δούμε αν στο τρέχον παράθυρο
εντός της εικόνας που εξετάζουμε υπάρχει κάποιο αντικείμενο (στη συγκεκριμένη
περίπτωση πρόσωπο). Μας βοηθάνε δηλαδή να \emph{ταξινομήσουμε} τα επιμέρους παράθυρα
επιλέγοντας εκείνα που περιέχουν πρόσωπα. Η διαδικασία αυτή ονομάζεται ταξινόμηση
(\emph{Classification}) και ο αλγόριθμος που υλοποιεί, ταξινομητής (\emph{Classifier}).

Όπως είδαμε, τα χαρακτηριστικά που χρησιμοποιούμε μπορούν να υπολογιστούν πάρα πολύ
γρήγορα. Το σύνολο όμως των χαρακτηριστικών παραμένει μεγάλο (περίπου 120.000 για μια
εικόνα 20x20).Συνεπώς, η διαδικασία υπολογισμού του συνόλου των χαρακτηριστικών
για όλα τα παράθυρα μέσα σε μια εικόνα παραμένει χρονοβόρα ~\cite{Viola01rapidobject}.
Χρειάζεται λοιπόν, να επιλέξουμε ένα υποσύνολο των χαρακτηριστικών αυτών τα οποία θα
είναι ικανά να μας παρέχουν την πληροφορία για την ύπαρξη αντικειμένου.

Ο αλγόριθμος AdaBoost είναι ένας αλγόριθμος μηχανικής μάθησης ο οποίος χρησιμοποιείται
τόσο για την επιλογή του υποσυνόλου των χαρακτηριστικών που θα χρησιμοποιηθούν
από τον classifier, όσο και για την εκπαίδευση του ταξινομητή. Ανήκει στην κατηγορία
των boosting algorithms, χρησιμοποιείται δηλαδή για να αυξήσει την απόδοση ενός
οποιουδήποτε απλού αλγορίθμου ταξινόμησης (\emph{weak classifier}). Στην πραγματικότητα,
αυτό που κάνει ο AdaBoost είναι να φτιάξει μια αλυσίδα από από weak classifiers
χρησιμοποιώντας ένα άπληστο αλγόριθμο, ώστε να σχηματίσει τελικά από αυτούς έναν
ισχυρότερο classifier.

Η βελτίωση του ασθενούς αλγορίθμου ταξινόμησης πραγματοποιείται, καλώντας τον
αλγόριθμο να επιλύσει μια αλληλουχία προβλημάτων ταξινόμησης. Αρχικά, όλα τα παραδείγματα
(θετικά και αρνητικά) παίρνουν μια τιμή βάρους, η οποία είναι ίδια για όλα. ∆ίνονται στον
αλγόριθμο και πραγματοποιείται ο πρώτος κύκλος εκμάθησης, όπου ο
αλγόριθμος ταξινομεί όλα τα παραδείγματα με κάθε διαθέσιμη συνάρτηση ταξινόμησης. Έπειτα,
οι συναρτήσεις ταξινόμησης διατάσσονται σύμφωνα με τα αποτελέσματά τους, λαμβάνοντας υπόψη
το βάρος κάθε παραδείγματος. Επιλέγεται ένας μικρός αριθμός συναρτήσεων ταξινόμησης, από
αυτές με τα καλύτερα αποτελέσματα, που αποτελούν τον πρώτο ασθενή ταξινομητή. Ο πρώτος
κύκλος εκμάθησης ολοκληρώνεται και τα βάρη των παραδειγμάτων ισοσταθμίζονται, δίνοντας
μεγαλύτερο βάρος στα παραδείγματα που ταξινομήθηκαν λανθασμένα από τον πρώτο ασθενή
ταξινομητή. Έτσι, στον δεύτερο κύκλο εκμάθησης ο αλγόριθμος ταξινόμησης θα θεωρήσει πιο
σημαντικά τα παραδείγματα που ταξινομήθηκαν λανθασμένα από τον προηγούμενο ταξινομητή.
Τα βήματα επαναλαμβάνονται διαδοχικά, μέχρι να φτάσουμε στο επίπεδο του συνολικού λόγου
λανθασμένης ταξινόμησης που επιθυμούμε. Τελικά, ο ισχυρός ταξινομητής προκύπτει από τον
συνδυασμό των ασθενών ταξινομητών που επιλέχθηκαν και ένα κατώφλι. Κατά την διαδικασία της
ταξινόμησης ενός υποπαραθύρου εικόνας από τον ισχυρό ταξινομητή, εφαρμόζονται στο
υποπαράθυρο όλοι οι ασθενείς ταξινομητές. Τα αποτελέσματα των ασθενών ταξινομητών
αθροίζονται, και αν το άθροισμα ξεπερνά το κατώφλι του ταξινομητή, το υπό εξέταση αντικείμενο
ταξινομείται ως θετικό, αλλιώς ως αρνητικό.

%------

Ο αλγόριθμος AdaBoost διαθέτει 4 διαφορετικές εκδοχές ~\cite{Friedman98additivelogistic}.
Η αρχική εκδοχή ονομάζεται ∆ιακριτός AdaBoost (\emph{Discrete AdaBoost - DAB}) καθώς η συνάρτηση ταξινόμησης
κάθε weak classifier παίρνει 2 διακριτές τιμές ${-1,1}$ ανάλογα με το αν ένα
δείγμα ταξινομείται ως θετικό ή αρνητικό. Η δεύτερη εκδοχή ονομάζεται Πραγματικός AdaBoost
(\emph{Real AdaBoost - RAB}), καθώς η συνάρτηση ταξινόμησης έχει ως πεδίο τιμών ολόκληρο
το διάστημα $[0,1]$. Με τη χρήση του RAB, μπορούμε να έχουμε μια
ένδειξη εμπιστοσύνης για τα αποτελέσματα της ταξινόμησης, χρησιμοποιώντας τις τιμές που
επιστρέφονται από τον αλγόριθμο και όχι μόνο το αποτέλεσμα της θετικής ή αρνητικής
ταξινόμησης. Άλλη εκδοχή είναι ο LogitBoost, ο οποίος έχει δύο παραλλαγές, αυτή που
χρησιμοποιεί δύο κλάσεις και αυτή που χρησιμοποιεί J κλάσεις
Τέλος υπάρχει και ο Gentle AdaBoost, οποίος ουσιαστικά βασίζεται στον Real AdaBoost
παράγει όμως τα επιμέρους βήματα χρησιμοποιώντας τη μέθοδο Newton αντί να χρησιμοποιεί
ακριβή υπολογισμό σε κάθε βήμα.

Στη μέθοδο που χρησιμοποιήσαμε για την ανίχνευση προσώπων, κάθε ασθενής αλγόριθμος
εκμάθησης παράγει τιμές από το περιορισμένο σύνολο συναρτήσεων ταξινόμησης που αποτελούνται από ένα
μόνο χαρακτηριστικό τύπου Haar. Προφανώς, από ένα μόνο χαρακτηριστικό δε μπορούμε να
περιμένουμε ιδιαίτερα χαμηλό λόγο σφάλματος. Σε κάθε στάδιο του αλγορίθμου AdaBoost
επιλέγεται το χαρακτηριστικό που διαχωρίζει καλύτερα τα θετικά από τα αρνητικά δείγματα. Για
κάθε χαρακτηριστικό, ο weak classifier προσδιορίζει ένα κατώφλι της τιμής του
χαρακτηριστικού, που ελέγχοντάς το περιορίζονται οι λανθασμένες ταξινομήσεις από το
συγκεκριμένο χαρακτηριστικό στις ελάχιστες δυνατές. Έπειτα, επιλέγεται ως ασθενής ταξινομητής
το χαρακτηριστικό τύπου Haar, που, για το δεδομένο κατώφλι του, κάνει τη συνολικά καλύτερη
ταξινόμηση. Ο AdaBoost συνεχίζει εκπαιδεύοντας όλους τους ασθενείς ταξινομητές, μέχρι το
σημείο που ο ισχυρός συνολικός ταξινομητής επιτυγχάνει το επίπεδο ταξινόμησης που ζητάμε.

Ο αλγόριθμος AdaBoost παρέχει αρκετά ισχυρές εγγυήσεις για την ορθότητά του. Έχει
αποδειχθεί, ότι το σφάλμα ταξινόμησης του ισχυρού ταξινομητή που προκύπτει από την εφαρμογή
του αλγορίθμου, τείνει προς το μηδέν εκθετικά ως προς τον αριθμό των κύκλων εκπαίδευσης
~\cite{schapire1998}. Εξίσου σημαντικό αποτελεί το γεγονός ότι η όλη διαδικασία της
εκμάθησης πραγματοποιείται  σχεδόν σε πραγματικό χρόνο.
Για να κατασκευάσουμε έναν Classifier από τον αλγόριθμο AdaBoost με Μ επιμέρους
Weak Classifiers από ένα συγκεκριμένο πλήθος Κ χαρακτηριστικών Haar για Ν εικόνες
χρειαζόμαστε χρόνο $O(MKN)$. Αντίθετα άλλοι αλγόριθμοι χρειάζονται $O(MNKN)$ βήματα.

\subsection{Χρήση ενός Cascade Classifier}

Σε αυτή την ενότητα παρουσιάζεται η μέθοδος ταξινόμησης εικόνων με τη χρήση ενός
Cascade Classifier ~\cite{Viola01rapidobject}. Η μέθοδος αυτή βοηθά στη επίτευξη υψηλού λόγου
ανίχνευσης, μειώνοντας σημαντικά τον απαιτούμενο χρόνο. Η γενική ιδέα είναι ότι αντί για
έναν μεγάλο και χρονοβόρο classifier μπορούμε να χρησιμoποιήσουμε διαδοχικά πολλούς
μικρότερους (άρα και γρηγορότερους) classifiers. Στη συγκεκριμένη περίπτωση χρησιμοποιούμε
απλούστερους και πολύ γρήγορους ταξινομητές (ελέγχουν λιγότερα haar features) αρχικά,
οι οποίοι θα απορρίπτουν γρήγορα την πλειονότητα των αρνητικών υποπαραθύρων,
και πιο σύνθετους και χρονοβόρους που ελέγχουν περισσότερα haar features αργότερα ,
ώστε να μειώσουμε το λόγο λανθασμένων ανιχνεύσεων.

Κάθε υποπαράθυρο της εικόνας εισέρχεται στον πρώτο ταξινομητή. Αν ο ταξινομητής το κατατάξει ως
θετικό, αυτό περνά ως είσοδος στον δεύτερο ταξινομητή. Αν και αυτός το κατατάξει ως θετικό,
τότε περνά στον τρίτο κ.ο.κ. Αν σε αυτή τη διαδοχή κάποιος ταξινομητής κατατάξει το
υποπαράθυρο ως αρνητικό, τότε αυτό απορρίπτεται και δεν εξετάζεται από κανένα άλλο
ταξινομητή (βλέπε Σχήμα \ref{fig:cascadeclassifier}). Θα μπορούσαμε να παρομοιάσουμε
αυτή τη διαδικασία με έναν μεγάλο ταξινομητή που αποτελείται από το σύνολο των χαρακτηριστικών
Haar, όπου όμως δεν περιμένουμε να υπολογιστεί όλο το πλήθος των χαρακτηριστικών.
Αντίθετα, σε κάθε στάδιο ελέγχονται μερικά χαρακτηριστικά και ανάλογα με το άθροισμα
των τιμών τους ο ταξινομητής αποφασίζει αν το υπό εξέταση υποπαράθυρο απορρίπτεται ή όχι.

\begin{figure}[htbp]
  \begin{center}
    \includegraphics[width=0.5\maxwidth]{../figures/cascadeclassifier.png}
    \caption{Τα επιμέρους βήματα ενός Cascade Classifier\label{fig:cascadeclassifier}}
   \end{center}
\end{figure}

Κάθε ταξινομητής του Cascade Classifier εκπαιδεύεται χρησιμοποιώντας ένα σύνολο θετικών και ένα
σύνολο αρνητικών παραδειγμάτων. Το σύνολο των θετικών παραδειγμάτων είναι το ίδιο
κατά την εκπαίδευση κάθε ταξινομητή. Το σύνολο των αρνητικών παραδειγμάτων όμως,
μεταβάλλεται. Συγκεκριμένα, κάθε ταξινομητής εκπαιδεύεται χρησιμοποιώντας ως
αρνητικά παραδείγματα, τα παραδείγματα που ταξινομούνται από τους προηγούμενους ταξινομητές
ως θετικά ~\cite{Viola01rapidobject}. Αυτό αυξάνει σε πολύ μεγάλο βαθμό τα
αρνητικά παραδείγματα τα οποία θα εξεταστούν συνολικά. Για να φτάσει ένας συγκεκριμένος
αριθμός αρνητικών παραδειγμάτων στον τρέχοντα ταξινομητή, θα πρέπει τα παραδείγματα αυτά
να ταξινομηθούν από όλους τους προηγούμενους ταξινομητές (λανθασμένα) ως θετικά.
Ας θεωρήσουμε ότι σε κάθε στάδιο ενός Cascade Claasifier θέλουμε να εξετάζονται 1.000 αρνητικά
παραδείγματα και κάθε στάδιο έχει λόγο λανθασμένης θετικής ανίχνευσης 0,5. Τότε, για να
περάσουν στο δέκατο στάδιο 1.000 αρνητικά παραδείγματα, αυτά θα χρειαστεί να έχουν
ταξινομηθεί από τα προηγούμενα εννιά στάδια ως θετικά. Έτσι, συνολικά θα πρέπει να εξεταστούν
περίπου 512.000 αρνητικά παραδείγματα.

Η εξέταση πολύ μεγαλύτερου αριθμού αρνητικών παραδειγμάτων αυξάνει την τελική απόδοση
του CC. Αντίθετα, κάθε ταξινομητής καλείται να πραγματοποιήσει μια πιο δύσκολη ταξινόμηση
από αυτές των προηγούμενων ταξινομητών. Τα αρνητικά παραδείγματα που θα έχει στη διάθεσή
του θα είναι πιο δύσκολα στην ταξινόμηση από τα παραδείγματα που είχαν τα προηγούμενα από
αυτό στάδια. Έχοντας, λοιπόν, πιο δύσκολο σύνολο εκπαίδευσης ένας ταξινομητής που βρίσκεται
σε προχωρημένο στάδιο, θα παρουσιάσει αυξημένες λανθασμένες ταξινομήσεις, θετικές και
αρνητικές.

Οι απλοί επιμέρους ταξινομητές, θα πρέπει να έχουν πολύ χαμηλό λόγο λανθασμένων
αρνητικών ταξινομήσεων, ώστε να μην χάνονται τα πραγματικά αντικείμενα στη συνολική
ταξινόμηση. Για να διασφαλίσουμε τη σωστή λειτουργία του CC, θα πρέπει να αυξήσουμε
περαιτέρω τις θετικές ταξινομήσεις (είτε αφορούν πραγματικά αντικείμενα είτε όχι)
~\cite{Viola01rapidobject}. Μια τεχνική για να πετύχουμε αυτό το αποτέλεσμα είναι
να επέμβουμε στις τιμές των κατωφλιών των ταξινομητών που προσδιόρισε ο αλγόριθμος
AdaBoost κατά την εκπαίδευση. Το κατώφλι ενός ταξινομητή ορίζει την ελάχιστη τιμή
του σταθμισμένου με βάρη αθροίσματος των τιμών των χαρακτηριστικών που θα πρέπει
να έχει ένα υποπαράθυρο για να ταξινομηθεί ως θετικό. Ένα υποπαράθυρο ταξινομείται,
δηλαδή, ως θετικό, όταν το (σταθμισμένο) άθροισμα των τιμών των χαρακτηριστικών που
υπολογίστηκαν για αυτό ξεπερνά το κατώφλι του ταξινομητή. Έτσι, μειώνοντας τις
τιμές των κατωφλιών θα αυξηθεί ο αριθμός των παραθύρων που ταξινομούνται ως
θετικά, άρα και ο λόγος θετικών ταξινομήσεων.

Ο ολοκληρωμένος CC που χρησιμοποιήθηκε για την ανίχνευση προσώπων αποτελείται
από 38 επιμέρους στάδια και ένα σύνολο 6000 χαρακτηριστικών. Ο χρόνος που
απαιτείται για να τρέξει ο CC είναι άμεσα συνδεδεμένος με τον συνολικό αριθμό των
χαρακτηριστικών που υπολογίζονται για κάθε υποπαράθυρο που εξετάζεται.
Με βάση αξιολογήσεις που έχουν γίνει ~\cite{Viola01rapidobject}~\cite{Viola2004}
πάνω στο MIT-CMU test set ~\cite{Rowley:1998:NNF:275341.275344} υπολογίζονται κατά
μέσο όρο 10 Haar features από τα συνολικά 6061 ανά υποπαράθυρο. Αυτό οφείλεται
στο γεγονός ότι το μεγαλύτερο ποσοστό από τα υποπαράθυρα απορρίπτεται σε πρώτα
στάδια.

Η απόδοση του CC εξαρτάται επίσης από το πλήθος των επιμέρους σταδίων και την
απόδοση του καθενός από αυτά.


\section{Άλλες μέθοδοι που έχουν χρησιμοποιηθεί}\label{sec:othermethods}

Στο κεφάλαιο αυτό κάναμε μια σύντομη αναφορά στις πιο βασικές μεθόδους αναγνώρισης
προσώπου σε εικόνα με βάση την απόδοση που επιτυγχάνουν και το χρόνο εκτέλεσής τους.
Οι τεχνικές αυτές βασίζονται κατά κύριο λόγο στον υπολογισμό κάποιων χαρακτηριστικών
σκίασης από την εικόνα σε συνδυασμό με κάποιο προεκπαιδευμένο μοντέλο αναγνώρισης.
Γενικά, στο επιστημονικό πεδίο αυτό έχουν προταθεί αρκετές μέθοδοι με βάση διαφορετικά
χαρακτηριστικά της εικόνας. Μερικά από αυτά είναι το χρώμα, η κίνηση καθώς και
συνδυασμός αυτών. Ενδεικτικά αναφέρουμε μερικές τεχνικές.

Τεχνικές σε ελεγχόμενο περιβάλλον (πχ μονοχρωματικό φόντο)
\begin{description}
  \item[Με βάση το χρώμα] \hfill \\
    \begin{itemize}
        \item Explanation of basic color extraction for face detection
            \flink{http://web.archive.org/web/20040815172250/http:/www.dcs.qmul.ac.uk/research/vision/publications/papers/bmvc97/node2.html}
        \item Face detection in color images using PCA
            \flink{http://web.archive.org/web/20070621092425/http:/www.ient.rwth-aachen.de/forschung/bebi/facedetection/publi/men99b.pdf}
    \end{itemize}
  \item[Με βάση την κίνηση] \hfill \\
    \begin{itemize}
        \item Explanation of basic motion detection for face finding
            \flink{http://web.archive.org/web/20080522171806/http:/www.ansatt.hig.no/erikh/papers/hig98_6/node2.html}
        \item Blink detection: human eyes are simultaneously blinking
            \flink{http://www-prima.imag.fr/ECVNet/IRS95/node13.html}
    \end{itemize}
  \item[Με συνδυασμό των δυο προηγουμένων] \hfill \\
    \begin{itemize}
        \item A mixture of color and 3D
            \flink{http://people.eecs.berkeley.edu/~trevor/papers/1998-021/}
        \item A mixture of color and background removal
            \flink{http://web.archive.org/web/20030821151710/http:/atwww.hhi.de/blick/Head_Tracker/head_tracker.html}
    \end{itemize}
\end{description}

\chapter{Ganeti backend}\label{ch:ganeti}

This chapter concentrates on Ganeti. We will discuss the architecture of Ganeti,
and we will analyze in details its most basic parts and how they interact with
each other. We will try to provide small documentation to familiarize the reader
with this tool, avoiding presenting superfluous information. Summing up, the
objective for the reader is at the end of this chapter to have a comprehensive
view of Ganeti and its basic structure.

\section{Overview}

Ganeti is a software tool to manage computer clusters, and also assumes the
management task of the virtual instances of the cluster. Is is being
developed by Google and is an Open Source Project since 2007. It is build on top
of existing virtualization technologies, such as XEN or KVM hypervisors, and
other open source software. It also uses LVM for disk management, optionally
DRBD for disk replication across physical hosts, and other disk templates such
as RBD, external shared storage providers and more.

Ganeti uses a daemon-based model. Each daemon deals with specific tasks that the
cluster has to face, and communicates with other daemons using various
protocols, mainly HTTP based, and custom ones like LUXI. Many of those daemons
are written in Haskell, but most of the project's code is in Python. Ganeti is
actually a wrapper around hypervisors. Once installed, the tool will take over
the management part of virtual instances. It also makes it convenient for system
administrators to setup and handle clusters of physical nodes.

Some of the main features that Ganeti provides, and controls are the following:

\begin{itemize}
  \item Cluster management of physical nodes.
  \item Support for XEN virtualization.
  \item Support for KVM virtualization.
  \item Support for virtual console to control instances ,e.g, VNC.
  \item Support for live instance migration.
  \item Support for virtio or emulated devices.
  \item Disk management: plain LVM volumes, files, Across-the-network RAID1
        (using DRBD) for quick recovery in case of physical system failure.
  \item Export/import mechanisms for backup purposes or migration between
        cluster.
  \item Fast and simple recovery after physical failures using commodity
        hardware.
\end{itemize}

\section{Terminology}

In this section, we provide a small introduction of the most basic Ganeti terms,
in order to facilitate the reader in the rest of the document. In
Figure~\ref{fig:gnt_abst}, we present an abstract Ganeti architecture, which will
help us to briefly explain those terms.

\begin{figure}[htbp]
  \begin{center}
    \includegraphics[width=1.0\maxwidth]{../figures/ganeti_abst_arch.pdf}
    \caption{Ganeti base components, version 2.7.2\label{fig:gnt_abst}}
   \end{center}
\end{figure}

\begin{description}
  \item[Cluster] \hfill \\
    A set of computers (nodes) working together to provide a coherent, reliable,
    scalable, highly-available virtualization service under a single domain.
  \item[Node] \hfill \\
    A physical machine which corresponds to the basic cluster infrastructure. If
    they host no instances, nodes can be added and removed at will from the
    cluster. They do not have to be fault-tolerant in order to achieve
    high availability for the instances they host. The loss of a single node,
    in a HA cluster, will not cause disk data loss for any of the instances it
    hosts.

    A node belonging to a cluster can serve different roles; VM-hosting and/or
    Administrative roles, which will be explained later in this document.
    Independently of its role, nodes can be in different statuses like online,
    drained or offline. Depending on the role they serve, each node will run a
    set of daemons:

    \begin{itemize}
      \item ganeti-masterd
      \item ganeti-noded
      \item ganeti-rapid
      \item ganeti-confd
    \end{itemize}
  \item[Instance] \hfill \\
    A virtual machine that runs on a cluster. Instances in Ganeti are
    highly-available entities, that can also become fault-tolerant depending on
    the disk template they use, e.g., DRBD. An instance has various parameters
    which can be modified either at instance level or at cluster level via
    cluster default parameters. Those parameters can be classified in three
    main categories: hypervisor related parameters, called \texttt{hvparams},
    general parameters, called \texttt{beparams}, and network interface
    parameters, called \texttt{nics}.
  \item[Disk Template] \hfill \\
    The layout disk type for the instance. Instances in Ganeti see the same
    virtual drive in all cases, but the node-level configuration varies between
    them. The available storage templates are the following:
    \begin{description}
      \item[diskless] \hfill \\
        This creates an instance with no disks. Useful for testing purposes, or
        other special cases.
      \item[file] \hfill \\
        Disk devices will be regular plain files. No redundancy is provided.
      \item[sharedfile] \hfill \\
        Disk devices will be regular plain files under a shared directory. This
        option allows live migration and failover of instances.
      \item[plain] \hfill \\
        Disk devices will be LVM volumes.
      \item[drbd] \hfill \\
        Disk devices will be drbd on top of LVM volumes, compatible with DRBD
        versions 8.x.
      \item[rbd] \hfill \\
        Disk devices will be rbd volumes, short for RADOS block device, residing
        inside a RADOS cluster.
      \item[blockdev] \hfill \\
        Pre-existent block devices will be used as backend for its disks.
      \item[ext] \hfill \\
        The instance will use an external storage provider as disk backend,
        through the ExtStorage Interface, using ExtStorage providers.
    \end{description}
  \item[\emph{Primary} and \emph{Secondary} concepts] \hfill \\
    An instance has a primary node, and depending on the disk configuration
    chosen might also have a secondary one. Every DRBD instance runs in its
    primary node and uses the secondary for disk replication and
    fault-tolerance. When those terms used in node level, they refer to the
    instances having the given node as primary and secondary, respectively.
  \item[IAllocator] \hfill \\
    A framework for using external user-provided scripts to automatically
    compute the placement of new instances on the cluster nodes. This
    eliminates the need to manually specify the exact locations of an instance
    addition/move, and make the node evacuate operations an easy, and common
    cluster operation.

    In order for Ganeti to be able to use those scripts, we should place them
    under the \texttt{\$libdir/ganeti/iallocators} folder path.
  \item[Jobs and OpCodes] \hfill \\
    A \emph{Job} in Ganeti is the basic operation to modify the cluster's state.
    A job consists of multiple \emph{OpCodes} internally, short for ``Operation
    Code". This is the basic element of operation in Ganeti. Most of the
    commands in Ganeti are equivalent to one opcode, or in some cases a sequence
    of opcodes, all of the same type ,e.g., shutting down all instances in a
    cluster. The opcodes of a single job are processed serially, but different
    jobs can be executed in parallel, in different order than they have been
    submitted, depending on hardware resource availability, locks, or priority
    given by user.
\end{description}

\section{Architecture}\label{sec:architecture}

As we mentioned earlier in this section, Ganeti has a daemon-based architecture.
Every Ganeti-related command, (\texttt{gnt-*} commands), is an individual client
which ``talk" to the master daemon who executes every cluster operation.

In Figure~\ref{fig:gnt_arch}, we present the architecture of a Ganeti cluster in
a more detailed form than in Figure~\ref{fig:gnt_abst}, and we show how the most
basic daemons and elements interact with each other.

\begin{figure}[htbp]
  \begin{center}
    \includegraphics[width=1.0\maxwidth]{../figures/ganeti_arch_horizonal.pdf}
    \caption{Ganeti architecture, version 2.7.2\label{fig:gnt_arch}}
   \end{center}
\end{figure}

\emph{Nodes} are the basic cluster infrastructure in Ganeti. They serve
different roles and they can, and usually do, serve more than one. We could
group node roles into two major categories; The \emph{Administrative}
and the \emph{VM-capable} nodes. Nodes belonging in the first category, can
modify the cluster state, or take part in cluster related decisions like the
master node voting procedure. Nodes in second category can simply hosts
instances (VMs).

In more details, a node can belong in one, or more of the following roles:

\begin{description}
  \item[Master] \hfill \\
    It is the cluster coordinator node and it holds the authoritative copy of
    the cluster configuration. Every decision that could affect the cluster
    state managed by this node, because it is the only node which can execute
    commands. Only one master should exist every time in a Ganeti cluster.
  \item[Master Candidate] \hfill \\
    Nodes in this category have the full copy of the live cluster configuration
    and jobs. Only nodes belonging in this role can become master. This set of
    nodes called \emph{candidate pool}, and there is also a parameter called
    \texttt{candidate\_pool\_size}, which represents the number of candidates
    the cluster tries to maintain, automatically. Because the
    \texttt{candidate\_pool\_size} can have a huge impact in Ganeti performance,
    for reasons which we 'll explain later, it can be configured during
    initialization or modified via cluster related commands (\texttt{gnt-cluster
    *}).
  \item[Master Capable] \hfill \\
    Nodes in this category are not master candidates, but can become and
    promoted to master node, in cases when nodes in the candidate pool
    are less than the desired size. In such case, a randomly selected master
    capable node is promoted to master candidate. We could disable this flag
    and exclude some nodes from being master candidates in case when they have
    a less reliable hardware and we do not want to store sensitive information
    to them.
  \item[VM Capable] \hfill \\
    This is the default node state and means that the node can host instances.
    More specifically, the node will participate in instance allocation
    operations, capacity calculations, cluster checks, and other operations.
  \item[Offline] \hfill \\
    Nodes having this flag set have some special characteristics. They are still
    recorded in the Ganeti configuration, and can only take part in the
    master voting procedure, to ensure consistency. They are not allowed to
    become master. Enabling this flag to a master candidate node will demote
    it from candidate possibly, causing another node which is master capable to
    be promoted. Additionally, these nodes are not allowed to host primary
    instances. The main reason this role was added to Ganeti was to allow
    broken machines that are being repaired to remain in the cluster without
    introducing further problems.\\
  \item[Drained] \hfill \\
    Nodes in this state, will not participate in instance allocation operations,
    but all other operations as queries, or starting and stopping instances, are
    working without any restrictions. The actual intention is that nodes in this
    role have some issue and they are being evacuated for hardware repairs.
\end{description}

Previously, we made some references to Ganeti \emph{Cluster Configuration} and
\emph{Jobs} which are stored in the system as files. These files are stored
under the \texttt{/var/lib/ganeti} directory, and actually form a database for
the cluster. Every single piece of information that the cluster needs to operate
normally is stored in those files.

\subsection{Cluster Configuration}\label{subsec:config}

The cluster configuration is a set of files which are present in all, or a
subset of nodes depending on their usage. We could group them into three main
categories, namely:

\begin{description}
  \item[config.data] \hfill \\
  The cluster configuration database is a single JSON config file called
  \texttt{config.data}. The master node keeps a valid version of this file and
  it also replicates it to the master candidate nodes, for reliability
  reasons. Ganeti has a special way to handle \texttt{config.data} updates; It
  holds the \texttt{config.data} both in master memory and disk. The canonical
  version of the config exists at every moment in the master node memory, and
  the disk version will be updated from there. Every operation that updates a
  single object in the memory version of the \texttt{config.data}
  automatically causes a flush of the whole file to disk. If the config
  does not flushed successfully to disk, the operation will fail. The
  \texttt{config.data} file contains information about all the major Ganeti
  objects such as cluster, nodes, instances, networks, and their attributes.
  We will extensively talk about the \texttt{config.data} structure in the
  next chapters.

  \item[ssconf\_*] \hfill \\
  Besides the objects contained in the \texttt{config.data}, which change
  quite often, Ganeti also holds a set of configuration files which contain
  information that does not change frequently and needs to be present to all
  Ganeti nodes. These files are stored in the same directory as the
  \texttt{config.data} file and start with a \texttt{ssconf\_} prefix. For
  example, the ssconf file which contains the master IP value is called
  \texttt{ssconf\_master\_ip}, and so on. The main reason for the existence of
  the ssconf files is that the most frequent Ganeti operations should not need
  to contact the master node and overload him. In addition, we want some
  information to be accessible at every moment, even if the master node is
  down, so that we can use it from services external to the cluster, and
  avoiding the single point of failure that a master hard shutdown could
  introduce.

  \item[SSL certificates] \hfill \\
  Ganeti uses OpenSSL for encryption on the RPC layer, and SSH for executing
  commands. These SSL certificates are stored under the same directory as the
  rest of the configuration files and exist in all Ganeti nodes. The SSL
  certificates are automatically generated when the cluster is initialized, and
  are copied to the newly added nodes automatically along with the master's SSH
  host key. The cluster SSL key is stored in the \texttt{server.pem} file.
  There is a similar key for the RAPI daemon, the \texttt{rapi.pem} file. The
  \texttt{spice.pem} and \texttt{spice-ca.pem} files are used by SPICE
  connections to the KVM hypervisor, the \texttt{hmac.key} is used by the
  \texttt{ganeti-confd} daemon, the \texttt{cluster-domain-secret} file is used
  to sign information exchanged between separate clusters via a third party,
  and finally the Ganeti \texttt{known\_hosts} file are all the certificates
  maintained by Ganeti.
\end{description}

\subsection{Jobs}\label{subsec:jobs}

Jobs are the basic Ganeti operation, and the only way to modify the cluster's
state. They are stored as individual files in the file system, and serialized
using JSON format, which is the standard Ganeti serialization mechanism. A
job consists of one or more opcodes. That list of opcodes is processed
serially, and if an opcode fails, later opcodes are no longer processed and
the entire job will fail.

At any time, each job and each opcode can be, in a different status depending
on the stage of its execution. The job status is actually the status of its
first processed opcode. A complete status description follows:
\begin{description}
  \item[Queued] \hfill \\
    The job/opcode has been submitted, but has not been processed yet.
  \item[Waiting] \hfill \\
    The job/opcode is waiting for locks, or other factors, to proceed.
  \item[Running] \hfill \\
    The job/opcode is currently being executed.
  \item[Canceled] \hfill \\
    The job/opcode is waiting for locks, but is has been marked for
    cancellation by the user. It will never return to the \emph{Running}
    status.
  \item[Success] \hfill \\
    The job/opcode ran and finished successfully.
  \item[Error] \hfill \\
    The job/opcode has failed while executing, or the master daemon stopped
    before the job finishes its execution.
\end{description}

While job opcodes execute serially, jobs do not. Their execution order depends
on a variety of factors, apart from their incoming order, like their ability to
acquire all necessary locks, their priority, or probable dependencies with
other jobs. At any time, there are jobs that can be in one of the above
statuses. Similarly to the global cluster configuration files, jobs are stored
under a directory in the configuration path called \emph{queue}, which is
located by default under the \texttt{/var/lib/ganeti/queue} path.
The \emph{Job Queue} structure speeds
up operations because every job which is ready for execution can run
independently and so in parallel with the other jobs. In addition, storing
the jobs in a common folder makes it more convenient for the user to handle
and watch their progress, independently of Ganeti, and it also makes the
consistency checks that Ganeti does to the job list and jobs themselves a
simpler procedure. Queue structure also gives us the choice to prevent new
jobs from entering it by enabling the \emph{drained flag}. This is a feature
used mainly in cases when we have to make maintenance-related operations to
the cluster and we do not want any new incoming jobs affecting us.

Besides the regular jobs, the \emph{Job Queue} structure always contains three
more files, even if there are not any jobs running, pending, or canceled in
the cluster. Those are the \texttt{version} file, which denotes the queue
format version, the \texttt{lock} file, which is opened by the queue managing
process in exclusive mode, and the \texttt{serial} file, containing the last
job ID value used. Listing~\ref{lst:job_queue}, presents a high-level view of
the Ganeti Job Queue structure:

\includeminted[text]{../listings/job_queue.txt}{%
  Job Queue structure}{lst:job_queue}{frame=single}

In Listing~\ref{lst:job_structure}, we present the internal structure of a
randomly-selected job in Ganeti, with its most basic fields:

\includeminted[text]{../listings/job_structure.json}{%
  Job structure}{lst:job_structure}{linenos}

As we notice from this listing, a job consists of the \texttt{ops} field, short
for opcodes, the job \texttt{id}, and three \texttt{timestamp} fields that
indicate when the job passed from the various statuses we presented earlier.
This job, consists of an opcode list with a single element, and
represents an instance create operation as indicated by the \texttt{OP\_ID}
field, i.e, \texttt{OP\_INSTANCE\_CREATE}. Every opcode also has its own
timestamps, as the job, so that the user can be informed with more details
about the exact time every single opcode passed from the various statuses.
The rest fields presented, are related to the specific opcode, to make the
reader better understand how Ganeti stores and handles the parameters we
give in a commonly used job like the creation of an instance. These are
the iallocator algorithm, the hypervisor chosen, and the opportunistic
locking choice for the lock retrieval, a topic that we 'll cover later in
Section~\ref{subsec:locks}.

Besides the above fields, we distinguish two important fields that will help
us to better understand how a job executes by the Ganeti processor; the
\texttt{priority} and \texttt{depends} fields.

The job \texttt{priority} is an integer number; the lower the number the
higher the opcode's priority is. This is a very helpful attribute in job
queue handling, because in cases we want to run a job like an emergency
shutdown as soon as possible, we want to overcome factors that could
delay us. The priority range is [-20..19], and jobs submitted without
priority assigned the default zero value. To avoid starvation, a job can
change its own priority after a certain amount of retries, or a certain
amount of time. One interesting thing is that opcodes also have their own
priorities. So, the job priority is the same as its first unprocessed opcode.
This behavior, combined with the fact that the job processor returns the job
back to the queue after each opcode completion, means that if there are opcodes
of higher priority submitted in the meantime of a job execution, these will
first try to acquire their locks and as result the job that was
\texttt{Running} will go to the \texttt{Waiting} status again. That behavior
makes the job queue structure a lot more versatile.

The \texttt{depends} field of a job, is an optional property which defines
dependencies on other jobs. Clients can submit jobs in the right
order and proceed to wait for changes made to them. The master daemon will
take care of everything, Section~\ref{subsec:daemons} covers that topic. Jobs
waiting for dependencies are in the \texttt{Waiting} status. In Listing
\ref{lst:job_deps}, we present a simple example of job dependencies:

\includeminted[text]{../listings/job_deps.txt}{%
  Job dependency diagram}{lst:job_deps}{frame=single}

The job queue must be consistent between the master node and the master
candidates, just like the cluster configuration files. Failures to replicate a
job to other nodes will be only flagged as error in the master daemon log if
more than half of nodes fail to copy it, otherwise the failure will be ignored
and the operation will continue normally. This relies on the fact that the
next update for already running jobs, will retry the update.

Now we will present the job execution procedure from a high-level view; the
\emph{``Life of a Ganeti job"}, from the time the user submits it till its
completion:

\begin{enumerate}
  \item Client submits the job. The appropriate opcode or a list of opcodes
        if the job consists of multiple opcodes, will be built in the client
        side. The opcode contains all the available information that Ganeti
        needs to execute the operation. For example, an
        \emph{OpInstanceCreate} opcode contains the name of the instance,
        the os\_type, the hypervisor, or instance-related parameters such as
        the beparams, the hvparams, the nicparams, and so on.
  \item Then the list of the opcode\{s\} will be sent via the LUXI protocol
        in the master daemon, who will generate a new job identifier depending
        on the value of the serial file, and it will assign it to the job.
        Then the master daemon writes the job to his local disk and replicates
        it to the master candidates. The job must be copied successfully at half
        of the candidates at least, otherwise the operation will fail. Then,
        the identifier is returned to the client using the LUXI protocol again.
  \item After the job\_id is returned to the client, the master daemon builds
        the job object named \texttt{\_QueuedJob}, and adds a new task to
        the workerpool. The task is the job object and the workerpool is a
        heap queue. The tasks are ordered in the heap queue in respect to
        the job's priority primarily, and if the priorities match, to an
        increasing number which denotes their incoming order.
  \item As soon as a new task is added to the heap queue, a pool of job queue
        workers with currently 25 threads will be notified for new arrivals.
        Those threads wait for new jobs to arrive. If all threads
        are busy, the job will have to wait until one of them become
        available. The first worker finishing its work will grab it.
        Otherwise one of the waiting threads will pick up the new job.
  \item When a job assigned to a worker it is time for the job to start its
        execution. The worker does not know nothing about the opcodes that
        the job contains. He just passes the opcode to the Ganeti's
        processor who dispatches them to the appropriate Logical Units,
        the \emph{LUs} in short. There is a Logical Unit for each Ganeti
        opcode which knows how to deal with it. The LU is the part of Ganeti
        which finally executes the operation which will modify the
        cluster's state. The rest responsibilities of a worker thread include
        the appropriate handling of the job queue lock, the notification of
        other threads when it finishes its work, and generally taking care of
        the job's smooth execution.
  \item If the user chooses to wait for job status updates and does not make
        use of the \texttt{--submit} flag, he waits by calling a waiting RPC
        function. The mechanism underlying the waiting function is an
        inotify manager who responds to events happen in the job file
        located to disk. In this case, log messages may be shown to the user
        depending on the job. The user can also cancel the job while it is
        waiting in the queue.
  \item The client can also archive the job, which then moved to a history
        directory called \emph{archive} ,i.e., the default path is the
        \texttt{/var/lib/ganeti/queue/archive} directory. This can be done in
        order to speed up the queue handling, because by default, jobs in the
        archive directory do not touched by any function.
\end{enumerate}

\subsection{Ganeti Daemons}\label{subsec:daemons}

We have already made a few references to the Ganeti daemons in previous
sections. Now we will talk in more details about the internal structure of
Ganeti, and particularly the set of daemons that it is divided into. Ganeti
consists of a growing number of daemons. Each of these deals with a specific
task that the cluster has to face, and communicates with the rest using a
variety of protocols. Specifically, as of Ganeti version 2.7, we have four
daemons. The situation is as follows:

\begin{description}
  \item[Master Daemon] \hfill \\
    The master daemon runs on a single node only, the master node. Currently is
    written in python and deals with every cluster operation. It is the Ganeti's
    heart because it is responsible for the overall cluster coordination.
    Without it, no modification can be performed on the cluster. This is the reason
    why it is the most heavy loaded daemon of all. It receives the commands
    given by the clients, either through the Command Line
    tools or the Remote API, parses them, and executes the appropriate
    operation. Creates, and manages the jobs that will execute those commands,
    handles the locks, and ensures that race conditions will never occur. It is
    also responsible for managing, and maintaining the cluster configuration files,
    updating them when it is necessary, and replicating them to the master
    candidates, in addition to the job queue. Each job is managed by a separate
    python thread. The basic python threads which managed by the master daemon
    are presented below:

    \begin{itemize}
      \item \emph{The main I/O thread}: It is a single thread. The
            \texttt{masterd} is build around this thread. It accepts connections
            in the master socket and setups/shutdowns the other thread pools.
      \item \emph{The job queue worker threads}: This pool consists of 25
            threads, each of which executes the jobs submitted by the clients.
            They are long-lived threads and are initialized during the daemon
            startup.
      \item \emph{The client worker threads}: The client worker pool contains 16
            threads. They handle the connections in the master socket, one thread
            per connected socket, parse LUXI requests, and send data back to the
            clients. They are also being built during daemon startup.
      \item \emph{The RPC worker threads}: This is not actually a pool like the
            above two categories. The thread size depends on the RPC call;
            single-node or multi-node. They interact with nodes using HTTP based
            RPC calls.
    \end{itemize}

    The \texttt{masterd} keeps some interaction paths for the communication with
    the rest Ganeti daemons. More specific, the interaction between the Command
    Line tools which are located in the master node, and the RAPI daemon is done
    with a custom protocol called LUXI. LUXI is a UNIX-socket based protocol of
    JSON-encoded messages. The UNIX socket permissions itself will determine the
    access rights. The LUXI API allows both job related operations, and cluster
    query functions.

    The communication between the master daemon and the rest node daemons is
    done through RPC calls, using HTTP\{S\} simple PUT/GET of JSON-encoded
    messages. Communication between master and nodes is protected using SSL/TLS
    encryption. Both the clients and the server must have the cluster-wide
    shared SSL/TLS certificate, and verify it when establishing the connection
    by comparing fingerprints. For highly-traffic commands like image dumps, or
    low level commands such as restarting the \texttt{node-daemon}, a simple SSH
    protocol is used. The master node must share the cluster-wide shared SSH key
    with the rest nodes of the cluster.

    During startup, the \texttt{masterd} will confirm in coordination with the
    node daemons that the node it is running, is the master node of the cluster,
    indeed. This is done via a voting procedure where all the nodes take part,
    even the offline ones. For successful confirmation the \texttt{masterd} has
    to get half plus one positive answers. When the \texttt{masterd} receives a
    SIGINT or SIGTERM signal, it stops accepting new jobs, and prepares to shut
    down as soon as the jobs that are currently running finish their execution.
    At the meanwhile, it still answers to LUXI
    requests. Pending jobs are re-added to the queue in \texttt{Queued} state
    after the daemon restarts. If a hard shutdown requested the cluster may be
    leaved in an inconsistent state.

    The current Ganeti daemon structure suffers from many performance problems
    caused by the various protocols involved in interaction between daemons, and
    by the many python threads that are created which increase lock contention,
    log pollution and memory usage. This is the reason why from version 2.9,
    Ganeti daemon subdivision will change to improve the current situation.
  \item[Node Daemon] \hfill \\
    The \texttt{noded} runs on all the nodes of a cluster. It is also written in
    python, and it is responsible for receiving the requests made by the
    \texttt{masterd} over RPC, and executes them using the appropriate backend
    tool ,e.g., hypervisors, DRBD, LVM. It executes almost all operations that
    modify the node's state, like creating disks for instances, activating disks,
    starting/stopping an instance and so on. If a \texttt{noded} stops, the
    \texttt{masterd} will not be able to talk to this daemon, but the instances
    will not be affected.
  \item[Rapi Daemon] \hfill \\
    The \texttt{rapid} is written in python, and runs automatically on the master
    node only. By default, listens on TCP port 5080 and uses SSL/TLS encryption.
    Both those parameters can change via command line. Ganeti supports a Remote
    API protocol which is JSON over HTTP, designed over the REST principle, for
    enabling communication with external clients, to easily retrieve information
    about the cluster state or modifying it. The \texttt{ganeti-rapid} waits for
    requests issued remotely through that protocol. Then, it forwards them via
    the LUXI protocol to the master daemon to deal with them.

    \texttt{Rapid} reads its users and their rights from a file on startup,
    which is usually located under the \texttt{/var/lib/ganeti/rapi/users} path.
    Changes to that file will be loaded automatically. Most query operations
    are allowed without authentication. Modification operations though, require
    authentication in order to be executed.
  \item[Configuration Daemon] \hfill \\
    The configuration daemon is written in Haskell and runs on all master
    candidate nodes, since the configuration exists only on that group of nodes.
    This daemon is used to answer queries related to the configuration of a
    cluster. It makes sure that we have a highly-available and very fast way to
    query cluster configuration values. The \texttt{config.data} is reloaded
    automatically from disk every time it is updated. The requests are made
    through an HMAC authenticated JSON-encoded custom protocol over UDP, and
    meant to be used by parallel querying all the master candidates, or a
    subset of them, getting the most up to date answer by comparing the
    value of the \texttt{config.data}'s serial number, named \texttt{serial\_no}.
    The queries are answered from a cached copy of the config which it keeps in
    memory, so no disk space is required in order to get an answer. Queries are
    also contain a ``salt" which they expect the answers to be sent with, and
    clients are supposed to accept only answers which contain salt generated by
    them. The configuration daemon answers simple queries such as:

    \begin{itemize}
      \item master node
      \item master candidate, offline nodes
      \item instance list, primary nodes
      \item cluster info
      \item job list, and more
    \end{itemize}

    In Ganeti 2.7 we can also disable the \texttt{confd} during build time
    using the \texttt{--disable-confd} flag, if it is not needed in our setup.
    The \texttt{confd} serves both network-oriented queries about the static
    configuration, and local UNIX socket queries about the current status of the
    system including live data configuration. To answer queries of the second
    category the daemon has to communicate with the node daemons through RPC
    calls. In next Ganeti versions, it is intended those two functionalities to
    be separated into two different daemons, for simplicity and security
    reasons.
\end{description}

Finally, we have to mention that there exists a log file per daemon model, which
are by default stored under \texttt{/var/log/ganeti} directory. Those log files
are:

\begin{itemize}
  \item The master-daemon.log, for the MasterD.
  \item The node-daemon.log, for the NodeD.
  \item The rapi-daemon.log, for the RapiD.
  \item The conf-daemon.log, for the ConfD.
\end{itemize}

\subsection{Ganeti Locking}\label{subsec:locks}

We have already covered the most major Ganeti parts. The last, but not least,
part we will cover is the Ganeti \emph{Locking} library and the way it is
implemented. Locking libraries are vital for every project, affecting the overall
performance. They must preserve data coherency, prevent deadlocks and thread/job
starvation. Ganeti Locking library has passed through many stages but still
improves and extends its features. In earlier Ganeti versions (\emph{1.x}),
there was a single global cluster lock for most operations, which made
inevitable the execution of parallel operations. In Ganeti \emph{v2.0} a
complete redesign of the locking library has been made, which allowed the
parallel execution of multiple operations. The locking library was also
drastically improved in version \emph{v2.1}, but the last major change was made
in \emph{v2.3} when the job priorities was firstly introduced. A feature called
\emph{Opportunistic Locking} was added lately, at \emph{v2.7}, which also
improved the parallel execution of some operations, mainly the instance
creations. Below we will present the current Ganeti Locking library and how
it is working \emph{``under the hood"}.

\begin{description}
  \item[The Locks] \hfill \\
    Locks are represented by objects of \texttt{locking.SharedLock} class. These
    locks are declared by the Logical Units located in the \texttt{cmdlib.py}
    module, and are acquired by the Processor which is found in the
    \texttt{mcpu.py} module, with the aid of the Ganeti Locking library, in
    \texttt{locking.py}. There are several locking levels which must acquired in
    specific order. These levels are the following:

    \begin{enumerate}
      \item Cluster level or BGL from Big Ganeti Lock.
      \item Instance level.
      \item Node allocation level or NAL.
      \item Nodegroup level.
      \item Node level.
      \item Node resource level.
      \item Network level.
    \end{enumerate}

    These locks must be acquired in an increasing order. Each lock has the
    following possible statuses:

    \begin{itemize}
      \item \emph{Unlocked}, anyone can grab the lock.
      \item \emph{Shared}, anyone can grab the lock but in shared mode only.
      \item \emph{Exclusive}, only one can hold the lock.
    \end{itemize}

    Besides the order in which the locks acquired, there are some extra rules
    which must be preserved:

    \begin{itemize}
      \item \emph{Cluster level}, resides the Big Ganeti Lock, or BGL. It is the
        first lock which must be acquired before performing any operation in the
        cluster. Can be acquired either in shared or exclusive mode, but
        acquiring it in exclusive mode is discouraged and should be avoided.
      \item \emph{Instance level}, resides the instance locks. They have the
        same name as the instances they protect, and are created when a new
        instance is added to the cluster. They are acquired as set, which means
        that if we need more than one instance locks we must acquire them at the
        same time. Internally the locking library acquire them in alphabetical
        order.
      \item \emph{Node level}, resides the node locks and have the same names as
        the nodes they protect. They are also
        acquired as a set, and internally acquired in alphabetical order. We
        should first acquire all the instance level locks that reside in a node,
        before we acquire the node lock itself. Ofcourse, before the node locks,
        we should already have the BGL acquired, preferably in shared mode.
      \item \emph{Node Resource level}, it is used for node resources protection,
        as it name reveals, and should be used by operations with possibly high
        impact on the node's disks.
      \item \emph{Node Allocation level}, this lock is similar to the BGL in the
        sense that it has its own level and there is only one. It must be acquired
        after the instance locks and before the nodegroup locks, and used for
        instance allocation related operations. As a rule-of-thumb, NAL must be
        acquired in the same mode as the node and/or the node-resource locks. It
        blocks instance allocations for the whole cluster and can be acquired
        either in shared or exclusive mode. OpCodes doing instance allocations should
        acquire it in exclusive mode. When an Opcode blocks all or a significant
        amount of the cluster's locks, it should be acquired in shared mode. The
        NAL lock should be released when the set of acquired locks for an opcode
        reduces to the working set, to allow allocations to proceed.
    \end{itemize}

    Besides the above levels, we also have the \texttt{ConfigWriter} lock which is
    shared among those functions that read the \texttt{config.data} file, and
    acquired exclusively by functions that modifying it. This extra lock level allows
    the \texttt{config.data} replication to the master candidate nodes using the
    \texttt{rpc.call\_upload\_file} call, without holding the node level locks since
    the RPC function caller already holds the config lock in exclusive mode. This
    have the advantage that the config distribution can run in parallel with other
    cluster operations.

    Similarly to the \texttt{ConfigWriter} lock, exists the Big \emph{Job Queue}
    lock. It is used from all classes involved in the queue handling. Job queue
    functions acquiring it can be safely called from the rest of the code,
    because the lock is released before leaving the job queue again, something
    that prevents deadlocks. Unlocked queue functions must only be called from
    those functions, which have already acquired the lock beforehand.

  \item[Ganeti Locking Library] \hfill \\
    As we have already mentioned, locks in Ganeti are represented by objects.
    The basic class which implements a lock in Ganeti is the \texttt{SharedLock}
    class located in the \texttt{locking.py} module. All locks needed in the
    same level must be acquired together. So, a class is needed to take care of
    acquiring the locks always in the same order, thus preventing deadlocks.
    This class is the \texttt{locking.LockSet} class, a container of one or more
    \texttt{SharedLock} instances, which provides an interface to add/remove
    locks, to acquire, and subsequently release any number of those locks
    contained in it, distinguished by name. As this class is beyond the scope of
    this document, we will not present it further. In this section we will
    focus in the \texttt{SharedLock} class, to understand the Ganeti's approach
    to its locking requirements.

    \texttt{SharedLock} class implements a shared lock. Multiple threads can
    acquire the lock by calling \texttt{acquire(shared=1)}. Exclusive acquirers
    should call \texttt{acquire(shared=0)}. Since Ganeti first introduced job
    priorities in \emph{v2.3}, the internal structure of \texttt{SharedLock}
    class also changed to support them. All pending acquires for a lock with
    different priorities is contained in a heap queue similar to the worker pool
    structure, named \texttt{\_\_pending}. The heap queue does automatic sorting,
    automatically taking care of priorities. For each priority there is a single
    plain list ([]) of pending acquires. This is a normal in-order list of
    conditions~\footnote{A condition variable in Ganeti is a bit different from
    the Python's built-in \texttt{threading.Condition} class. It uses POSIX pipes
    in addition to the operating system support on timeouts on file descriptors
    (see \texttt{select(2)}). All clients of the condition use \texttt{select} or
    \texttt{poll} to wait for notifications. In a higher level-of-view a condition
    variable has \texttt{acquire()} and \texttt{release()} methods that call the
    associate lock methods. Also has a \texttt{wait()}, \texttt{notify()} and
    \texttt{notifyAll()} methods. Threads waiting for a particular change of state
    call \texttt{wait()} repeatedly until they see the desired state. Threads that
    modify the state will call \texttt{notify()} or \texttt{notifyAll()} when they
    change the state in a desired way for the waiting threads.} to be notified
    when the lock can be acquired. Shared acquires are grouped together by
    priority and the condition for them is stored in a separate dictionary
    of shared acquires called \texttt{\_\_pending\_shared}. There is also a
    dictionary called \texttt{\_\_pending\_by\_prio} which keeps references for
    the per-priority queues indexed by priority for faster access.

    When the lock is released, the code locates the list of pending acquires
    with the highest priority waiting. Due to the heap queue behavior, this is
    the first element in the structure. The first, zero indexed condition of the
    list is notified. Once all waiting threads receive the
    notification, the condition is removed from the list, the code processes the
    second condition and so on. When the list of conditions is empty it is
    removed from the list, and the list of conditions of the second priority in
    the heap is processed. In
    Listing~\ref{lst:lock_queue}, we present a possible state of the internal
    queue from a high-level view. Conditions are shown as waiting threads.
    Assuming we have no timeouts or other modifications, for simplicity reasons,
    the lock will be acquired by the threads in the following order (concurrent
    acquirers in parenthesis):

    thread-Ex1, thread-Ex2, (thread-Sh1/thread-Sh2/thread-Sh3),
    (thread-Sh4/thread-Sh5), thread-Ex3, thread-Sh6, thread-Ex4, thread-Ex5

    \includeminted[text]{../listings/lock_queue.txt}{%
      Structure of the \texttt{SharedLock} class}{lst:lock_queue}{}

  \item[Locking Granularity] \hfill \\
    With the current locking policy, each Logical Unit acquires/releases the
    locks it needs; this means that locking is at the Logical Unit level.
    Ofcourse, each LU has its own locking requirements. Logical Units declare
    their locks and then execute their code with the appropriate locks held. In
    Listing~\ref{lst:lock_order}, we present how the Ganeti Processor with the
    aid of the Logical Units executes an OpCode from an abstracted point of
    view, which pays more attention to the lock handling.
  \item[Opportunistic Locking] \hfill \\
    The last major change in Ganeti locking library was made in \empty{v2.7},
    when firstly introduced the \emph{Opportunistic Locking} feature. The
    motivation behind this change was the need of more instance creations in a
    shorter amount of time. As of Ganeti \emph{v2.6}, instance creations acquire
    all locks when an iallocator algorithm was used, causing a lot of lock
    congestion on node locks when someone tried to create many instances at
    once. This situation can become worse when we are waiting for DRBD
    synchronization between disks, if we choose the \texttt{drbd} template for
    an instance. As a result, even on big clusters with multiple nodegroups all
    instance creations were serialized. The main objective was to speed up
    instance creations in combination with an iallocator even when the cluster's
    balance is sacrificed in the process. The cluster can be rebalanced latterly,
    by using external Ganeti tools ,e.g., \texttt{hbal}. So, the opportunistic
    locking reduces the number of node locks acquired for instance creations,
    causing many creation operations to run in parallel. More specific, instead
    of forcibly acquiring all node locks for creating an instance using an
    iallocator, only those locks available will be acquired, and the iallocator
    algorithm will run on those nodes we have succeeded to acquire their locks.
\end{description}

\includeminted[text]{../listings/lock_order.py}{%
  \text{OpCode} execution path}{lst:lock_order}{linenos}

\chapter{Ανίχνευση αντικειμένων χρησιμοποιώντας το νευρωνικό δίκτυο GoogLeNet}\label{ch:googlenet}

In Chapter~\ref{ch:ganeti}, we discussed about Ganeti's architecture; we covered
extensively its most basic components and all the prerequisites needed for
someone to get familiar with this tool. In this chapter, we will discuss how
Ganeti's current design impacts its performance and its scalability, and
then we will provide a design solution for limiting some of those issues and make
the tool even more suitable for cloud environments.

Specifically, section~\ref{sec:obj}, discusses in a few words the current
document's objective and how we are going to succeed it. Section
\ref{sec:back}, provides a detailed view of the configuration and job queue
storage, in addition to the main issues resulting from those design choices. The
options which lead us to the product choice are discussed in section
\ref{sec:cho}, while section \ref{sec:couch} covers in more details the tool
we chose to overcome those performance issues, and more specific the Apache
CouchDB database. Section~\ref{sec:des} finally, provides a detailed
presentation of our software design.

\section{Objective}\label{sec:obj}

Ganeti has been evolved since first introduced, and has become a mature software
tool for managing the low level VM of big clusters. In version 2.7, many new
features like the opportunistic locking where introduced, but many scalability
and performance issues are still arise from the current design.

In the current chapter, we will introduce a different approach for handling the
Ganeti's configuration and job queue storage, using a NoSQL database as the
backend storage layer, which will attempt to remedy some of the performance and
scalability issues that exist in Ganeti version 2.7.

\section{Background}\label{sec:back}

While Ganeti v2.7 is usable, it limits the flexibility and the performance of
the cluster. The current design for handling the configuration data and the job
queue storage, in addition to how it replicates those files among the master
candidate nodes, are some of the main reasons of those limitations. In the
current section we will analyze this design, and we will extensively discuss
about the main issues arisen from it.

More specifically, Section~\ref{sec:config} presents analytically the
configuration data form, in more details than the previous chapter, while
section~\ref{sec:queue}, concentrates on the job storage.
Section~\ref{sec:caveats}, points out the most important performance issues
that arise from the current design, both for the configuration file and the job
queue storage.

\subsection{Cluster configuration data}\label{sec:config}

In section~\ref{sec:architecture}, we saw that the Ganeti's cluster
configuration database is stored in a single file, the \texttt{config.data}
file, on the master node filesystem. In this section, we will dive into the
internal structure of the \texttt{config.data} file, which will lead us to the
conjecture that the current configuration management is imperfect and suffers
from scalability problems mainly on bigger clusters.

The configuration data uses JSON format, consisting of key/value pairs.
The keys that the configuration file consists of, are a combination of Ganeti
specific object collections, and default JSON objects of name/value pairs.
In detail, there are five Ganeti objects namely: \texttt{Cluster},
\texttt{Node}, \texttt{Instance}, \texttt{Nodegroup}, and
\texttt{Network}. From these objects the \texttt{cluster}, \texttt{nodes},
\texttt{instances}, \texttt{nodegroups}, and \texttt{networks} attributes are
composed of. The default name/value pairs are the \texttt{serial\_no},
\texttt{version}, \texttt{ctime}, and \texttt{mtime}. Ganeti configuration
objects provide the appropriate functions for serializing, de-serializing, and
handling them in a safe way, in order to by easily handled by external parties.
They also provide recursive checks for their derived classes, and are also
responsible for handling appropriately any attribute error that will arise.

The configuration file is represented internally by a \texttt{ConfigData}
object, which actually is the topmost-level configuration object.
The \texttt{cluster} attribute contains all the available information that
the cluster needs to operate normally like the master node name, the master ip,
the default hypervisor chosen during cluster initialization, the
candidate\_pool\_size, and more. The \texttt{nodes} attribute, contains all the
nodes that the cluster consists of. Each distinct \texttt{Node} object, contains
information about the corresponding node such as its name, its primary and
secondary IP, the node role information, and more. The \texttt{instances},
\texttt{nodegroups}, and \texttt{networks} attributes, contain relevant
information for the instances that the cluster contains, the nodegroups which
the user created, and the networks that exist in the cluster, as their names
denote.

Besides the Ganeti specific objects, \texttt{config.data} also contains
information about four general attributes. That are, the \texttt{serial\_no},
which is an increasing number denoting the number that the \texttt{config.data}
has been modified since it has been created, and it is used for
consistency checks in case when some candidates are stalled in the middle of a
configuration update, or in order to find the most recent answer when used by
the configuration daemon ,i.e., \texttt{confd}. The \texttt{version} attribute,
contains the current Ganeti version, and the \texttt{\{c/m\}time} timestamps
contain the exact time when the cluster was created and modified, respectively.

Listing~\ref{lst:config}, briefly presents the \texttt{config.data} internal
structure with its most basic key/value pairs. Due to its size, most of its
attributes and values have been intentionally removed for simplicity reasons.

\newpage
\includeminted[text]{../listings/config_data.json}{%
  Structure of the \texttt{config.data} file}{lst:config}{linenos}

\subsection{Job storage}\label{sec:queue}

Jobs are stored in the filesystem as separate, individual files using JSON
format just like the \texttt{config.data} file. The choice of storing each job
in its own separate file was made for a number of reasons. The most important of
them are summarized below.

In chapter~\ref{ch:ganeti}, we saw that a job can change its status many times
during its execution ,e.g., Queued, Waiting, Running, and more. Moreover, an
opcode, so as the job, passes from several execution phases like acquiring
the needed locks, running the iallocator algorithm, and so on. The user should
have access to all that information when requests it, either for debugging
reasons, or simply for monitoring the execution path of a job. Those information
is saved in the job object as a list of log messages in the \texttt{log}
attribute. It is obvious that a job is modified many times during its execution.
The choice of using a file per job, is based on the fact
that a job changes quite often, and a file that can easily and atomically be
(over)written facilitates this behavior. Furthermore, a file can be easily
replicated to the master candidate nodes. The replication is done atomically for
every single file with a multi-node \texttt{Remote Procedure Call (RPC)} call
with a timeout of about a minute. In addition, a consistency check in the job
queue across master candidate nodes through a third partie, can very easily be
implemented, since all job files should be identical.

It is also interesting to see how a job is internally represented by Ganeti.
Jobs are stored in the filesystem, but until the job completes its execution
there is also an in-memory representation of it. Any modification to the job
object is flushed on the disk, and then replicated to the candidate nodes. That
in-memory representation of a job is a python class definition called
\texttt{\_QueuedJob}. This class contains attributes and implements all the
appropriate methods needed for the smooth execution of a job. The most important
attributes that a \texttt{\_QueuedJob} object contains are: the job \texttt{id}
field, the \texttt{queue} object on which the job belongs, the \texttt{ops}
field which is a list of the job's opcodes where the \texttt{\_QueuedOpCodes}
objects are encapsulated, and three timestamps the \texttt{received},
\texttt{start}, and \texttt{end}, providing information about the time when a
job was received, started and finished its execution respectively.

\subsection{Caveats}\label{sec:caveats}

We extensively discussed about the design of Ganeti. While it scales quite well
for small clusters, it does not as the number of nodes grow, and some drawbacks
start to appear. This is mainly due to the fact that several documents are
shared among the nodes of the cluster. The \texttt{config.data} file and the job
queue need to be present, and replicated, to all master candidates for
reliability reasons mainly, and the \texttt{ssconf\_} static configuration
files have to be replicated to all the nodes of the cluster, as well. In
addition, the \texttt{confd} which is present on all the master candidate nodes,
might need to have access to the configuration data file even if the master node
is down. Due to those needs, an increase in the size of the cluster will bring
up the configuration management imperfections that Ganeti suffers of. In
particular:


\begin{itemize}
  \item Some operations like \texttt{instance-(add/remove/rename)},
        \texttt{network-(add/remove)}, or
        \texttt{node-(add/remove/offline/drained)} are the most commonly used
        operations in a Ganeti cluster. Some of these operations like the
        instance related ones, grow in frequency as the cluster grows. We would
        ideally want the  time needed in order to those operations to complete,
        be steady. This does not apply at all, due to the fact that all these
        operations need to contact all the nodes of the cluster in order to
        update the \texttt{ssconf\_} files, so they become significantly slower.
  \item Any other operations that do not affect the \texttt{ssconf\_} files
        will contact, at least, all the master candidate nodes for two reasons.
        The first one is to inform the configuration data files for the new
        updates been made, and the second one is to synchronize the job queues
        among them. As the number of nodes in a cluster grows, we expect the
        number of jobs to grow as well. While the number of candidate nodes is
        constant, an increase on the jobs will have impact on the cluster
        performance. This is due to the fact of an overloaded master daemon
        which will affect the whole cluster performance, because besides the
        growth in the number of jobs it has to deal with, it also has to
        supervise the replication process of those files among the candidate
        nodes, manage the locking, an so on.
  \item The candidate pool size is not affected as the cluster size
        increases. It is a constant number independent of the nodes of the
        cluster. Ganeti though, interacts with the candidate nodes every time
        it updates one of its configuration and job queue files in order to
        maintain the file consistency among them. The replication procedure is
        the main reason that prevents Ganeti from scaling. It is handled and
        monitored exclusively by the master daemon, and is a procedure that
        breaks the scalability because many should-be-fast operations are slowed
        down by replicating the changes to remote nodes, thus waiting with locks
        held on remote RPC calls to complete.
  \item Another issue arisen from the increase in the number of jobs, is the
        \emph{config lock}. Any job that modifies the cluster state must
        exclusively acquire that lock before apply its changes to the cluster,
        and so as to the configuration file. The single config lock becomes a
        bottleneck, when a huge number of jobs is in execution and try to
        acquire it. In addition, the lock will not be released until the
        modification have been successfully replicated to the master candidate
        nodes, something that increases the congestion on that lock.
  \item The configuration file is a JSON formatted file which needs to be
        serialized before it is flushed to disk. The time needed to serialize
        it, is quite small in smaller clusters and it can be ignored. Ganeti
        though, is a tool used by big open source projects, such as Synnefo
        \flink{http://www.synnefo.org}, which means it is mainly used in bigger
        clusters. Providing an example, a cluster with around \emph{1.000}
        instances, will have a \texttt{config.data} file of about \emph{2.5 MB}
        in size. The serialization cost starts to be a bottleneck when the
        configuration file enlarges, even from sizes at around \emph{1 MB}. This
        claim will be justified in Chapter~\ref{ch:performance}. The
        de-serialization cost from disk is also raised respectively, but it does
        not affect the overall cluster performance at all, because the
        \texttt{config.data} de-serializes only at the master-daemon startup.
        Then it exists at the master node memory, and its disk version is
        updated from there.
  \item Besides the serialization cost, more factors are affected when the
        configuration data size increases. These are the time needed to
        flush the changes to disk, in addition to the time needed to distribute
        those modifications to the master candidates. These two extra costs, in
        addition to the serialization time discussed in the bullet above, slows
        down the cluster operations, reduce the overall performance of the
        cluster and make even the quick commands last several minutes,
        instead of some seconds in order to complete.
  \item The current configuration management groups all the cluster's attributes
        in a single JSON file ,e.g., instances, nodes, networks, and more. This
        design choice, in combination to the global config lock, forbids any
        type of concurrent update to distinct Ganeti objects, and makes every
        modification access to the configuration file be serialized. There are
        various cases when that restriction reduces the cluster performance. An
        example is when a client wants to add an instance to the cluster, while
        another one tries to create a new network. While the clients modify two
        distinct objects of the cluster with an indirect
        relationship, they can not make the updates concurrently, but they have
        to wait instead. It would be very convenient for Ganeti and its users to
        allow concurrent updates when they do not affect other cluster
        operations, because it would have a positive impact in the cluster's
        throughput.
  \item Moreover, in case of a faulty update on the configuration file, there is
        no way to roll back the changes made in it and return to a previous
        state. Even if we keep backups of the configuration file and want to
        revert a modification, we should alter the whole configuration file and
        not only the section that we modified, increasing the probability of
        causing a breakage in the cluster configuration state. This attribute
        could be very useful in case when the \texttt{config.data} breaks due to
        a faulty-update on it, and we want to recover it painless.
\end{itemize}

\section{Choice of product}\label{sec:cho}

Our design solution aims to address some, and not all, of the above issues.
It replaces the Ganeti configuration and job queue storage with a NoSQL
distributed, document-oriented database. There are plenty of NoSQL choices in
the database market which would meet Ganeti's requirements, such as MongoDB
\flink{http://www.mongodb.org/}, or Riak~\flink{http://basho.com/riak/}. Some of
the criteria for deciding which one was the best fitted NoSQL solution for
Ganeti were the replication style, the reliability provided, the available API,
and how the database handles reads and writes. For our implementation we chose
\emph{Apache CouchDB}~\flink{http://couchdb.apache.org/} as our backend
database.

Apache CouchDB (acronym of \emph{``Cluster Of Unreliable Commodity Hardware"}),
is an open source database built on the Erlang OTP platform~\flink{http://www.
erlang.org}, a functional, concurrent programming language, and a development
platform too. Erlang was developed for real-time telecom applications with an
extreme emphasis on reliability and availability using lightweight ``processes"
and message passing for concurrency. It is an
ideal solution for a database server due to its robustness and concurrent
nature. CouchDB is a database that focuses on being \emph{``a database that
completely embraces the web"}, as its developers promote. It is a NoSQL
document-oriented database, using JSON format for storing its data, a simple
RESTful API based on HTTP, and a powerful query server using Map/Reduce
techniques that is written in JavaScript, by default, but there are also servers
available for nearly any language we can
imagine~\flink{http://en.wikipedia.org/wiki/CouchDB}.
It was created in April 2005 by Damien Katz, and since February 2008 is part of
the Apache foundation. In July 2013, the CouchDB community merged the codebase
for BigCouch, Cloudant's clustered version of CouchDB, into the Apache project.
The BigCouch clustering framework is prepared to be included in an upcoming
release of Apache CouchDB~\cite{bigcouch}.

The main reasons we chose CouchDB for our modifications to Ganeti,
are presented in the current section.  We do not claim it is the best fitted
database solution for Ganeti or the only one, but without lost of generality it
is a representative NoSQL database which will show us how Ganeti reacts with a
different underlying storage solution. The criteria which were evaluated are the
following:

\begin{itemize}
  \item \underline{\emph{Free Software,}} CouchDB is an open source project
    which keeps up with the current open source Ganeti policy.
  \item \underline{\emph{Licensing,}} CouchDB is licensed under the Apache
    License, Version 2.0~\flink{http://www.apache.org/licenses/LICENSE-2.0}
    which does not affect Ganeti's license or model. Ganeti will remain a
    separate software, which connects through Apache licensed libraries to
    CouchDB.
  \item \underline{\emph{Bindings,}} Ganeti is written in both python and
    Haskell languages. CouchDB provides Haskell bindings which are available on
    Hackage, the Haskell package library, and a variety of Python libraries for
    working with CouchDB.
  \item \underline{\emph{Document Storage,}} In order to avoid big changes in
    the current configuration and job queue storage we want a solution that
    would fit this design. CouchDB stores its data as documents; one or more
    key/value pairs using JSON format, that fits the Ganeti's needs.
  \item \underline{\emph{Replication model,}} CouchDB is a peer-based distributed
    database system. Its replication process synchronizes two copies of the same
    database. If you change one document of a database, the replication process
    will propagate these changes to the second database.
    This is very similar to the master-slave
    replication procedure used by Ganeti. In addition, the BigCouch merge in the
    CouchDB project will give us a native clustering support which could later
    provide different design solutions for the Ganeti data handling.
  \item \underline{\emph{Simplicity,}} CouchDB comes with good documentation in
    the form of books, presentations, blog posts, wikis, and a strong and active
    community which makes it much simple to be installed and configured. In
    addition, the RESTful HTTP API which uses, is quite straightforward and
    simple, and it does not requires much effort to learn using it.
  \item \underline{\emph{Security model,}} CouchDB comes with a simple reader
    access and update validation model to protect who can read and update
    documents, that can also be extended to support custom security models. In
    addition, every CouchDB database instance can have one or more administrator
    accounts. These accounts come with specific privileges and user credentials
    in order to secure the access to selected databases and documents.
    Validation functions are written in JavaScript, and can be used as documents
    are written to disk. If the documents pass the validation criteria, the
    update is allowed to continue, otherwise the update is aborted and an error
    message is returned to the client.
  \item \underline{\emph{Resource Usage,}} CouchDB is designed from the ground
    up to service highly-concurrent use cases. There is not fixed RAM, CPU or
    disk space needs for CouchDB. It is flexible enough to run from a smart
    phone to a cluster. Apparently, more RAM is better, because CouchDB works
    completely through file I/O, delegating caching to the operating system,
    the filesystem cache. CouchDB makes the assumption that disk space is
    cheap, so it does not take great care of it. The good news are that some
    operations like database \emph{compaction} reclaim a lot of disk space.
    CouchDB is  written in Erlang, so the more CPU in a server, the most
    \emph{beam} processes~\flink{http://www.erlang.org/documentation/doc-5.8.3/
    doc/efficiency\_guide/processes.html} can be created. CouchDB (or Erlang)
    take great advantage of this resource. Summing up, memory and disk have the
    great ``pain", as a result faster disks and more memory will be handy and
    will increase the database overall performance.
  \item \underline{\emph{Debugging,}} CouchDB maintains a log file where every
    single operation, or event made to the CouchDB server is
    recorded. The debugging level of the log file can be modified,
    and the user can define how verbose and detailed the logging will be. We can
    choose from a very informative log file, where every HTTP header, external
    process communications, authorization operations, and much more information
    is recorded, to a status where any debug message is disabled. Ganeti also
    maintains a set of log files to record its updates, and CouchDB will just
    insert one more log file to the already existent.
  \item \underline{\emph{Reliability,}} CouchDB comes with a fault-tolerant
    storage engine that puts the safety of the data first. As it name denotes,
    is build for Clustering On Unreliable Commodity Hardware and the main goal
    is to provide data integrity, high-scalability and reliability in a
    fault-prone environment. The fact that CouchDB is written in Erlang, a
    concurrent, functional programming language with an emphasis on
    fault-tolerance reinforces the succeed of the data safety goal. Its internal
    structure is fault-tolerant, and failures occur in a controlled environment
    and are dealt without letting single problems cascade to the whole system,
    but are isolated in single requests. For CouchDB specifically, if an
    operation fails, we will never end up in a state with partially updated
    objects, or corrupted objects that was previously written successfully to
    the server. It provides a total reliable storage engine, which we will
    extensively present in section~\ref{sec:couch}.
  \item \underline{\emph{Recovery from failures,}} The current CouchDB design
    does not provide automate recovery from failures. The recovery procedure
    will be handled by Ganeti with the administrator help, as it currently is.
    Various commands like the \texttt{master-failover}, or the
    \texttt{redist-conf} will be extended to meet up with the new needs. The
    CouchDB replication procedure, along with the CouchDB tools, will make
    those operations relatively painless.
  \item \underline{\emph{Backups,}} CouchDB stores each database in a separate
    file in the disk, as we are extensively cover in later sections. We can take
    backups of a database file, silently and without stopping the database by
    simply running a \texttt{cp} Unix command ,i.e., \texttt{cp db.couch
    /mnt/backup}. We can store periodically flat-file copies of the database
    files on the master and master candidate nodes, and create a method to
    ``re-init" the Ganeti status from those flat-files during disaster-recovery.
  \item \underline{\emph{Backwards compatibility,}} We would like to have a way
    to easily converting from the current Ganeti storage management, to CouchDB.
    We decided to support both CouchDB and file configuration as different
    storage engines, with different limitations for each case. Maybe it is not
    the simplest solution since it requires to convert a fairly great amount of
    code, but the future ability of further expanding the underlying storage
    options is a tradeoff that reinforces that approach.
\end{itemize}

\section{Apache CouchDB}\label{sec:couch}

In the previous section, we mentioned the factors why we chose CouchDB for our
implementation, depending on current Ganeti needs, keeping also in mind that we
do not want a choice that will introduce important Ganeti design changes. In the
current section we will discuss extensively about Apache CouchDB and its main
characteristics and features, in order to proceed with the detailed design
section.

CouchDB is a document-oriented, distributed, schema-free NoSQL database, using
views for aggregating and reporting on documents in a database. For a complete
overview of CouchDB's technical information there is a well structured
documentation~\flink{http://kxepal.iriscouch.com/docs/dev/contents.html}. Let's
review some of the basic elements of CouchDB:

\begin{description}
  \item[Schema-Free] \hfill \\
    Unlike SQL databases which are designed to support highly-structured data,
    with CouchDB no schema is required. New document types can be added at
    will, alongside with the old ones. CouchDB is designed to perform on
    document-oriented applications with large amounts of semi-structured data.
  \item[Document-Oriented] \hfill \\
    Documents are the primary unit of data in CouchDB. Each document is a JSON
    formatted object and consists of any number of named fields and attachments
    \footnote{Documents in CouchDB can have attachments just like an email. For
    creating an attachment, we need to provide a file name, the MIME type and
    the base64 encoded binary data. Its even possible to have multiple
    attachments for a single document.}.
    Field values can be strings, numbers, dates, or even ordered lists and
    associative arrays. Documents also include metadata that are maintained and
    used by the CouchDB server. An example of a document in CouchDB would be a
    contact document as shown in Listing~\ref{lst:couch_doc}. In that document,
    ``Type" is a field that contains a single string value ``Contact", ``Email"
    is a field containing a list of two values and so on.

    A CouchDB database is a flat collection of documents, each having a unique
    identifier named \texttt{\_id}, and a revision/version number named
    \texttt{\_rev}. The version number is a special field in CouchDB with great
    importance, about which we will talk later in this chapter. The underlying
    data structure used to store the database files is a B+tree structure.
    CouchDB implementation is a bit different from the original B+trees; while
    it maintains all the important properties, it adds an append-only design
    \flink{http://guide.couchdb.org/editions/1/en/btree.html}, along with a
    \emph{Multi-Version Concurrency Control}. That append-only variation of the
    original B+tree structure, trades a bit of (disk) space for speed. A B+tree
    is an excellent data structure for storing huge amount of data for fast
    retrieval. The B+trees are very shallow but wide data structures. The leaf
    nodes contain the actual data in an ordered manner, while the intermediate
    nodes contain indexes/pointers to the nodes beneath them. While
    other tree structures can grow very high, a typical B+tree has a
    single-digit height, even on millions of entries. \emph{Jan Lenhard}, one of
    the core CouchDB developers, said during the \emph{Berlin's Buzzwords 2013}
    conference~\footnote{Berlin Buzzwords is a Germany's conference that
    focuses on the issues of scalable search, data-analysis in the cloud and
    NoSQL-databases} that a B+tree node in CouchDB has a size of about
    \emph{60.000} entries. That actually means that even on billions of entries
    in a database, we will have a tree depth of 6, at most. This is very
    interesting for CouchDB particularly, where the leaves nodes of the B+tree
    are stored in a slow medium such as a hard drive. CouchDB does not make use
    of a built-in cache layer, but it uses the operating system's cache instead.
    Due to the small height of the structure, the filesystem cache keeps the
    upper nodes of the tree cached, so reading, or writing to a document
    requiring only a few seeks to disk on the final tree node, making it a quite
    fast data structure for both read and write requests.

    \includeminted[text]{../listings/couch_doc.json}{%
      Document sample in CouchDB}{lst:couch_doc}{linenos}

  \item[Views] \hfill \\
    CouchDB is a schema-less database. However, for some
    applications a kind of structured data may be required. In order to deal
    with that problem, CouchDB integrates a view model using JavaScript for the
    view description. Views are also a useful tool for many other purposes like
    document filtering based on specific fields, extracting and presenting data
    in a special order, building indexes among documents to find a value that
    resides in them, and generally performing all sorts of calculations on the
    databases data.

    CouchDB views are stored inside special \emph{design documents}, and a design
    document can contain any number of views. A view is actually a map function
    of a map/reduce system. All map functions have a single parameter
    \texttt{doc}, which corresponds to a single document in the database. A simple
    view example which checks whether the database's documents have a date and a
    title field is shown in Listing~\ref{lst:view}. When we query the view,
    CouchDB takes the source code and runs it on every document in the database
    our view was defined. We query the view to retrieve the view result. Because
    the view runs on all documents of a database, it would take a lot amount of
    time to run it, if it should traverse the whole database every time we
    query it. Instead, a view runs on all documents only the first time it is
    queried; if a document is updated the map function will only run to
    recompute the new keys and values for the updated document. Views in CouchDB
    are stored in separate flat files just like the databases, using B+tree data
    structure as well. Initially, the view file is empty because no index has
    been built yet. Views are being built \emph{lazily} when the first query is
    made. The next view query will incrementally update the not updated view
    indexes.

    \includeminted[text]{../listings/view.txt}{%
      View function in Javascript in CouchDB}{lst:view}{linenos}

  \item[ACID properties] \hfill \\
    CouchDB is a database, so every transaction should ensure the ACID
    (Atomicity,  Consistency, Isolation, Durability) properties. CouchDB
    provides ACID semantics, and in this part we will examine carefully each of
    those properties.

    \begin{itemize}
      \item \emph{Atomicity}, refers to the ability of database to guarantee
      that either all the tasks of a transaction are performed or none of them
      are. Each transaction is said to be atomic in case when one part of a
      transaction fails, the whole transaction fails. CouchDB database
      modifications follow the ``all or nothing" rule, ensuring that property.
      \item \emph{Consistency}, is the property that ensures that any
      transaction will bring the database from one valid state to another,
      according to some defined rules. The valid state does not necessarily
      guarantee correctness of the transaction in all ways the application
      programmer might have wanted, but that the database will remain consistent
      even if the transaction succeeds, or fails. For distributed systems, as
      CouchDB is, the system is either strongly consistent or has some form of
      weak consistency, also referred as eventual consistency. CouchDB is an
      eventual consistent database, ensuring that the database will eventually
      reach at a consistent state. The \emph{MVCC} method ensures that each
      client sees a consistent snapshot of the database from the beginning to
      the end of the read operation. The latest version is sitting somewhere in
      the cluster. Older versions are still out there and eventually all nodes
      will see the latest version.
      \item \emph{Isolation}, refers to the requirement that other operations
      cannot access or see the data in an intermediate state during a
      transaction. This constraint is required to maintain the performance as
      well as the consistency between transactions in a database. Thus, each
      transaction is unaware of another transactions executed concurrently in
      the system. It ensures that the concurrent execution of transactions
      results in a system state that would be obtained if transactions were
      executed serially. Documents updates in CouchDB (create, delete, modify)
      are always serialized on disk. In addition, the concurrent update of the
      same document will result in two new documents, where none of the clients
      that modify the same document is aware of the other client existence.
      \item \emph{Durability}, refers to the guarantee that once the user has
      been notified of success, the transaction will persist, and will
      not be undone.
      This means it will survive system failure, and that the database system
      has checked the integrity constraints and will not need to abort the
      transaction. CouchDB uses by default the \emph{fsync} system call, to
      ensure that a transaction have reached the disk before declaring it as
      successful. It also gives the ability to the user to loose that property
      and increase the write performance by using intermediate buffers, but by
      default it always ``fsyncs" for every transaction. In addition, another
      transaction will never overwrite any changes made by a previously
      successful transaction, due to the append-only model followed. So, it
      cannot corrupt anything that has been written and committed to disk
      already.
    \end{itemize}

    In order to understand even better how CouchDB ensures the above properties,
    we are going to give an overview of some more technical, but important,
    CouchDB features:

    CouchDB implements a form of \emph{Multi-Version Concurrency Control (MVCC)}
    \flink{http://en.wikipedia.org/wiki/Multiversion\_concurrency\_control},
    instead of locks. Requests are run in parallel, making excellent use of the
    CPU, but writes are always serialized on disk. Database readers will never
    have to wait for writers or other readers on the same document. Each reader
    sees a consistent snapshot of the database from the beginning till the end
    of the operation. We mentioned earlier that every document always contains
    besides the \texttt{\_id} field, a \texttt{\_rev} field as well. This is the
    document's version, because in CouchDB documents are versioned. If we want to
    modify a document, we create an entire new version of it and save
    it after the old one. So, we end up with more than one versions of the same
    document, depending on how many times we modify it. The version is a
    composition of two values ,i.e., \texttt{``23-a2a33fdabad1376f58a12ea0ff4b"}.
    The first one is an increasing sequential number, which denotes how many
    times the file was modified. The second part of the number after the dash,
    is a hash composition of the document's contents plus the sequential number
    of the revision. If we found two documents, in different databases, with
    same \texttt{\_id} and \texttt{\_rev} values its not necessary to look at the
    contents of each file to know that are identical, we already know they are.
    In that lockless update model we may end up with two clients updating the
    same document. In that case, a  conflict error will be produced, where two
    versions of the same document will exist. There is a conflict mechanism used
    by CouchDB to resolve those errors, and the user can also involve in the
    conflict resolution. There are three possible states after a conflict
    detection, which are explained in the \texttt{Eventual Consistency} section
    \ref{item:eventual}, and ensure that the database remains consistent even if
    sometimes the application should involve in that procedure.

    \emph{Committing} is the process of updating the database file to reflect
    the changes requested. It is a CouchDB process with great importance. It is
    not needed in order to use CouchDB, but it will help us to deeper understand
    the CouchDB design, and how the ACID properties are preserved. CouchDB uses a
    B+tree data structure for both the ``storage" and the ``view" needs. In that
    paragraph we will focus on the storage part. Every database file consists
    internally of a number of components. The most important of them follow:

    \begin{itemize}
      \item \emph{B+tree, by\_id\_index}. It is a B+tree structure that uses the
      document ID as the index key. This index stores the mapping from document
      IDs to their positions on disk. Is is mainly used to lookup documents by
      their ID. The data on disk it points to, contain the list of revisions,
      along with the document's revision history.
      \item \emph{B+tree, by\_seqnum\_index}. It is a B+tree structure that uses
      the sequence number as the index key. A new sequence number is a
      monotonically increasing number, and is generated every time a document is
      created, deleted, or modified. This index stores the mapping from the
      update sequence number to the document's position on disk. This B+tree
      actually answers the question \emph{``what happened since?"}, and is very
      useful to keep track of the last point of the replication synchronization
      or the last point of view index updates, by the \emph{compaction}
      operation, and more.
      \item \emph{Header}. The file header contains the pointer to the roots of
      the above two B+tree structures, some metadata needed by the CouchDB
      server like the database name, or size, and a checksum to ensure the
      file's integrity.
    \end{itemize}

    CouchDB database files are purely append-only. This means that all document
    updates like create, delete, and modify happen in a purely append-only
    mechanism. All writes occur at the end of the file, and old versions of
    documents are never overwritten, or deleted when new versions come in. Either
    inserting or deleting a document, the database file still grows only at the
    end. More specifically, CouchDB uses a kind of copy-on-modified approach.
    This means that the update is not happening in-place, but after we located
    the B+tree node that contains the document to be modified, we copy it over,
    make the appropriate modifications, and append it at the end of the file.
    After that, the parent B+tree nodes should also be informed to point to the
    new location. A modification to the parent node is triggered, which will
    also cause a new copy of the parent node, and so on all the way back to the
    root node of the B+tree. Finally, the file header must be modified to point
    to the new root node location. That means that every update will trigger 1
    write to the document and \emph{logN} writes to the B+tree nodes, where N is
    the B+tree height. A graphical representation of this procedure from a
    high-level of view is shown in Figure~\ref{fig:btree}, which was taken from
    a blog post by Ricky Ho titled, ``NoSQL patterns"
    \flink{http://horicky.blogspot.gr/2009/11/nosql-patterns.html}. Notice that
    the update of a data slot causes the creation of the \emph{green} nodes,
    while the \emph{yellow} ones will be removed as soon as no one uses them,
    after a compaction operation.

    \begin{figure}[htbp]
      \begin{center}
        \includegraphics[width=0.9\maxwidth]{../figures/b-tree.pdf}
        \caption{The power of B+trees in CouchDB\label{fig:btree}}
       \end{center}
    \end{figure}

    The update mechanism also forbids partial updates; either the update will
    succeed or will fail completely, there is nothing in between. The file
    footer is the last region that is updated and appended at the end of the file
    during a transaction. It is the last 4K of the database file, and is actually
    the database header we explained above. To declare an operation successful
    the footer must be written twice. It is separated in two identical chunks
    of 2K each. The committing process occurs in two phases. During the first
    phase, CouchDB appends any changes in terms of document data and their
    associated indexes to the file. After recorded the new file's length, it
    records it to the first 2K of the file footer. Those updates are
    synchronously flushed to disk. In the second phase it just copies the first
    2K of the footer, over the second 2K of the footer, and flushes again. Only
    when both footers are flushed to disk CouchDB declares the transaction as
    successful. If a failure happens during phase 1, the partially flushed
    updates are simply forgotten. Incomplete writes and index updates simply
    appear as garbage data at the end of the file, and are ignored on database
    open. If a failure happens during the header committing (phase 2), CouchDB
    start reading the file backwards~\footnote{To differentiate data from
    headers, CouchDB appends a single byte every 4K of the database file. If the
    byte value is 0x01, everything preceding is database header. If the value is
    0x00 then document data precedes.}. If the first 2K is corrupted, CouchDB
    replaces it with the second identical 2K footer. The same happens if the
    second 2K are corrupted. If the header is intact then the data are intact as
    well. Checks never happen in the document data or their associated indexes,
    but only in the database header. With this design data are never lost, and
    data on disk are never corrupted.

    This append-only design results in very fast updates, and query processing.
    Provided that the B+tree nodes are in system memory, they require the
    minimal seeks possible. A tradeoff of this scheme is the disk space needs.
    CouchDB is built with the assumption of cheap disk space, but since every
    document update causes a whole new copy of the B+tree indexes, care should
    be taken for limiting the disk space needs. CouchDB answer is the
    \emph{compaction} operation. Compaction compresses the database file by
    removing unused sections created during updates. Old revisions of documents
    are also removed from the database, though a small amount of meta data is
    kept for server related needs. It is manually triggered per
    database and retrieves a great amount of disk space. Technically, the
    compaction process opens the file and reads the \emph{by\_seqnum\_index}.
    It traces the B+tree all the way to the leaf node and
    copy the corresponding document content to the new file. The database
    remains completely online the entire time and all updates and reads are
    allowed to complete successfully. The old file is deleted only when all the
    data has been copied and all users transitioned to the new file.
  \newpage
  \item[Distributed Updates and Replication]\label{item:replication} \hfill \\
    Maybe the most powerful CouchDB's feature is the simplicity of replicating
    databases among different servers in the web. Replication is an incremental,
    fast, and one way process involving a \emph{Source} and a \emph{Target}
    database. The aim is very simple; synchronize the independent replica copies
    of the same database, between the source and the target nodes. All active
    documents should co-exist after the replication finished, and the deleted
    documents should also be deleted in both databases. During replication, only
    the last version of each document is copied from the source to target
    database, along with the document's revision history.
    Previous revisions are only accessible via the source database. Not
    all documents replicated over and over again, replication process continues
    from the last replicated documents. CouchDB replication protocol is not
    something magical, but a simple agreement of its public HTTP API in a
    specific way. The replicator is actually a separate, independent Erlang
    application with its own processes, where processes are CouchDB client
    workers with some logic on synchronizing documents between two databases.
    The CouchDB replication framework comes with many features and can be
    modified depending on the distributed model we want to follow. We can chose
    between \emph{Master-Master} replication, the common \emph{Master-Slave}
    replication, and also \emph{Filtered} replication which managed by
    Javascript functions so that only particular documents fulfilling specific
    criteria will be replicated.

    Before we extensively present the CouchDB Replication Protocol, we should
    explain some terms that we will confront at a later point. It is important
    to know that CouchDB saves every replication in a separate special database
    called \texttt{\_replicator}. A replicator object is a normal JSON document
    just like all the other CouchDB documents. In Listing~\ref{lst:repl_doc},
    we see a replication document where the replication is set to
    \texttt{continuous} which means that it will replicate as soon as new
    documents appear in the source database. The \texttt{create\_target}
    attribute means that if the database does not exists in the target node
    it will be created, the \texttt{\_id} is the identifier provided by user to
    the replication which is different form the \texttt{replication\_id} which
    internally assigned by the CouchDB server. The \texttt{\_replication\_state}
    shows the current replication status, while the
    \texttt{\_replication\_state\_time} is a Unix timestamp which denotes the
    time when replication was set. Finally the \texttt{source} and \texttt{target}
    are the databases which involve in the replication procedure. Now we are
    ready to proceed with the replication algorithm.

    \includeminted[text]{../listings/repl_doc.json}{%
      Replication document in CouchDB}{lst:repl_doc}{linenos}

    \textbf{CouchDB Replication Protocol}

    The CouchDB protocol is a synchronization protocol between two peers over
    HTTP using the RESTful CouchDB API. Anyone familiar with Git
    \flink{http://gitscm.com/}, a well-known distributed source control system,
    actually knows how replication works. Replicating is very similar to the
    \emph{push} and \emph{pull} methods used in distributed source managers like
    Git. Once a replication task is posted or a replication document is created
    in the \texttt{\_replicator} database, the \emph{couch\_replicator} process
    is started. This process watches the whole replication procedure which we
    are extensively explain below:

    \begin{enumerate}
      \item Firstly, \emph{Replicator} (the process responsible for the
      replication) verifies that both the Source and Target peers/databases
      exist. This is done via a \texttt{HEAD /{db}} request in both databases.
      If the Target does not exist and the \texttt{create\_target} field is set
      to True, an additional \texttt{PUT /<target>} request will be produced.
      \item \emph{Replicator}, retrieves basic information from both Source and
      Target, in order to get some important for the replication fields like the
      \texttt{update\_seq}. Each update of a document during his life generates
      a serial sequence number and that Update Sequence gives us powerful
      information about the modifications made in a specific update, or a range
      of updates. This is done via a \texttt{GET /<db>} request.
      \item Next, a unique identifier must be generated by the Source and
      assigned to the replication. That replication ID is very useful in order
      to find and resume previously interrupted replications, and identify each
      separate replication process.
      \item The replication ID generated in the previous step, is saved in a
      special non-replicating document interface, the \texttt{\_local} document
      on both peers via a \texttt{PUT /<db>/\_local/<unique\_id>}. This document
      will contain the last sequence ID, \texttt{last\_seq} field, from the
      previously run replication. The \texttt{last\_seq} is also mentioned as a
      \emph{Checkpoint} for the replication process.
      \item Using the replication ID, \emph{Replicator} will retrieve the
      replication log history from both Source and Target via a \texttt{GET
      /<db>/\_local/<unique\_id>}. Then it will compare the two logs in order to
      find common ancestry. If there is not common ancestry, or if there are not
      any replication logs, it means that the replication is triggered for
      first time and an error will occur during that step. This will not affect
      replication because it is only an optimization to speedup the replication
      by not re-replicating already copied files.
      \item \emph{Replicator} then listens to the \texttt{\_changes} feed from the
      Source's database. The \texttt{\_changes} feed is actually a list of
      changes made to the database's documents. The \emph{Checkpoint} can be
      used as input to the \texttt{since} option of that call, in order to
      retrieve changes immediately after the sequence number given.
      \texttt{feed, style, heartbeat,} and \texttt{filter} are some more
      parameters that can be used while listening to
      the changes feed. Only the list of current revisions changed will be
      returned, and not the whole revision tree of the changed documents. An
      example request for the \texttt{\_changes} feed is:\\\texttt{GET
      /<source>/\_changes?feed=normal\&style=all\_docs\&since=last\_seq}
      \item Collect a group of documents/revision ID pairs from the changes feed
      and send them to the Target via a \texttt{POST /<target>/\_revs\_diff}
      request. The Target's response will contain only those pairs that are
      missing from its database.
      \item Fetch all the documents contained in the response from the previous
      step, from the Source database. A \texttt{GET /<source>/<doc\_id>?revs=
      true\&rev=<revision>} request will be made for each document. There are
      also some optimization options which can be used like \texttt{?atts\_since},
      but will not be further presented. Every document fetched is put in a
      local stack for bulk uploading in order to utilize the network's bandwidth
      effectively. When all documents are fetched they will be uploaded at once
      to the Target database using \texttt{POST /<target>/\_bulk\_docs} request.
      \item After the batch of changes uploaded to the Target, the Target must
      ensure that every single bit is on persistent storage (on disk). The
      request is: \texttt{POST /<target>/\_ensure\_full\_commit}.
      \item Then, Source and Target update their replication logs and the
      \emph{Checkpoint} value, so next replication will resume from that point. The
      request is as in a previous step: \texttt{POST /<db>/\_local/<unique\_id>}
    \end{enumerate}
    If the replication feed is set to continuous, the \emph{Replicator} will
    listen to the \texttt{\_changes} feed until someone cancels it. When new
    changes encountered, the replication process will repeat from
    \underline{step 7}. \emph{Replicator} does not
    have to run either on Source or Target servers. It could run from anywhere
    with read access to the Source's database and write access to the Target's
    database. However, it is nearly always run by either the Source or the
    Target server.
  \item[Eventual Consistency]\label{item:eventual} \hfill \\
    In Chapter~\ref{ch:background}, we discussed about the relational databases
    constraints and some compromises that a NoSQL database have to make, in
    order to achieve better performance and scalability, mainly in distributed
    environments. CouchDB loosens the consistency checks that a traditional
    database would make, and makes it really simple to build distributed
    applications with huge performance improvements that would scale, by
    sacrificing immediate consistency. Maintaining consistency between multiple
    database servers is a common and complex problem with many books devoted to
    its solution. Sharding, master/slave replication, multi-master replication,
    and many more techniques are used to deal with it.

    CouchDB's way of address that problem is an incremental replication
    procedure along with a MVCC, which provides eventual consistency to the
    nodes involved. CouchDB operations stay within the context of a single
    document. Incremental
    replication is a process where document changes are periodically
    synchronized between different servers. The level the nodes will interact
    and communicate is decided by the user who sets up the replication. We can
    have from a cluster with independent nodes, up to a cluster where all nodes
    communicate and synchronize their databases. When two or more databases
    are synchronized from both directions, we may face a condition when the same
    document has been modified from both databases. This is a conflict, and
    CouchDB system comes with automatic conflict detection and resolution. What
    it does, is detect the document both changed, flag it as a conflicted
    document, and then decide which one version of the two is the ``winning"
    one. The winning version saved as the most recent version of the two, and
    the losing version also stays in the database as a previously edited
    version. The two databases make exactly the same choice in order to achieve
    consistency between them. The user can decide between three possible
    options. Leave the CouchDB choice as it is, revert it, or merge the two
    versions and generate a new one with the appropriate changes made.
\end{description}

\section{Detailed Design}\label{sec:des}

The changes we made, can be split into three main entities:

\begin{itemize}
  \item \emph{Core Changes}, that affect the software design.
  \item \emph{Feature Changes}, which do not have impact on the design but
        facilitate some cluster wide operations.
  \item \emph{Interface Changes}, user-level (\emph{CLI}) and Remote-API level
        (\emph{RAPI}) changes, which enable the features added.
\end{itemize}

\subsection{Core Changes}

One main core change will be extending Ganeti's design to support additional
solutions for the configuration and the job queue storage. A use case will be
Apache CouchDB along with the current disk implementation, an approach that
aims to increase the job queue throughput and allow Ganeti to scale even
better. Furthermore, the configuration data file will be divided into its most
heavily used components, in order to increase the resource utilization in bigger
clusters.

Besides these major changes, another ``core" change which will not be visible
to the users, will be abstracting a few python modules, and more specifically
those responsible for the job queue and configuration storage management, in
order to provide support for alternative storage solutions. This will allow
future flexibility in defining additional drivers by moving away from the
current static Ganeti approach which complicates and actually prohibits
additional storage options.

\begin{description}
  \item[Module Abstraction] \hfill \\
    The modules related to the configuration and job queue storage management
    that will be re-written in a more generic form, are the \texttt{config.py},
    \texttt{jqueue.py}, and \texttt{jstore.py}. In Python terms, we will convert
    the \texttt{\{config/jqueue\}.py} modules to packages, while the
    \texttt{jqueue.py} module will remain a single entity, with the appropriate
    changes applied. A package is simply an extension of the module mechanism to
    a directory. A detailed overview of those modules and how they restructured
    follows: \\

    $\bullet$ \emph{\textbf{{\Large{config.py}}}} \\

    \emph{\textbf{Overview}}

    The \texttt{config.py} module will be transformed into the \texttt{config/}
    package. The configuration related code will be split into smaller modules.
    The \texttt{config/\_\_init\_\_.py} file will contain the imports of the
    various sub-modules of the package, in order to expose their functionality
    to the clients that will make use of it. It is a necessity those
    abstraction changes be invisible to the rest Ganeti code and to the
    user as well. Any new storage type that is created and added, should not
    disturb the existing code. The most reasonable step was to use polymorphism
    and to create a common interface for all those types, which would separate
    the rest Ganeti functionality from the knowledge of the specific type it
    uses. The solution we adopted was to force the creation of new objects
    occur through a common \emph{Factory}~\flink{http://www.itmaybeahack.com/
    book/python-2.6/html/p03/p03c03\_patterns.html\#factory}, rather than to
    allow the creation code be spread all over Ganeti's code. With this
    approach, every new type designed is silently added to the factory and the
    new feature is available. The factory along with its auxiliary method and
    the appropriate lookup table for the configuration storage types are also
    contained in the \texttt{\_\_init.py\_\_} file. The relevant code is
    presented in Listing~\ref{lst:cfg_fact}.

    \includeminted[text]{../listings/cfg_fact.py}{%
     Factory method for the configuration storage objects}{%
       lst:cfg_fact}{linenos}

    \bigskip
    The next task was to identify the features and functions that should be
    extracted to each sub-module of the package. The \texttt{config.py} file
    will be renamed to \texttt{config/base.py}. Only the should-be-common
    classes and methods for all the implemented drivers will remain in that
    module. The rest class methods and functions that will not be contained in
    the base module, will be contained in a newly created file named
    \texttt{config/default.py}, which will provide the default disk storage type
    functionality for the configuration data file.

    \emph{\textbf{Technical Details}}

    The \texttt{config/base.py} file, will provide the basic interface to the
    rest configuration drivers. The ``globally" needed objects, from classes to
    functions and variables will appear in that module. The rest drivers will
    inherit~\footnote{Because Python is a dynamically typed language, it does
    not really cares about interfaces and types. All it cares about is applying
    operations to objects. The interface and inheritance keywords have different
    meaning in Python from other Object-Oriented languages such as C++ or Java,
    where it is often to inherit from a common interface. In Python, and as in
    our implementation, the only reason to inherit is to re-use the code in the
    base class.} the base functionality, which they have to extend depending on
    their specific needs.

    In the previously single configuration module, all the needed functionality
    were contained in a single class interface named \texttt{ConfigWriter}. This
    interface will be transformed to a more generic form, named
    \texttt{\_BaseConfigWriter} that will implement only a subset of the original
    class methods. We can define a sort of rule about the functions that will be
    implemented by the base module, and those by each driver. We can group the
    configuration methods to those that modify the cluster state, and those that
    simply query it. The second
    group of methods will be implemented by the base module, while the first one
    will not. Providing an example, a method that modify the cluster state
    like the \texttt{AddInstance}, is presented in Listing~\ref{lst:add_inst}.
    It is an empty method, that raises a \texttt{NotImplementedError} exception
    in order to indicate to the derived classes that it is required to be
    overridden. In addition, some method's definition were needed to be modified,
    in order to allow them to accept an arbitrary number of arguments depending
    on the driver's needs. The Python's \texttt{*args} and/or \texttt{**kargs}
    special syntax, facilitated those definition modifications.

    \includeminted[text]{../listings/add_inst.py}{%
      Implementation of the \texttt{base.AddInstance} method}{%
        lst:add_inst}{linenos}

    \bigskip
    The default disk driver class methods will be implemented by the
    \texttt{config/default.py} module. We will not analytically present any of
    the functions of that module because they correspond to the default Ganeti
    methods. For a detailed comprehensive view of the code, refer to the
    following Github link~\flink{https://github.com/dblia/nosql-ganeti}.
    Listing~\ref{lst:disk_cfg}, presents a part of the disk constructor method
    of the \texttt{DiskConfigWriter} class, which shows how the base
    functionality is inherited via the Python \texttt{super()} call, before any
    disk specific changes been made.

    \newpage
    \includeminted[text]{../listings/disk_cfg.py}{%
      Constructor of the \texttt{DiskConfigWriter} class}{%
        lst:disk_cfg}{linenos}

    \smallskip
    $\bullet$ \emph{\textbf{{\Large{jqueue.py}}}} \\

    \emph{\textbf{Overview}}

    The \texttt{jqueue.py} module that implements the job queue handling will
    also be re-written in more generic form. The new package will be named
    \texttt{jqueue/}, apparently. The \texttt{jqueue/\_\_init.py\_\_} file,
    will make all the appropriate package imports, and it will also contain
    the corresponding factory for the job queue storage needs. The factory is
    implemented in the same sense as the relevant one for the configuration
    package. The previously global job queue module will be renamed to the
    \texttt{jqueue/base.py} module, and the driver functionality for the default
    disk storage type will be implemented by the \texttt{jqueue/default.py}
    file, similarly to the configuration package.\\

    \emph{\textbf{Technical Details}}

    The ``rule" we defined for the configuration package about the methods that
    should remain in the base module and those that will not, similarly applies
    for that package. The \texttt{\_JobFileChangesWaiter} class, will be
    re-written in a more generic way, and will be renamed to the
    \texttt{\_BaseJobFileChangesWaiter} class. This class implements by default
    an \emph{inotify}~\flink{http://en.wikipedia.org/wiki/Inotify} manager using
    \emph{Pyinotify}, a Python module for monitoring filesystem changes
    \footnote{Pyinotify relies on a Linux Kernel feature (merged in kernel
    2.6.13) called inotify. inotify is an event-driven notifier, its
    notifications are exported from kernel space to user space through three
    system calls. Pyinotify binds these system calls and provides an
    implementation on top of them offering a generic and abstract way to
    manipulate those functionalities. (\emph{From Pyinotify Project's Wiki
    page}).}. Different drivers will apply alternative methods for notifying
    about the events happen in a job queue object, so this class will be
    overridden by every driver. Similarly, the \texttt{\_WaitForJobChangesHelper}
    class, which is a wrapper over the \texttt{\_JobFileChangesWaiter}, will be
    abstracted and renamed to the \texttt{\_BaseWaitForJobChangesHelper} class.
    The last class which must be transformed, is the \texttt{JobQueue} class,
    which is responsible for the job queue management. It will be renamed to
    \texttt{BaseJobQueue}, and it will only implement the common methods for the
    individual drivers.

    The \texttt{jqueue/default.py} module, inherits and extends the base module
    functionality depending to the disk driver needs. The
    \texttt{\_DiskJobFileChangesWaiter}, \texttt{\_DiskWaitForJobChangesHelper},
    and \texttt{DiskJobQueue}, are the corresponding implementations of the
    above-mentioned class transformations. This module will also contain the
    default \texttt{\_JobChangesWaiter} class, which is needed by the file
    notification mechanism only. In Listing~\ref{lst:diskqueue}, we present a
    small part of the constructor method of the \texttt{DiskJobQueue} class.

    \includeminted[text]{../listings/diskqueue.py}{%
      Constructor of the \texttt{DiskJobQueue} class}{%
        lst:diskqueue}{linenos}

    \smallskip
    $\bullet$ \emph{\textbf{{\Large{jstore.py}}}} \\

    \emph{\textbf{Overview}}

    In Listing~\ref{lst:diskqueue}, and more specifically in line 8, we observe
    that an instance of the jstore class is created. This is due to the
    abstraction of the \texttt{jstore.py} module, which is the last one that
    must be transformed. Many auxiliary handling functions for the job queue are
    located in this module. In contrast to the previous modules, we did not
    convert this one to a separate package, mainly due to its small length.

    \emph{\textbf{Technical Details}}

    The generic base jstore class will be renamed to \texttt{\_Base}, where all
    the original methods like the \texttt{ReadSerial}, or the
    \texttt{ReadVersion}, will be contained. The \texttt{FileStorage} class
    overrides them in order to provide the disk type functionality. The
    appropriate factory method was also created and added to that module, so
    that the user can chose silently the desired jstore object. \\
  \item[Configuration Data]\label{item:config} \hfill \\
    We already discussed about the main reasons why the current
    configuration data structure prevents Ganeti from achieving better resource
    utilization, particularly in bigger clusters. Ganeti is build for low-level
    VM management, so the most commonly used operations are those related to
    instance level modifications. Network and node related operations are also
    used quite often by cluster administrators. In bigger clusters where the
    single config file grows, a lot of congestion is observed and many
    should-be-fast operations, consume too much time in order to complete. The
    approach that is followed by the CouchDB driver, will try to remedy this
    limitation. In this section we will analytically discuss about the solution
    we designed, and how we applied it in the CouchDB driver.

    \textbf{Database Structure}

    We decided to separate the \texttt{config.data} file into
    its most heavily updated objects. Our claim was: \emph{Why should we flush
    the whole configuration file to disk, when we just want to update a single
    field, like modifying an instance parameter.} By dividing it properly,
    we could save a lot of operational time for the cluster.
    The CouchDB approach of handling the documents in a database facilitates
    that thought, and in fact gave us the opportunity to apply it in the CouchDB
    driver. All the changes we will make, are referred to the on disk
    representation of the configuration data. The in-memory
    representation should, and will, remain as it is, for compatibility reasons
    with the rest of the code.

    A database in CouchDB is a collection of documents, so we decided to
    separate the five primary components of the configuration file that are
    shown in Listing~\ref{lst:config}, into five distinct databases. As a
    result, we are introducing five new entities for the configuration storage;
    the \texttt{instances}, \texttt{nodes}, \texttt{nodegroups},
    \texttt{networks}, and the \texttt{config\_data} databases. With the new
    design, when we will want to add an instance, for example, we will add a
    single document in the \texttt{instances} database. This practically means
    that only few kilobytes will be serialized at a time instead of the whole
    configuration data file, which can become bigger than \emph{2 MB} in bigger
    clusters. The detailed structure of the \texttt{config.data} file and a
    complete database presentation follows:

    \begin{itemize}
      \item \emph{instances,} containing one document per instance object, and
      indexed (\texttt{\_id}) by the instance name.
      \item \emph{nodes,} containing one document per node object, and indexed
      by the node name.
      \item \emph{nodegroups,} containing one document per nodegroup object, and
      indexed by the uuid of the group.
      \item \emph{networks,} containing one document per network object, and
      indexed by the network name.
      \item \emph{config\_data,} containing one document in total, indexed as
      \texttt{config.data}. It will contain all the Ganeti \texttt{cluster}
      information, as well as the \texttt{\{c/m\}time, serial\_no,} and
      \texttt{version} fields. The \texttt{instances}, \texttt{nodes},
      \texttt{nodegroups}, and \texttt{networks} fields will remain in this
      document for compatibility reasons, and will contain an empty dictionary
      as a value.
    \end{itemize}

    Each of the objects above, must be updated to contain two extra necessary
    fields for the CouchDB needs; the \texttt{\_id} and \texttt{\_rev} fields.
    Since CouchDB has access control per database and even on document level,
    we will reserve the right to create private, and public databases, that will
    contain information which can be shared among the cluster administrators
    or third parties respectively.\\

    \textbf{Implementation Details}\label{sec:couch_details}

    Before we proceed to the detailed design of the CouchDB driver, we have to
    mention some internal technical details of the driver from a high-level
    point of view. While the \texttt{config.data} object will remain in memory
    as a unified entity, like it currently is, the on-disk representation will
    be totally different, as we explained above. With this approach all the
    modifications we will make will be invisible to the rest Ganeti code, and
    none function definition will have to change.

    Some aspects of that transformation must be clarified. The first one concerns
    the configuration loading from disk to memory. When the configuration loads
    from the CouchDB server must be unified, in order to construct the single
    Ganeti \texttt{objects.ConfigData} object, and achieve the compatibility we
    want to with the rest of the code. This is an easy task to do, and will
    introduce a quite small overhead during Ganeti reload, because Ganeti reads
    the configuration file from disk only when it starts, or reboots. The fact
    that Ganeti rarely reloads, makes that overhead negligible. The second
    aspect refers to the way that the updates will be flushed to the
    appropriate database. The previously globally used \texttt{\_WriteConfig}
    function will be converted and extended to dispatch any updates to their
    appropriate databases. Finally, the last modification concerns the
    replication procedure, which will be handled exclusively by the CouchDB
    server, removing this heavy task completely from Ganeti.

    While the CouchDB driver has been tested, and it seems to work smoothly, it
    brings an important change to the Ganeti configuration design. Since most of
    the Ganeti operations update more then one configuration object in a single
    call, we have to make more than one distinct updates in different databases.
    If we want to add an instance, for example, we also have to update the
    \texttt{serial\_no}, and the \texttt{mtime} fields of the config object that
    are located in a separate database. Most of Ganeti operations that modify
    the state of the cluster follow that update policy. While the cost of making
    two sequential updates of small objects is negligible comparing to the whole
    flushing of the configuration file to disk, the problem lies in the fact
    that the ACID property of a configuration transaction
    changes, since we may face a situation of a hardware failure where only
    one of the two updates has been completed. In the first version of the
    CouchDB driver, we do not deal efficiently with that case but it is intended
    to be fixed in a later version. Currently, we firstly update the global
    configuration fields, that are the serial number and the modification time
    of the file, and later the main object of the operation, ,i.e., an instance, a
    node, a network, or a nodegroup. If the failure occurs between the two
    updates, we will have a configuration file with an increased serial number
    by one, and an updated modification time, while the actual update have not
    been performed, because the operation failed. We believe that this is a
    small tradeoff we have to bear with for the \underline{first version of our
    driver} in order to achieve better resource utilization.\\

    \textbf{CouchDB Driver Design}

    Now we are ready to proceed with the detailed CouchDB driver design. The
    complete code of the CouchDB configuration driver, as the rest code of the
    project is hosted at Github~\flink{https://github.com/dblia/nosql-ganeti}.
    The top-level configuration management class, will be inherited and extended
    by the \texttt{CouchDBConfigWriter} class. Table~\ref{tbl:couch_cfg},
    presents the interface of the cluster configuration on CouchDB driver
    behalf. A presentation of the most important methods of the driver follows:

    \begin{table}[htbp]
      \begin{center}
      \begin{tabular}{l}
        \hline
        \multicolumn{1}{c}{\textbf{CouchDBConfigWriter Class}} \\
        \hline\hline
        \texttt{\_\_init\_\_(self, offline=False, accept\_foreign=False)} \\
        \texttt{IsCluster()} \\
        \texttt{AllocatePort(self)} \\
        \texttt{AddNodeGroup(self, group, ec\_id, check\_uuid=True)} \\
        \texttt{\_UnlockedAddNodeGroup(self, group, ec\_id, check\_uuid)} \\
        \texttt{RemoveNodeGroup(self, group\_uuid)} \\
        \texttt{AddInstance(self, instance, ec\_id)} \\
        \texttt{\_SetInstanceStatus(self, instance\_name, status)} \\
        \texttt{RemoveInstance(self, instance\_name)} \\
        \texttt{RenameInstance(self, old\_name, new\_name)} \\
        \texttt{AddNode(self, node, ec\_id)} \\
        \texttt{ RemoveNode(self, node\_name)} \\
        \texttt{MaintainCandidatePool(self, exceptions)} \\
        \texttt{\_UnlockedAddNodeToGroup(self, node\_name,
            nodegroup\_uuid)} \\
        \texttt{\_UnlockedRemoveNodeFromGroup(self, node)} \\
        \texttt{AssignGroupNodes(self, mods)} \\
        \texttt{\_OpenConfig(self, accept\_foreign)} \\
        \texttt{\_UpgradeConfig(self)} \\
        \texttt{DistributeConfig(self, node, replicate)} \\
        \texttt{\_WriteConfig(self, db\_name=None, data=None,
            feedback\_fn=None)} \\
        \texttt{SetVGName(self, vg\_name)} \\
        \texttt{SetDRBDHelper(self, drbd\_helper)} \\
        \texttt{Update(self, target, feedback\_fn, ec\_id=None)} \\
        \texttt{AddNetwork(self, net, ec\_id, check\_uuid=True)} \\
        \texttt{RemoveNetwork(self, network\_uuid)} \\
        \texttt{\_BuildConfigData(self)} \\
        \texttt{\_ClusterObjectPrepare(config\_data)} \\
        \hline
      \end{tabular}
      \end{center}
      \caption{Interface of the \texttt{CouchDBConfigWriter} class
        \label{tbl:couch_cfg}}
    \end{table}

    \bigskip
    $\bullet$ {\Large{\texttt{\_\_init\_\_()} method:}}

    This is the constructor method of the CouchDB driver. The relevant code is
    presented in Listing~\ref{lst:couch_init}. The code is quite
    straightforward; we create five new local class variables, one for each
    database we created, before we call the \texttt{\_OpenConfig} method
    which will construct the global \texttt{ConfigData} object. These
    variables are instances of the \texttt{couchdb.client.Database} class,
    actually a representations of the databases on the CouchDB server.

    \newpage
    \includeminted[text]{../listings/couch_init.py}{%
     Constructor of the \texttt{CouchDBConfigWriter} class}{%
       lst:couch_init}{linenos}

    \smallskip
    $\bullet$ {\Large{\texttt{\_OpenConfig} method:}}

    The \texttt{\_OpenConfig} method, is very similar to the default disk
    method. The main difference lies in the way that the document is loaded from
    disk. The default \texttt{utils.ReadFile} method, is replaced by the
    \texttt{\_BuildConfigData} method, which is responsible for the unification
    of the configuration databases to a single object. Listing
    \ref{lst:couch_buildconfig}, contains the relevant code.
    It was a challenge to collect all the documents from all databases in a
    small amount of time. The view mechanism provided by CouchDB, gave us the
    solution. The special \texttt{\_all\_docs} view, combined with the
    \texttt{include\_docs} attribute set to \texttt{True}, returns a listing of
    all the documents in a database, ordered by their \texttt{\_id}. We query
    that view on each database and construct the separate dictionaries of the
    instances, nodes, networks, and nodegroups of the cluster. Then we combine
    them to build the unified \texttt{ConfigData} object.

    \includeminted[text]{../listings/couch_buildconfig.py}{%
     Implementation of the configuration unify method of CouchDB}{%
        lst:couch_buildconfig}{linenos}

    \bigskip
    $\bullet$ {\Large{\texttt{\_WriteConfig} method:}}

    This method does not contain significant changes comparing to the default
    one, and it is unnecessary to provide its code. However, we have to mention
    two important differentiations from the default implementation. The
    \texttt{\_DistributeConfig} call have been completely removed, because
    CouchDB follows a different approach for the document replication, and it
    was not necessary any longer. In addition, we do not have to make a
    \texttt{json.dump} call to serialize the \texttt{config.data} object before
    we flush it to disk. Instead, we call the \texttt{utils.WriteDocument}
    function which is responsible for all the write requests to the CouchDB
    server. We will present the relevant CouchDB \texttt{utils} module in the
    upcoming paragraphs.\\

    $\bullet$ {\Large{\texttt{AddNode} method:}}

    Besides the primary methods for the configuration management that
    we presented earlier, we also chose to present the \texttt{AddNode} method.
    This is maybe the most representative method of the CouchDB driver, because
    it denotes the way we handle the writes, and how we setup the replication
    between the master candidate nodes. It is also a method of two object
    updates on different databases in a single operation. If we carefully take a
    look in Listing~\ref{lst:couch_nodeadd}, we observe that if the node is a
    master candidate, we call the \texttt{\_UnlockedReplicateSetup} function
    from the \texttt{utils/couch.py} module. This call enables the replication
    between the master and the new candidate node, and we do not longer have to
    worry about it. It will replicate any new changes made to the master
    databases to the candidate nodes database respectively. The last
    modification concerns the \texttt{cluster} dictionary updates in the
    \texttt{config\_data} database. As the configuration is in memory as a
    single \texttt{ConfigData} object, we surely do not want to flush the whole
    object to disk. Instead, we call the \texttt{\_ClusterObjectPrepare} method
    which actually clears the \texttt{instances}, \texttt{nodes},
    \texttt{nodegroups}, and \texttt{networks} dictionaries before flushing the
    updates of the \texttt{cluster} object to the \texttt{config\_data}
    database.

    \includeminted[text]{../listings/couch_nodeadd.py}{%
     Implementation of the \texttt{AddNode} method of CouchDB}{%
        lst:couch_nodeadd}{linenos}

  \item[Job Queue] \hfill \\
    A job is the only way to modify the cluster state in Ganeti. During its
    lifetime, a job interacts many times with the disk, as it passes from the
    several states of its execution. The interaction involves the
    information of the on-disk representation of the job, to correspond to the
    latest updates. The policy of Ganeti obliges every local update of a job to
    be spread among the master candidate nodes. That necessity, creates a
    bottleneck and reduces the throughput of the job queue and consequently the
    execution of jobs, specifically when many jobs are sent concurrently to the
    master I/O thread for execution.

    Our objective, is to take advantage of the CouchDB replication process, and
    write a driver without the need to replicate jobs inside Ganeti.
    We believe that splitting some of the work ,i.e., the replication task, to a
    separate thread, not tied with the Ganeti code path, will increase the job
    queue throughput. This actually means that in bigger clusters with a lot of
    congestion due to the number of clients who concurrently talk to the master
    daemon, the job execution will also be fasten up because jobs will go to the
    \texttt{Queued} state earlier than currently are and the waiting threads
    will grub them sooner too. In addition, the timeouts happen to the LUXI
    server due to a heavy loaded master daemon will be limited.

    \textbf{Database Structure}

    The structure of the database does not differ a lot from the original disk
    representation. We will create two new databases, named \texttt{queue} and
    \texttt{archive}, for the job queue and the archive directory respectively.
    In more details:

    \begin{itemize}
      \item \emph{queue}, containing one document per job, and indexed by the
      job identifier.
      \item \emph{archive}, containing one document per archived job, and
      indexed by the job identifier too.
    \end{itemize}

    The \texttt{\_QueuedJob} class, which corresponds to the in-memory
    representation of the job, already contains an \texttt{id} parameter which
    will be used for the CouchDB index needs ,i.e., \texttt{\_id} parameter. We
    will add an extra \texttt{rev} field that will be used as the job's version,
    and will be \texttt{None} by default if the job have not been written
    to disk yet. Otherwise it will contain the last revision number of the job.

    \textbf{CouchDB Driver Design}

    The document containing a job, will have three fields; the common
    \texttt{\_id}, and \texttt{\_rev} fields and a new one named \texttt{info},
    which will contain the \texttt{\_QueuedJob} instance. The same will apply
    for the documents in the archive database, but with the addition of an extra
    field named \texttt{archive\_index}, which corresponds to the original
    classification of the archived directory per \emph{10.000} jobs. With a
    simple query of a view in the archived database, we could easily fetch the
    desired jobs from the given range.

    For the CouchDB driver needs, three main classes should be created; the
    \texttt{\_CouchDBJobFileChangesWaiter}, the
    \texttt{\_CouchDBWaitForJobChangesHelper}, and the \texttt{CouchDBJobQueue}
    classes. We are going to present the main attributes of each one of those
    classes. For deeper investigation, the code is hosted as Github
    \flink{https://github.com/dblia/nosql-ganeti}. The \texttt{CouchDBJobQueue}
    class interface, the related class for the queue management for the CouchDB
    driver, is presented at Table~\ref{tbl:couch_jqueue}:

    \begin{table}[htbp]
      \begin{center}
      \begin{tabular}{l}
        \hline
        \multicolumn{1}{c}{\textbf{CouchDBJobQueue Class}} \\
        \hline\hline
        \texttt{\_\_init\_\_(self, context)} \\
        \texttt{\_InspectQueue(self)} \\
        \texttt{AddNode(self, node)} \\
        \texttt{RemoveNode(self, node)} \\
        \texttt{\_UpdateJobQueueFile(self, data, job)} \\
        \texttt{\_RenameFilesUnlocked(self, arch\_jobs, del\_jobs)} \\
        \texttt{\_NewSerialsUnlocked(self, count)} \\
        \texttt{\_GetJobIDsUnlocked(self, archived=False)} \\
        \texttt{\_GetJobsUnlocked(self, archived=False)} \\
        \texttt{\_LoadJobUnlocked(self, job\_id)} \\
        \texttt{\_LoadJobFromDisk(self, job\_id, try\_archived,
            writable=None)} \\
        \texttt{\_UpdateQueueSizeUnlocked(self)} \\
        \texttt{SetDrainFlag(self, drain\_flag)} \\
        \texttt{UpdateJobUnlocked(self, job, replicate=True)} \\
        \texttt{WaitForJobChanges(self, job\_id, fields, prev\_job\_info,
            prev\_log\_serial, timeout)} \\
        \texttt{\_ArchiveJobsUnlocked(self, job\_list)} \\
        \texttt{ArchiveJob(self, job\_id)} \\
        \texttt{AutoArchiveJobs(self, age, timeout)} \\
        \hline
      \end{tabular}
      \end{center}
      \caption{Interface of the \texttt{CouchDBJobQueue} class
        \label{tbl:couch_jqueue}}
    \end{table}

    We are about to present the methods of the \texttt{CouchDBJobQueue} class
    with the greatest importance. The relevant code will be presented, where it
    is necessary.

    \bigskip
    $\bullet$ {\Large{\texttt{\_\_init\_\_()} method:}}

    From the constructor method, we distinguish the creation of two new local
    class variables; the \texttt{self.\_queue\_db} and \texttt{self.\_archive}
    variables, containing the relevant \texttt{couchdb.client.Database}
    instances, similarly to the configuration driver.

    \bigskip
    $\bullet$ {\Large{\texttt{\_UpdateJobQueueFile method:}}}

    This is the method that writes a job to the database. Actually it is a
    wrapper method over \texttt{utils.WriteDocument}, which is the
    responsible method for the write requests, as we said in the configuration
    driver section. If a job is written for the first time, \texttt{\_rev}
    field is None, we create a document with a empty \texttt{\_rev} field. If
    a modification made in a job in the \texttt{queue} database, besides the
    data and the \texttt{\_id} field, we should also provide the
    \texttt{\_rev} field of the document we are about to change. This stands for
    all document updates in the CouchDB, because as we said CouchDB uses a
    \emph{MVCC} policy, and the right document version must always be provided
    in order to avoid conflicts. The \texttt{rev} attribute we added in the
    \texttt{\_QueuedJob} class, keeps the last revision number of the job and it
    is always updated with the most recent version of it, after each successful
    write. This speeds up the write requests, otherwise we should first
    fetch the document from the database to get the right revision, before we
    update it, which would introduce a great overhead to the whole operation.
    The same policy used for all the documents in the \texttt{queue} database,
    like the \texttt{serial} and \texttt{version} files.

    \bigskip
    $\bullet$ {\Large{\texttt{\_Get\{Job/ID\}sUnlocked methods:}}}

    These two methods return all the jobs from the \texttt{queue} database, and
    the \texttt{archive} one if requested, and all the job identifiers
    respectively. Consequently, we want a fast way to retrieve all the documents
    of those databases. We made use of the view mechanism again. We
    defined a view which returns only the jobs of the database where it runs,
    and not other documents like the \texttt{serial} or the \texttt{version}
    ones. The fact that a view runs only in the newly inserted documents and not
    in those already ran, speeds up that operation. The view, as we said in the
    Apache CouchDB~\ref{sec:couch} section, must be defined inside a
    \texttt{\_design} document. In our case, this is the
    \texttt{\_design/queue\_view} document, and contains a single view named
    \texttt{jobs}. This document is presented in Listing~\ref{lst:queue_view}.
    If we query the view with the \texttt{\_included\_docs} option set to
    \texttt{True}, it will return us a list with all the documents of the
    database. If that value is set to \texttt{False} we will get only the job IDs
    and no other information.

    \includeminted[text]{../listings/queue_view.txt}{%
      View in CouchDB for job retrieval from the \texttt{queue} db}{%
        lst:queue_view}{linenos}

    \bigskip
    $\bullet$ {\Large{\texttt{AddNode method:}}}

    This method is responsible to enable the replication tasks for the
    \texttt{queue} and \texttt{archive} databases, between the master and the
    new candidates. To enable/disable the replication, we make a call to the
    \texttt{utils.UnlockedReplicateSetup} method, just like the relevant method
    from the configuration driver.

    \bigskip
    $\bullet$ {\Large{\texttt{\_RenameFilesUnlocked method:}}}

    This is the last method we are going to present from that class. It is used
    for the archival of the jobs of the queue. This method may need to archive
    hundreds of jobs at a time, so we made use of the bulk update feature of
    CouchDB, which updates the given list of documents using a single HTTP
    request. After we collected the documents to be archived, we simply pass
    them as input to the \texttt{update(documents, **options)} method, which
    performs the bulk update.

    \bigskip
    $\bullet$ {\Large{\texttt{Notify Manager:}}}

    A challenge we faced during the job queue driver design, was the
    implementation of the notifying manager we will use while waiting for
    changes in a job. We can not make use of the inotify manager because the
    jobs will be at the database and not at the filesystem. We decided to use
    the CouchDB \texttt{\_changes} feed, where all the changes made to a
    database are recorded. This feed is also used by the replicator process, a
    topic that we have extensively covered earlier in this chapter. Two classes
    involve in the waiting procedure. The
    \texttt{\_CouchDBWaitForJobChangesHelper} and the
    \texttt{\_CouchDBJobFileChangesWaiter} classes. Those two classes with the
    aid of the default \texttt{utils.Retry} function, provide the desired
    feature.

    The \texttt{\_CouchDBWaitForJobChangesHelper} class is almost identical with
    the original \texttt{\_WaitForJobChangesHelper} class, and it will not
    be presented. we will focus on the \texttt{\_CouchDBJobFileChangesWaiter}
    class which actually implements the waiter. Unlike the default class, this
    one consists of two methods only; the \texttt{\_\_init\_\_()} and
    the \texttt{Wait} method. The \texttt{\_changes} feed and the relative
    functionality provided by CouchDB have been used for the waiter
    implementation. The \texttt{\_changes} feed contains every single
    modification made on the database. It would be a great
    waste of resources to search the whole feed for a modification of a single
    job at a time. This is why we created a filter function, which only searches
    for changes in the job ID we want to. This filter function is named
    \texttt{job\_id}, and is contained in a design document just like the view
    functions. The relevant document is presented in Listing~\ref{lst:filter}.
    Another decision we have to make, was the way we will poll for results in
    the \texttt{\_changes} feed. The most appropriate choice was the
    \texttt{longpoll} feed with a timeout to close the feed if nothing has
    changed. It is a very efficient form of polling, which avoids the need to
    frequently poll CouchDB to discover nothing has changed. It does not run any
    requests if nothing changed, but as soon as a result appears the HTTP
    connection between CouchDB and the Ganeti client closes. The timeout that
    was selected is identical to the one used by the default inotify condition.
    The last parameter of the \texttt{\_changes} feed we have made use of, was
    the \texttt{since} parameter. By providing the last sequence number to that
    parameter, we are waiting for new notifications only, ignoring the old ones.
    The sequence number refers to the number of the updates that have been made
    to the database, because any new update generates a unique sequence number.
    The \texttt{last\_seq} value corresponds to the upcoming update. As
    presented in Listing~\ref{lst:waiter}, if an event happened
    (\texttt{have\_events["results"] == True}), we return the relative
    \texttt{\_QueuedJob} object to the client. Otherwise, if the result returned
    is \texttt{False}, the \texttt{utils.Retry} function will re-poll repeatable
    to the desired job until an event happens.

    \includeminted[text]{../listings/filter.txt}{%
      Filter function in CouchDB}{lst:filter}{linenos}

    \includeminted[text]{../listings/waiter.py}{%
      Waiting manager function of CouchDB on the job-ID given}{%
        lst:waiter}{linenos}

  \item[Utility module for CouchDB] \hfill \\
    In the configuration and job queue CouchDB drivers, we made a lot of
    references to the relevant \texttt{utils/couch.py} module. The creation of
    such a module was essential, in order to make the main code of the CouchDB
    driver clearer, and also separate the main functionality of the driver from
    the utility functions needed, for the interaction with the CouchDB server.
    In Table~\ref{tbl:utils}, we provide the interface for the \texttt{couch.py}
    utility module:

    \begin{table}[htbp]
      \begin{center}
      \begin{tabular}{l}
        \hline
        \multicolumn{1}{c}{\textbf{utils/couch.py module interface}} \\
        \hline\hline
        \texttt{URIAuth(user\_info, reg\_name, port)} \\
        \texttt{URICreate(scheme, auth, path="", query="", fragment="")} \\
        \texttt{DeleteDB(db\_name, host\_ip, port)} \\
        \texttt{CreateDB(db\_name, host\_ip, port)} \\
        \texttt{GetDBInstance(db\_name, host\_ip, port)} \\
        \texttt{UnlockedReplicateSetup(host\_ip, node\_ip, db\_name,
            replicate)} \\
        \texttt{MasterFailoverDbs(old\_master\_ip, new\_master\_ip,
            db\_name)} \\
        \texttt{WriteDocument(db\_name, data)} \\
        \hline
      \end{tabular}
      \end{center}
      \caption{Interface of the utility CouchDB module
        \label{tbl:utils}}
    \end{table}

    The \texttt{URIAuth} method, creates the authority value within a URI, while
    the \texttt{URICreate} method, returns a general universal resource
    identifier. The \texttt{\{Create/Delete\}DB}, \texttt{GetDBInstance}, and
    \texttt{WriteDocument} methods, have a quite straightforward usage.
    A special mention have to be made for the \texttt{UnlockedReplicateSetup}
    and the \texttt{MasterFailoverDbs} methods.

    The \texttt{UnlockedReplicateSetup} method is responsible for
    enabling/disabling the replication tasks between the master node and the
    master candidates. During a server restart, we want the replication tasks to
    survive and continue from where they left of. In order to achieve that, we
    create a special replication document in the \texttt{\_replicator} database.
    Those document persist a server restart and can only be modified by the
    database administrator. This is what this function does. Depending on the
    value of the \texttt{replicate} parameter, it creates or removes the desired
    replication document. The replication document contains four fields; the
    \texttt{source} URI, pointing to the master node, the \texttt{target} URI,
    pointing to the master candidate, the \texttt{create\_target} attribute set
    to \texttt{True}, to create the database in the target node if it does not
    exists, and the \texttt{continuous} attribute set to \texttt{True} to make
    the replication process run forever. A continuous feed stays open and
    connected to the database until explicitly closed, and changes are sent to
    the client as they happen ,i.e., in near real-time.

    The \texttt{MasterFailoverDbs} method, is called in case of a
    master-failover. In that case the replication documents from the old master
    node, are moved to the new master node. The replication documents must be
    removed from the old master, otherwise we will end up with a bi-directional
    replication process with unknown results. We should also take care of cases
    where the master CouchDB server is down, when the master-failover is
    requested. In that case, we re-create the replication tasks to the new
    master candidate, and as soon as the old master is up again, it is informed
    that it is not longer the master node, and the cluster administrator should
    run a \texttt{redist-conf} command, which will remove the unwanted
    replication tasks from the old master.
\end{description}

Summing up, we presented the \emph{Core} modifications made to the Ganeti code,
to support multiple driver solution for the configuration and job queue
management. Changes have also been made in other Ganeti parts, but will not be
further presented, because are out of the scope of that document. For more
details, and further investigation on the full changes list, refer to the
above-mentioned Github link.

\subsection{Feature Changes}

The main feature-level changes will be:

\begin{itemize}
  \item the extension of \texttt{LUClusterRedistConf} functionality.
  \item global cluster-level parameter.
\end{itemize}

\begin{description}
  \item[Redistribute Config] \hfill \\
    \textbf{Current State} \\
    Currently, \texttt{LUClusterRedistConf} triggers a copy of the configuration
    file to all master candidates and of the ssconf files to all nodes. It also
    distributes every additional file which is part of the cluster configuration
    such as the certificate files. This is a call which should not normally
    needed, but in some cases like an upgrade of the Ganeti software, or if the
    \texttt{verify} call complains about configuration mismatches, must be run to
    ``re-synchronize" the cluster status.

    \textbf{Proposed Changes} \\
    With CouchDB driver, we may end up with such configuration mismatches in the
    following scenario. If the current master node fails, and the CouchDB server
    can not be contacted, a \texttt{master-failover} must be run. When the old
    master becomes available again, he will be informed by the Ganeti master
    voting procedure that he is no longer the master node of the cluster. The
    problem lies in the remaining replication tasks, that where never removed
    during the \texttt{master-failover} operation, because the old master node
    was unreachable. The \texttt{redist-conf} which would normally update its
    configuration values, will be extended to also remove the remained
    replication tasks, in order to avoid a bi-directional replication. Such
    solution is implemented with an additional function call to the
    \texttt{LUClusterRedistConf} class, named
    \texttt{\_RemoveRemainedReplicationTasks}. This function will search about
    remaining replication tasks in the candidate's \texttt{\_replicator}
    databases, and will simply remove them if there are any. The Ganeti
    replication tasks have a specific \texttt{\_id} format ,i.e.,
    \texttt{from\_\emph{source}\_to\_\emph{target}}, and can be identified
    from different replication tasks for other purposes.
  \item[Global Cluster parameter] \hfill \\
    \textbf{Current State} \\
    Currently, the configuration data and job queue are stored to disk, by
    default. There is no alternative storage solution provided. It would be nice
    for Ganeti users to have the ability to chose between several cluster-wide
    parameters the type of the underlying storage solution they would like to
    use.

    \textbf{Proposed Changes} \\
    The pluggable modular driver transformation of Ganeti's base components
    which we attempted, enables that feature. We added a new cluster-level
    parameter, which will modify the underlying storage type of Ganeti per
    will. This parameter will be kept into the \texttt{cluster} dictionary of
    the configuration data, which allow us to create generic instance
    ``classes" for the configuration and job queue storage handling. The
    default value will remain the disk storage.
\end{description}

\subsection{Interface Changes}

There is a single area of interface changes to expose the designed solution:

\begin{itemize}
  \item \emph{Command Line Interface}-level changes, \emph{CLI}.
\end{itemize}

\begin{description}
  \item[Command line changes] \hfill \\
    The new Ganeti feature we designed, introduces modifications in the way CLI
    arguments are handled. The command \texttt{gnt-cluster} will be
    modified and extended to allow setting, and changing the default parameter
    of the underlying storage type. The \texttt{init} command will be extended
    and a new command line argument will be added named
    \texttt{--backend-storage}, or \texttt{-S} for abbreviation. The default
    value will be \texttt{``disk"}, but user can also choose \texttt{``couchdb"}
    as a second alternative choice.

    The generic syntax of the cluster init command is presented in Listing
    \ref{lst:cli}.

    \newpage
    \includeminted[text]{../listings/cli.txt}{%
      Extension of the \texttt{gnt-cluster init} operation}{%
        lst:cli}{frame=single}
\end{description}

\chapter{Performance Evaluation}\label{ch:performance}

In this Chapter, we will evaluate the performance of the new CouchDB driver
comparing to the default disk implementation, that is currently used by Ganeti
for its storage requirements. Good benchmarks are non-trivial; each driver is
different, and different use cases need to tune different parameters. In next
sections, we will try to illustrate the benchmarking methodology of the diagrams
we are about to present, before we proceed to the detailed explanation of our
results.

The structure of this chapter is the following. Section~\ref{sec:specs} provides
details about the hardware and software on which we conducted our benchmarks,
while section~\ref{sec:bench} concentrates on the methodology behind our
measurements; the main factors we have taken into account in order to
decide which were the most interesting and applicable for Ganeti fields, were we
should evaluate our driver's performance. Finally, section
\ref{sec:perfom_couch} contains the actual performance evaluation of the CouchDB
driver. In each of its subsections concentrates on a specific field of interest
for the driver evaluation, which will subsequently lead us to the final
conjecture about how the driver responds to real-world workloads.

\section{Specifications}\label{sec:specs}

To evaluate our software tool, it was required to setup a Ganeti
cluster where our benchmarks would run. We decided to setup our testing cluster
into a virtualized environment. More specifically, we chosen \emph{Synnefo}
\flink{http://www.synnefo.org/}, an open source cloud software, to host our
testing environment. A bunch of reasons lead us to this decision. Firstly, as
Ganeti is a software tool for managing clusters of physical nodes, it is not a
facile task to obtain, setup, and maintain physical machines for the cluster
requirements. On contrast, using a virtualized environment makes it quite easy
to add and remove nodes from the cluster at will, without interacting with any
physical processes that would be running in a physical machine. It
provides a complete isolated environment where no one else can intervene with
our work. In addition, keeping up-to-date snapshots of our virtual nodes makes
disaster recovery quite a bit easier, and our cluster can be recovered in just a
few seconds in case of a hardware or a software failure. Moreover, it makes
incredibly simple and fast to modify the underlying hardware we use for our VMs,
through the hardware abstraction it is provided.

On the contrary, every virtualization solution comes with an additional overhead
in terms of computation, networking, and I/O operations. The overhead incurred
by virtualization has been the focus of many performance studies in the past,
including numerous of general-purpose benchmarks. A short review on those,
indicates an overhead below \emph{5\%} on computation~\cite{xen_art}, below
\emph{15\%} on networking~\cite{xen_art, diagnosing}, while the parallel I/O
performance losses  due to virtualization has been shown to be below
\emph{30\%}~\cite{xen_hpc}, respectively. Recently, there also has been occurred
a burst on the research activity related to the performance of using virtualized
resources in cloud computing environments~\cite{montage, Iosup_anearly,
parallel}, that provide additional metrics of the effects of using the cloud
computing services for running many types of scientific tasks.

In our case, the testing environment has been setup in a \emph{7-node}
cluster, where each node was armed with a \emph{24-core AMD Opteron(tm) 6172}
processor at \emph{2.10 GHz}, with \emph{189 GB} of primary memory, and
\emph{3.7 TB} of storage, running on a \emph{SMP Debian GNU/Linux} with
\emph{3.2.0-4} kernel in \emph{64-bit} mode. The virtualization software used,
was \emph{QEMU 1.7.0} with the aid of the \emph{KVM} kernel module. When used as
a virtualizer, QEMU achieves near native performance by executing the guest code
directly on the host CPU. The redundancy on the physical resources of the
nodes we setup our cluster, provides 1:1 mapping between the CPUs and vCPUS, as
for the primary memory too. This results to a
minimum performance overhead because there is no overcommitment in the physical
resources at all. On the contrary, we can not come through the overhead during
I/O operations with the disk and the network usage, but keeping in mind that both
of our drivers interact with the disk and make use of the network resources,
that overhead is linearly applied to each of them and will not affect the final
results. The specifications of each one of the VMs constituting our virtual
\emph{5-node} Ganeti cluster, where we conducted our benchmarks, are the
following.

\begin{table}[htbp]
  \centering
  \begin{tabular}{ | l | l | }
    \hline
    Component & Description \\ \hline \hline
    CPU & 8 x QEMU Virtual CPU Version 1.7.0 \\
    \hline
    RAΜ & 8192 MB  \\
    \hline
    Disk & 80 GB \\
    \hline
  \end{tabular}
  \caption{Test-VM hardware specs}
  \label{tab:hw-specs}
\end{table}

\begin{table}[htbp]
  \centering
  \begin{tabular}{ | l | l | }
    \hline
    Software & Version \\ \hline \hline
    OS & Debian 7.1 Wheezy Base System \\
    \hline
    Linux Kernel & 3.2.0-4-amd64  \\
    \hline
  \end{tabular}
  \caption{Test-VM software specs}
  \label{tab:soft-specs}
\end{table}

\section{Benchmark methodology}\label{sec:bench}

Real benchmarks require real-world load. We will try to test our driver on
real-world examples under situations when hundreds of clients try to interact
with the master daemon concurrently, meaning that the masterd has to deal with
them properly. Ganeti is a distributed software tool. So it is a premise to
scale well and perform-fast, when used in real production environments with tens
of nodes on each cluster.

There are a plenty of attributes affecting the performance of distributed
systems and multiple ``knobs" we could turn on to make a system perform better
in one area, but affecting another area when doing so. A use case is the CAP
theorem that was discussed in section~\ref{item:cap}. If we want our system to
scale out, for example, there are three distinct areas to deal with; increased
read and write requests, and data. In addition, reducing latency for a given
system, affects concurrency and throughput capabilities. These two examples are
graphically illustrated in Figure~\ref{fig:compr}.
Orthogonal to those attributes, there are many more factors that affect a system
such as Ganeti, and more of the figures below can be drawed, that display
different features such as reliability, simplicity, availability, and more.
CouchDB is very flexible and gives us enough tools to create a system shaped to
suit many, \underline{but not all}, of our problems.

\begin{figure}[htbp]
  \begin{center}
    \makebox[\textwidth]{%
    \subfloat[Performance: Throughput, latency, or concurrency]{{%
      \includegraphics[width=0.35\paperwidth]{../figures/figure2.pdf}}}
    \qquad
    \subfloat[Scaling: read requests, write requests, or data]{{%
      \includegraphics[width=0.35\paperwidth]{../figures/figure3.pdf}}}}
    \caption{Compromises of distributed systems\label{fig:compr}}
   \end{center}
\end{figure}

Keeping the above in mind, the benchmarks we have executed can be roughly
classified into four main categories, all of which having their own specific
goals.

The first category, aims to expose the effect in the job submission rate, when
the number of the master candidate nodes increases. In
order to effectively measure the overhead that is introduced with an increase in
the \texttt{candidate\_pool\_size} of the cluster, we will sent concurrently
\emph{errored} jobs to the master node, and then we will measure the rate that
Ganeti adds them to the queue. We intentionally chose to sent errored jobs,
because we simply want to observe how the job enqueue throughput of Ganeti
responds in various candidate pool sizes, and not any other factors that may be
affected by that type of measurement. The behavior we will observe, will also
indicate a suitable size of master candidate nodes, to conduct the rest of our
tests.

The second category, is a performance comparison test between the two driver
implementations; the default disk implementation that Ganeti currently uses, and
the CouchDB one. The comparison concentrates on the job submission rate, by
sending concurrently \emph{errored} jobs of various sizes similarly to the
previous category, but using a constant value of master candidate nodes instead,
the one that was determined previously. The aim is to measure the performance
against a tough workload, explain the differentiations, if any, that are
observed on both of the drivers, and expose the factors that have the greatest
impact on each of our drivers.

The third category deals solely with the configuration data management. We
measure the performance of the configuration related write requests, on several
file sizes, and in clusters with various candidate pool sizes, to examine the
bottlenecks on having a single configuration file on big production clusters
mainly. We will investigate all the factors that are delaying the update of the
configuration file, on both of the drivers, and then we conduct a comparison
test among them.

Lastly, in the forth benchmark category, we position the CouchDB and the disk
driver in a real-world scenario; an attempt to create a bunch of instances and
compare the total execution time of those jobs on each implementation. We aim to
present the overall performance of each driver, and explain any differentiations
that may arise.

\section{Evaluating CouchDB}\label{sec:perfom_couch}

The overview of the benchmark methodology we provided, points out the various
situations along with different metrics and workloads where we tested our
implementations, in order to measure accurately their performance. The
abovementioned categories, are presented in more details in the upcoming
sections.

\subsection{Impact of the candidate pool size}\label{subsec:cand_size}

This category is a sort of an introductory section for the rest of our tests.
Ganeti maintains a set of master candidate nodes, those that also contain a copy
of the full cluster configuration, i.e., configuration and jobs files. The
existence of those nodes has a great impact in the overall performance of the
cluster, due to the fact that each modification in a disk configuration file
causes a copy of it file to the candidate nodes. Creating a cluster with no
master candidates at all is a risky attempt, because in case of a master
failure all the cluster information will be lost. On contrast, maintaining a lot
of master candidate nodes is a redundant waste of resources, as modifications
have to be replicated to more nodes. We would ideally want to find out the set
of master candidate nodes that fits a production environment, in terms of having
the less impact in the cluster performance, and reducing the probability of a
cluster failure.

In order to decide which is the most appropriate candidate pool size, we
proceed with the following scenario. We sent jobs concurrently to Ganeti with
varying candidate node numbers, and we measure the rate in which jobs are
submitted to the queue, for each case.
This metric is the job enqueue throughput, and denotes the average number
of jobs that are added to the queue per second. It is a representative metric
for our purpose, because every new entry to the job queue will also be
replicated to the master candidate nodes before the operation is declared as
successful. Since we are just
interested for the enqueue rate, we decided to submit jobs that will
never be executed and will be declared as \emph{errored}. An example of those
jobs is the modification of an instance that does not exist in the cluster. The
jobs will be normally inserted to the queue and replicated to the candidate
nodes, but when they will start their execution, they will immediately fail as
\emph{errored}.

Jobs have been sent to Ganeti in batches of \texttt{10}, \texttt{20},
\texttt{30}, \texttt{40}, \texttt{50}, \texttt{100}, \texttt{150}, and
\texttt{200} jobs. We ran that benchmark in a \emph{5-node} cluster consisting
of \texttt{none},
\texttt{one}, \texttt{two}, and \texttt{four} master candidate nodes, and the
whole procedure has been repeated ten times in total. Since Ganeti writes every
information to filesystem and then distributes it to the candidate nodes, there
is a lot of disk and network I/O interaction. As a result, we expect a short of
deviation in our sample data values because there are external factors that may
affect the performance. The ``outliers" values that may arise should also be
included in the final results, because are part of the Ganeti behavior.
Consequently, we believe that the \emph{mean} value of our distribution is the
most appropriate metric for our case, because is a metric that represents the
\emph{central tendency} of the distribution by taking into account the whole
data information.

The benchmark outputs are summarized in two figures. Figure~\ref{fig:mc_comp},
presents the total results in a normal \emph{line-points} plot style, while
Figure~\ref{fig:const_jobs} concentrates on the heavier workload of our
benchmark that are closer to a real-world environment, in a clustered
\emph{bar-graph} plot.

\begin{figure}[htbp]
  \begin{center}
    \includegraphics[width=1.0\maxwidth]{../figures/mc_comp.pdf}
    \caption{Job submission rate per number of candidates}
    \label{fig:mc_comp}
  \end{center}
\end{figure}

\begin{figure}[htbp]
  \begin{center}
    \includegraphics[width=1.0\maxwidth]{../figures/const_jobs.pdf}
    \caption{Job submission rate per number of candidates \#2}
    \label{fig:const_jobs}
  \end{center}
\end{figure}

\textbf{Performance Analysis}

There are several interesting points that we can conclude from these diagrams,
which we will address below. The most obvious observation is the significant
drop in the throughput performance, as we increase the master candidate
nodes of the cluster. Furthermore, in case of no master candidate nodes, the
distribution appears skewed characteristics and visible differentiations in the
throughput performance, something that is not observed when the candidate pool
size is increased. A closer explanation of those points follows.

In both figures, there is an obvious relationship between the job submission
rate and the number of master candidates that Ganeti maintains. As we stated in
the Caveats section, i.e., \ref{sec:caveats}, Ganeti writes jobs to disk and
then concurrently replicates them to the master candidates. The performance
slowdown of the replication process is displayed in these figures. With a single
master candidate node, we have a drop in the throughput of about \emph{8 times}
comparing to the case of no master candidates. An additional increase in the
candidate number, has no significant performance drop. This is an expected
behavior because Ganeti uses a multi-node RPC call to update the files in the
candidate nodes. This small dropdown is totally expected due to the checks that
Ganeti does after each RPC call, to get informed about the success, or not, of
the operation.

Another interesting finding from our results, are the ``outliers" that were
observed in the output sample data. When we ran the benchmarks in a cluster with
no master candidates, the data tended to have a more sparse behavior than in a
cluster with one, or more, candidate nodes. In order to make that variation
visible to the reader, we calculated the \emph{standard deviation} metric of our
distribution, which shows how much dispersion from the average value exists, and
we present it in Figure~\ref{fig:std} to justify our claim. The question that
may arise is: \emph{Why the deviation exists only in the case of a cluster with
no candidates}. The submission rate of Ganeti is the rate that the data are
written to disk, and replicated to the candidate nodes. Obviously, the disk and
network I/O are some of the factors that affect our results. As we know, Ganeti
writes the jobs to its queue, so when a worker thread grubs a job for execution,
it will subsequently update the job file in disk too. This causes
a lot of congestion in the job queue lock both from the master thread and the
job queue workers. The order that the workers grub jobs for execution causes
those differentiations in the throughput performance, in case of
no master candidates. On contrast, when the cluster contains master candidate
nodes, we have a totally smooth distribution with data around the mean value.
This behavior is justified by the replication process of the job files to the
candidate nodes. It is a quite time-consuming operation, that covers the
rest operations that are executed, and actually determines the result's form. At
this point, many types of measurements can be taken that will clarify those
claims and probably will expose more, but are out of the scope of this document
and this test category specifically, and we will not expand further.

\begin{figure}[htbp]
  \begin{center}
    \includegraphics[width=1.0\maxwidth]{../figures/std.pdf}
    \caption{Standard Deviation [σ] of the job submission rate}
    \label{fig:std}
  \end{center}
\end{figure}

Summing up, from the results we presented, we believe that having a cluster with
two candidate nodes is the best choice to conduct the rest of our benchmarks. It
is also a choice that reflects a real production environment because it provides
the appropriate backup degree in case of a hardware failure, and also does not
have great performance impact comparing to a cluster with a single candidate
node.

\subsection{Comparison of the job submission rate}\label{subsec:enqueue_rate}

To some extend, we already discussed about the job submission performance rate
of Ganeti's default disk storage implementation, and we explained some of the
drawbacks that start to appear as the candidate pool size increases. In this
category, we make use of exactly the same metrics that we used in the previous
one, with the difference that the tests are conducted in a cluster with a
candidate pool size equal to \emph{three} nodes, as we determined in the first
benchmark category.

It is the first category where we actually compare the two driver
implementations, and we explain the main factors that affect their performance.
The results are summed up in two figures. Figure~\ref{fig:comp}, contains the
comparison of the job submission rate between CouchDB and the disk storage type,
while Figure~\ref{fig:couch_comp} presents the comparison of the CouchDB driver
performing under different socket options.

\begin{figure}[htbp]
  \begin{center}
    \includegraphics[width=1.0\maxwidth]{../figures/comp.pdf}
    \caption{Comparison of the throughput performance}
    \label{fig:comp}
  \end{center}
\end{figure}

\begin{figure}[htbp]
  \begin{center}
    \includegraphics[width=1.0\maxwidth]{../figures/couch_comp.pdf}
    \caption{Throughput performance of CouchDB on various socket options}
    \label{fig:couch_comp}
  \end{center}
\end{figure}

\textbf{Performance Analysis}

Before we proceed with the interpretation of the diagram results, we have to
mention that the metrics we used, are exactly the same as in the first benchmark
category, and the final values correspond to the \emph{mean} value of every
distribution too.

Figure~\ref{fig:comp}, is the first performance comparison test that we
made between the two peers. The initial results look very promising. In a
\emph{5-node} cluster with two master candidate nodes, we have a speedup in the
job submission rate at about \emph{7-times} in the CouchDB driver, comparing to
the default disk storage implementation. The job submission rate has impact in
the overall job execution duration, as we will show in the last benchmark
category, and also reduces the timeouts happen in the LUXI server when many
clients try to submit jobs to Ganeti. We extensively talked about the reasons
that performance drops in a Ganeti cluster when jobs are saved to disk, in the
first test category. Now we are going to have a closer inspection in the
factors that prevent CouchDB from presenting similar behavior.

We discussed a lot about how Ganeti distributes the configuration files
to the candidate nodes, and we presented the performance impact of this
operation in the first category's figures. One of the main reasons we have
chosen CouchDB for Ganeti, is the replication feature, as it was discussed in
section~\ref{item:replication}. CouchDB is a database the replicates, and with
this term we mean that its fundamental function is to provide a simple, fast,
and convenient way to \emph{synchronize} two or more CouchDB databases.
Replication is handled completely by a separate process, external to Ganeti,
which listens on changes to the \emph{source} database, and replicates them to
the \emph{targets}. Obviously, the source refers to the master database, while
the targets are the databases of the candidate nodes. The \texttt{replicator}
process listens continually to the source's \texttt{\_changes} feed, and a new
modification to the source will immediately be replicated to the candidate
nodes. In order to ensure safety, CouchDB makes an \emph{fsync} call before a
\emph{201 Created} request is returned to the client. As soon as the
nodes are ``up" and running CouchDB will replicate, and there is no need to make
extra checks similar to the RPC checks that Ganeti has to make to find out that
the updates have reached a majority of the nodes, before declaring the
operation as successful. Instead, it is sufficient to check that the CouchDB
servers in the candidate nodes are accessible. This operation can be handled by
individual clients, independent to Ganeti that will not affect the performance,
and is an issue that is expected to be fixed in a future driver version, as we
will discuss in the Conclusion, i.e., Chapter~\ref{ch:conclusion}.

Besides the comparison test between the two implementations, we also make
a comparison test for the CouchDB drive,r on various socket options that CouchDB
provides, and have a great impact in the overall performance of the tool.
Figure~\ref{fig:couch_comp} shows this performance comparison test, on the
\texttt{delayed\_commits} attribute of CouchDB. When we set this attribute to
\texttt{True}, it is observed a further increase in the throughput performance
at about \emph{9-times} comparing to the default disk implementation, and at
about \emph{35\%} comparing to the CouchDB driver with this attribute set to
\texttt{False}. Delayed commits is probably the most important CouchDB
configuration setting for performance. When is set to true (the default),
CouchDB allows operations to be run against the disk without an explicit
\emph{fsync} call after each operation. Fsync operations take time in order to
complete, and calling them on each update limits the CouchDB performance for
sequential writers. It is clear that setting this option to true, opens a
window for data loss, because data are being kept in a write buffer and are
fsync-ed after a certain amount of time, or when the buffer is full. Ganeti is
an environment where we absolutely need to know when the updates have been
received, so we set this attribute to false, by default. The aim of this
test, is to expose an important setting of CouchDB that could be enabled
periodically, in several cases, like in a situation with an overloaded master
daemon, and then disabled at will. It is up to the cluster administrator to
measure the tradeoff between loosing some data in case of a hardware failure,
and the ``relief" that the performance improvement will bring to the cluster.

To achieve the results we presented for the CouchDB driver, another
important configuration option must be modified, related to the TCP buffering
behavior. This is the \emph{nodelay} option which must be set to \texttt{True},
in order to disable the \emph{Nagle's algorithm}~\flink{http://en.wikipedia.org/
wiki/Nagle\%27s\_algorithm}, which introduces an additional delay when using
keep-alive HTTP-connections. By setting this option to true, the
\emph{TCP\_NODELAY} option is turned on for socket, which means that even small
amounts of data sent to the TCP socket, like the reply to a document write
request, or reading a very small document, will be sent immediately to the
network. They will not be buffered hoping that it will be asked to send more
data through the same socket in order to transfer them all at once. The main
reason that this important option is disabled by default, is that the last
releases of CouchDB ships with a more recent version of the HTTP server library
\emph{MochiWeb}~\flink{https://github.com/mochi/mochiweb}, which by default sets
the \emph{TCP\_NODELAY} socket option to false.

\subsection{Comparison of the config.data performance}\label{subsec:config_perf}

The CouchDB driver, besides the alternative storage solution that provides to
the configuration files of Ganeti, also introduces a variation in the
way it handles the \texttt{config.data} file, as it was extensively discussed
in Section~\ref{sec:config}. The ultimate aim of this category is to compare the
performance of the two alternative implementations of handling the configuration
file, but before we reach to this point, we will investigate in deep all the
factors that affect the configuration file performance, and that were discussed
in Section~\ref{sec:caveats}.

In this category we will present three diagrams in total. The first two
of them [\ref{fig:total-cfg}, \ref{fig:couchdb}], one for each driver,
show the total execution duration of the \texttt{\_WriteConfig} method, and all
the sub-method calls that are been made. This is the responsible method for
applying the changes of the configuration file to the permanent storage, and
replicate them to the master candidate nodes.
Every operation that modifies the cluster state calls this function to
make the changes permanent. It is the most time consuming function related to
the configuration file, and has a great importance in the performance of Ganeti,
because is must be called with the \texttt{ConfigWriter} lock held in exclusive
mode, which starts to become a bottleneck when a huge number of jobs is in
execution. If we manage to reduce the time the lock is held by the workers, we
will also reduce some of the congestion in the config lock. The last diagram
[\ref{fig:comp-cfg}], is the actual performance comparison plot between the two
implementations.

The benchmarks of this section have been conducted on a cluster with a candidate
pool size equal to \texttt{one}, \texttt{three}, and \texttt{five} nodes,
respectively. In order to measure the performance of the \texttt{\_WriteConfig}
method, we intentionally increased the size of the \texttt{config.data} file,
from \texttt{100 KB} up to \texttt{5 MB}. A cluster with about \emph{2.000}
instances has a configuration file of around \texttt{5 MB}, which corresponds to
a real workload for a production environment. Our test concentrates on modifying
a parameter of a single configuration object. We chose an instance object as a
use case. Starting, restarting, or stopping an instance is a quite commonly used
operation, that while it aims to modify a single field of the instance object,
the whole configuration object is flushed to disk, and moreover the
\texttt{ssconf\_*} files are not affected; a variance that we do not want
to take into account in this test category. The test was repeated \emph{20}
times for each pair, and we find it appropriate to make use of the
\emph{trimmed mean} value of our distribution. The trimmed mean is a method of
averaging, that removes a small percentage of the largest and the smallest
values before calculating the mean. This method aims to reduce the effects of
the outliers on the calculated average, and stated as mean trimmed by
\emph{X\%}, where \emph{X} is the sum of the percentage of observations
removed from both the upper and lower bounds. In our case, we trimmed the mean
by \emph{20\%}. The reason we did not calculate the normal mean value, is that
we wanted to reduce the effects of the outliers that were observed, and to
obtain a more accurate average performance overview, for both the
implementations.

\bigskip
\textbf{Performance Analysis}

In the begging of our analysis, we will take a closer inspection on all the
factors related to the performance of the write operation of the
configuration file. An operation that affects the configuration state, passes
from the following execution phases in general. Firstly, some preliminary checks
are being made on the object that it is requested to change, and then the
update of the in-memory representation of the \texttt{config.data} object
follows. Then the \texttt{\_WriteConfig} method is called, which flushes the
updates to disk, and replicates them to the candidate nodes. This method
consists of a number
of time-consuming sub-method calls that affect the overall performance of any
cluster operation that modifies the configuration file. These calls contain the
verification of the config object for configuration errors, to maintain the
consistency of the object, and is named \texttt{\_UnlockedVerifyConfig}, the
serialization of the config object in order to be prepared for applying the
changes to disk, through the \texttt{serializer.Dump} call, and then the actual
flushing of the in-memory object to disk, using the \texttt{utils.SafeWriteFile}
function call. Finally, the modifications have to be replicated to the candidate
nodes with the \texttt{\_DistributeConfig} method. All these operations are
affected from the size of the configuration file, and from the candidate pool
size as well. Figure~\ref{fig:total-cfg}, extensively examines this behavior.

From Figure~\ref{fig:total-cfg}, the following conclusions can be made. The most
time-consuming method, is the serialization of the configuration file. It
is a totally independent cost from the candidate pool size, but it is 100\%
bounded with the size of the file. The file is stored in
memory as a \texttt{ConfigData} object, a generic config object defined by
Ganeti. In order to be saved to disk, it must be transformed to a string format,
and this is the role of this method. The cost of the file replication to the
candidate nodes is increases along with the number of master candidates, and the
size of the file too. In the same sense, the verification check consumes a lot
of time in bigger file sizes because it traverses the whole file, same as the
time of the function that applies the changes to disk, but with the difference
that even in bigger file sizes it
does not consume a noticeable amount of time. From this diagram it is understood
that in a cluster with three candidate nodes, even from a file of \emph{1.0 MB}
in size, it takes at about \emph{1 sec} to complete a single operation.
If we combine it with the congestion on the config lock, we can see
the great impact of that delay in the overall cluster's performance.

It is obvious that a single configuration file comes with a number of
disadvantages. Modifying a single field of the file requires the serialization
of the whole config object and the distribution to the candidates, as well. This
approach reduces the cluster performance due to the increased operational cost
that is implied. In Figure~\ref{fig:couchdb}, we will present a different
approach that the CouchDB driver introduces, by conducting the same test on the
\texttt{\_WriteConfig} method of the CouchDB driver.

In Figure~\ref{fig:couchdb}, we observe a great improvement in the total
execution duration of updating the configuration file. This differentiation can
be justified by the different approach of CouchDB of handling the config object.
The configuration file has been separated to its sub-components, as
we extensively discussed in section~\ref{item:config}. Modifying an instance, a
node, and generally a single object of the configuration file, does not updates
the whole object to disk, but only the single object we want to.
Moreover, the serialization cost does not
implies anymore. The transformation of an object before it is written to disk
is a simple conversion to dictionary, and it is part of the
\texttt{utils.WriteDocument} method. Since we convert only few kilobytes each
time the cost is negligible. The distribution cost is also disappears due to
the different approach of CouchDB on handling the replication process, as we
already extensively explained. The verification cost is the only one that does
not changes, since we continue to verify the whole in-memory configuration
object.

The last diagram we are going to present in this category, is presented in
Figure~\ref{fig:comp-cfg}. It displays the total execution duration of an
instance modify operation. It is a representative job, as all Ganeti operations
modify a specific field of the configuration file. The same output would appear
if we were modifying a node, a network, or a nodegroup. As we said in the
\emph{Implementation Details} section of the CouchDB driver, i.e., Section
\ref{sec:couch_details}, every update causes two consecutive object updates to
the CouchDB server. One for the cluster general information, and one for the
object we want to change. In this figure, we present the total execution time
for CouchDB, that is the aggregated execution duration of the two objects.
Even with this overhead the difference of flushing the whole file to disk
comparing to the single object that is updated is significant. For example, for
a file of \emph{2 MB} in size, we have at about \emph{5-times} increase in the
performance of CouchDB, while the gap is widening as the file size further
increases.

\begin{figure}[htbp]
  \begin{center}
    \includegraphics[width=1.0\maxwidth]{../figures/total-cfg.pdf}
    \caption{Performance evaluation of the default \_WriteConfig method}
    \label{fig:total-cfg}
  \end{center}
\end{figure}

\begin{figure}[htbp]
  \begin{center}
    \includegraphics[width=1.0\maxwidth]{../figures/couchdb.pdf}
    \caption{Performance evaluation of the \_WriteConfig method of CouchDB}
    \label{fig:couchdb}
  \end{center}
\end{figure}

\begin{figure}[htbp]
  \begin{center}
    \includegraphics[width=1.0\maxwidth]{../figures/comp-cfg.pdf}
    \caption{Comparison of execution performance for instance modify ops}
    \label{fig:comp-cfg}
  \end{center}
\end{figure}

\subsection{Aggregate evaluation of the CouchDB driver}\label{subsec:total_eval}

Up to this point, we have tested our implementation in a variety of situations
that may occur in a Ganeti cluster. We also examined some of the main factors
that limit Ganeti from scaling and achieving better performance. In this last
benchmark category, we will attempt to measure the overall performance of
our drivers in a real-world scenario. In order to explain the results of this
section, we will also make use of the findings from the previous categories.

In a Ganeti cluster with \emph{5 vm-capable} nodes, and three master candidates,
we will concurrently submit jobs of \emph{OpInstanceCreate} opcodes, in batches
of \texttt{1}, \texttt{10}, \texttt{20}, \texttt{50}, and \texttt{100} jobs. We
will measure the average time of the phases that a job passes through, and then
the total execution duration from the first job that it is enqueued since the
last that is completed. Since the \texttt{Running} times of the jobs are
independent to the underlying storage layer that it is used, we will minimize
it by creating instances with \emph{1 GB} file disk, using the
\emph{--no-install}, \emph{--no-start} options that disable the OS installation
and the start-up of the instances respectively.

\bigskip
\newpage
\textbf{Performance Analysis}

Figure~\ref{fig:jobs_avg}, displays the average duration of the execution phases
of the \emph{InstanceCreate} jobs we submitted. For a short reminder about the
execution phases of a job, refer to Section~\ref{subsec:jobs}.

\begin{figure}[htbp]
  \begin{center}
    \includegraphics[width=1.0\maxwidth]{../figures/jobs_avg.pdf}
    \caption{Comparison of execution performance for the phases of a job}
    \label{fig:jobs_avg}
  \end{center}
\end{figure}

What we observe from Figure~\ref{fig:jobs_avg}, is a minimized \emph{Running}
time for reasons we already covered, while the most time is consumed in the
\emph{Waiting} phase. In this phase the jobs are waiting for locks, held by other
threads that are in execution. The \emph{Opportunistic locking} that it is used
since Ganeti version \emph{2.7}, improved the lock congestion in instance create
operations, but since we create a lot of instances in a small cluster, it is a
normal behavior. It is also observed that the average time of CouchDB in the
\emph{Queued}, and \emph{Waiting} phase is quite smaller comparing to the disk
implementation. We already covered the improved performance in the submission
rate of CouchDB. The new finding, is the increase in the average \emph{Waiting}
performance time. This behavior can be justified by the increased job submission
rate, as we presented in Figure~\ref{fig:comp}. The worker threads, are waiting
in the queue for new jobs to appear. As soon as a job is submitted in the queue,
and a worker thread is available, it grubs it for execution. An increased job
enqueue rate translates to workers that acquire their workload earlier. As a
result, we have an immediate impact in the \emph{Waiting} average time, due to
the fact that the workers are idle for less time than they previously were, and
the resources of the cluster are exploited more efficient than before. An
immediate consequence of the increase in the performance of the \emph{Queued}
and \emph{Waiting} time, is an increase in the total execution duration of the
jobs. This induction is justified by~Figure~\ref{fig:total_secs}.

\begin{figure}[htbp]
  \begin{center}
    \includegraphics[width=1.0\maxwidth]{../figures/total_secs.pdf}
    \caption{Comparison of the throughput performance for instance create ops}
    \label{fig:total_secs}
  \end{center}
\end{figure}

What we conclude from Figure~\ref{fig:total_secs}, is that the CouchDB driver
performs better under a heavy loaded environment. The performance gap between
the two implementations widens as the number of jobs in the cluster increases.
CouchDB is designed to service highly concurrent use cases, and perform under a
heavy application load. The \emph{Multi-Version Concurrency Control} that
implements, makes CouchDB able to handle a high volume of concurrent readers and
writers without conflicts to each other. As a result, there will not appear any
performance gaps as the cluster workload is increased, and the requests will
continue to be serviced efficiently.

\chapter{Επίλογος}\label{ch:conclusion}

\section{Συμπεράσματα}\label{sec:remarks}

Αυτό που εύκολα παρατηρεί κανείς είναι το γεγονός ότι η χρήσης της  απόσταση
\emph{ChiSquare} για τη σύγκριση των ιστογραμμάτων στη μέθοδό μας φαίνεται να
δίνει μεγαλύτερη ακρίβεια για όλες της βάσεις προσώπων που χρησιμοποιήσαμε.

Παρατηρούμε επίσης μια σημαντική διαφορά στην ακρίβεια της μεθόδου στη βάσε MyLucce.
Το γεγονός αυτό οφείλεται στο ότι η βάση MyLucce δημιουργήθηκε παίρνοντας τις εικόνες
των προσώπων από πραγματικό πολυμεσικό υλικό (βίντεο). Ως αποτέλεσμα δεν είχαμε καθόλου
επιρροή στις συνθήκες φωτισμού, στις εκφράσεις των προσώπων και στα χαρακτηριστικά
σε αντίθεση με τις άλλες βάσεις οι οποίες σχεδιάστηκαν ειδικά για σκοπούς πειραματισμού.
Θα μπορούσαμε όμως να θεωρήσουμε ότι η βάση MyLucce μας δίνει μια εικόνα για την
ακρίβεια της μεθόδου όταν προσπαθούμε να κάνουμε αναγνώριση χωρίς να έχουμε ένα
προεκπαιδευμένο σύστημα (Unsupervised Learning\flink{https://en.wikipedia.org/wiki/Unsupervised_learning})

Όσος αφορά την αξιολόγηση της τροποποιημένης μεθόδου, παρατηρούμε μια διαφοροποίηση
στη συμπεριφορά της ανάμεσα στη βάση AT\&Τ και τις υπόλοιπες. Στη βάση ΑΤ\&Τ φαίνεται
πως οποιαδήποτε αύξηση της τιμής του \emph{k} μειώνει την ακρίβειά της συνολικά.
Όμως όπως προαναφέραμε, λόγω των συνθηκών δημιουργίας της, η βάση ΑΤ\&Τ είναι
χρήσιμη για κάποια αρχικά πειράματα αλλά στην ουσία θεωρείται μια 'εύκολη' σχετικά
βάση. Επομένως δεν προσφέρεται για ασφαλή συμπεράσματα. Αντίθετα στις βάσεις Yale A
και Yale B ακόμη και σε ένα βαθμό και στην MyLucce παρατηρούμε ότι η αύξηση της τιμής
του \emph{k} δαιτηρεί σταθερή την ακρίβεια ή ακόμη και την αυξάνει μέχρι ένα σημείο
καμπής όπου από εκεί και πέρα η ακρίβεια της μεθόδου πέφτει δραματικά. Είναι λοιπόν
εμφανές ότι αρκετές φορές πρέπει να συμπεριλάβουμε περισσότερους από έναν κοντινότερους
γείτονες στη πρόβλεψη για να έχουμε ένα καλύτερο αποτέλεσμα.

Τέλος στα τελευταία γραφήματα επιχειρήσαμε και μια σύγκριση της μεθόδου με τη μέθοδο
FisherFaces ή οποία δεν βασίζεται στα local binary pattern histograms αλλά στην
Linear Discriminant Analysis \ref{sec:fisher}.
Η συμπεριφορά της FisherFaces είναι ίδια όσο αυξάνεται το \emph{k}. Όμως η ακρίβειά
της φθίνει πολύ πιο γρήγορα από εκείνη της μεθόδου μας. Αντίθετα,  στα μικρά \emph{k} η
ακρίβεια είναι σχετικά ίδια και για τις δύο μεθόδους.


Συμπερασματικά, θα λέγαμε ότι η τροποποίηση που πραγματοποιήσαμε εδώ κάποια ενδιαφέροντα
στοιχεία. Το βασικότερο είναι ότι η τιμή του \emph{k} όντως επηρεάζει την ακρίβεια
της μεθόδου και κατά περιπτώσεις την αυξάνει, μέχρι ένα σημείο καμπής. Γίνεται
λοιπόν εμφανές ότι αν θέλουμε να μειώσουμε τα false positives της μεθόδου LBPH πρέπει
η τελική πρόβλεψη να μην βασίζεται μόνο στον κοντινότερο γείτονα αλλά και στους υπολοίπους.
Πρέπει να γίνεται ξεκάθαρο σε κάθε βάση δεδομένων πιο είναι το σημείο καμπής της
όπου από κει και πέρα η ακρίβεια πέφτει δραματικά. Αυτό εξαρτάται από της συνθήκες
των προσώπων της κάθε βάσης. Και τέλος να σημειώσουμε το εύρος της διαφοράς στην ακρίβεια
που παρουσιάζει η αναγνώριση με μια βάση κατασκευασμένη σε ελεγχόμενες συνθήκες
και μια βάση δημιουργημένη με μη ελεγχόμενες.


\section{Προτάσεις για μελλοντική έρευνα}\label{sec:future}

Η ανίχνευση προσώπων και αντικειμένων καθώς και η αναγνώριση προσώπων, είναι
προβλήματα που απασχολούν ιδιαίτερα την επιστημονική κοινότητα τα τελευταία χρόνια,
με μεθόδους και τεχνικές να εμφανίζονται με εξαιρετικά μεγάλο ρυθμό. Ήδη από την
στιγμή που αρχίσαμε την εκπόνηση της εργασίας μέχρι σήμερα έχουν γίνει εξαιρετικά
βήματα προς την επίτευξη καλύτερων και πιο γρήγορων ανιχνεύσεων.

Ειδικότερα, οι σύγχρονες τεχνικές βασίζονται πλέον στα νευρωνικά δίκτυα. Αρχικά
χρησιμοποιείται ένα νευρωνικό δίκτυο για την εξαγωγή υποψήφιων θέσεων ενός
αντικειμένου και στη συνέχεια οι θέσεις αυτές ταξινομούνται στην κατάλληλη κλάση
και εξάγεται το αντίστοιχο παράθυρο με τη χρήση κάποιου άλλου συνελικτικού
κυρίως νευρωνικού δικτύου. Τα τελευταία χρόνια εμφανίστηκε η τάση της εξαγωγής
της κλάσης και του παραθύρου ταυτόχρονα με τον προσδιορισμό των υποψήφιων θέσεων
των αντικειμένων. Οι ανιχνευτές μονής λήψης όπως ονομάζονται φαίνεται να δίνουν
της κατεύθυνση της ανίχνευσης αντικειμένων στο μέλλον καθώς παρουσιάζουν αρκετά
καλή ακρίβεια σε μικρό χρόνο εκτέλεσης.

Παράλληλα, λόγω της αυξημένης χρήσης συσκευών περιορισμένης επεξεργαστικής ισχύς
εμφανίζεται η ανάγκη περιορισμού των απαιτήσεων που χρειάζονται οι παραπάνω μέθοδοι
τόσο σε μνήμη όσο και σε επεξεργαστική ισχύ. Το νευρωνικό δίκτυο MobileNet σχεδιάστηκε
με τέτοιο τρόπο ώστε η μέθοδος ανίχνευσης αντικειμένου που το χρησιμοποιεί να μπορεί
να τρέξει με μια συσκευή κινητής τηλεφωνίας. Επομένως φαίνεται να πως ενώ στον τομέα
της ακρίβειας του αποτελέσματος της ανίχνευσης έχει επιτευχθεί ένα ικανοποιητικό
ποσοστό, το μέλλον προϋποθέτει την προσαρμογή των μεθόδων σε συσκευές με μικρότερες
δυνατότητες όπως κινητά τηλέφωνα και tablets με τη χρήση 'ελαφριών' νευρωνικών
δικτύων.


\backmatter
\cleardoublepage % start at the next odd page
\phantomsection  % correct hyperlinking
\addcontentsline{toc}{chapter}{\bibname} % add bibliography section to toc
\bibliography{references}
\bibliographystyle{abbrv} % plain/abbrv/alpha/abstract/apalike/...
% \include{glossary}
% \chapter{Appendix}
% \printindex

\end{document}
